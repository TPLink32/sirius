%\chapter*{{\Large \textbf{9 Utilities}}}}
\chapter{Utilities}

\section{adios\_lint}

We provide a verification tool, called adios\_lint, which comes with ADIOS. It 
can help users to eliminate unnecessary semantic errors and to verify the integrity 
of the XML file. Use of adios\_lint is very straightforward; enter the adios\_lint 
command followed by the config file name.

\section{adios\_config}

This script provides the necessary compile and linking flags to use ADIOS in your 
application and the version information of the ADIOS installation. It also 
lists the write/read transport methods as well as the data transformation methods
and query methods available in the actual ADIOS installation. See Section 
2.5 for how to use it or run \texttt{"}adios\_config -h\texttt{"} to see the options. 

\section{adios\_index\_fastbit}

This tool is used to generate a FastBit index file on a .bp file. It will be a separate file, named 
<filename>.idx if the data file is <filename>.bp. The format of the index files is also the 
ADIOS-BP file format. If the index file is present, an ADIOS query will be evaluated by FastBit using this index, which can speed up the reading of data of interest by several orders of magnitude.

\vspace{6pt}
\noindent {\bf Note:} In ADIOS 1.8, this tool is a sequential code and tries to generate the whole index in one swoop. It may be very slow or buggy for a large dataset.


\section{bpls}

The bpls utility is used to list the content of a BP file or to dump arbitrary 
subarrays of a variable. By default, it lists the variables in the file including 
the type, name, and dimensionality. Here is the description of additional options 
(use bpls \-{}h to print help on all options for this utility).

\begin{itemize}
\item[-l]  Displays the global statistics associated with each array (minimum, maximum, 
average and standard deviation) and the value of each scalar. Note that the detailed 
listing does not have extra overhead of processing since this information is available 
in the footer of the BP file. 

\item[-t]  When added to the -l option, displays the statistics associated with the variables 
for every timestep. 

\item[-p] Dumps the histogram binning intervals and their corresponding frequencies, if 
histograms were enabled while writing the bp file. This option generates a ``\texttt{<}variable-name\texttt{>}.gpl'' 
file that can be given to the `gnuplot' program as input. 

\item[-a] Lists attributes besides the variables

\item[-A] Lists only the attributes

\item[-m] Print the visualization mesh definitions present in the file

%
%\item[-r] Sorts the full listing by names. Name masks to list only a subset of the variables/attributes 
%can be given like with the -ls command or as regular expressions (with -e option).
%

\item[-v] Verbose. It prints some information about the file in the beginning before listing 
the variables. 

\item[-S] Dump byte arrays as strings instead of with the default numerical listing. 2D 
byte arrays are printed as a series of strings. 

\item[-D] Show the decomposition of variables as written in parallel into file. 
\end{itemize}

Since bpls is written in C, the order of dimensions is reported with row-major 
ordering, i.e., if Fortran application wrote an NxM 2D variable, bpls reports it 
as an MxN variable. 

\begin{itemize}
\item[-d] Dumps the values of the variables. A subset of a variable can be dumped by using 
start and count values for each dimension with -s and -c option, e.g., -s ``10,20,30'' 
-c ``10,10,10'' reads in a 10x10x10 sub-array of a variable starting from the (10,20,30) 
element. Indices start from 0. As in Python, $-1$ denotes the last element of 
an array and negative values are handled as counts from backward. Thus, -s ``-1,-1'' 
-c ``1,1'' reads in the very last element of a 2D array, or -s ``0,0'' -c ``1,-1'' 
reads in one row of a 2D array. Or -s ``1,1'' -c ``-2,-2'' reads in the variable 
without the edge elements (row 0, colum 0, last row and last column).
\end{itemize}

Time is handled as an additional dimension, i.e., if a 2D variable is written several 
times into the same BP file, bpls lists it as a 3D array with the time dimension 
being the first (slowest changing) dimension. 

In the example below, a 4 process application wrote a 4x4 array (each process wrote 
a 2x2 subset) with values from 0 to 15 once under the name /var/int\_xy and 3 times 
under the name /var/int\_xyt. 

\begin{lstlisting}[language=bash,caption={bpls utility},label={}]
$ bpls -latv g_2x2_2x2_t3.bp 
File info:
  of groups: 1
  of variables: 11
  of attributes: 7
  time steps: 3 starting from 1 file size: 779 KB
  bp version: 1
  endianness: Little Endian
Group genarray:
  integer /dimensions/X scalar = 4
  integer /dimensions/Y scalar = 4
  integer /info/nproc scalar = 4
  string /info/nproc/description attr = "Number of writers"
  integer /info/npx scalar = 2
  string /info/npx/description attr = "Number of processors in x dimension"
  integer /info/npy scalar = 2
  string /info/npy/description attr = "Number of processors in y dimension"
  integer /var/int_xy {4, 4} = 0 / 15
  string /var/int_xy/description attr = "2D array with 2D decomposition"
  integer /var/int_xyt {3, 4, 4} = 0 / 15
  string /var/int_xyt/description attr = 
             "3D array with 2D decomposition with time in 3rd dimension"
\end{lstlisting}

The content of /var/int\_xy can be dumped with
\begin{lstlisting}[language=bash,caption={},label={}]
$ bpls g_2x2_2x2_t3.bp -d -n 4 var/int_xy
  integer /var/int_xy {4, 4} 
    (0,0) 0 1 2 3
    (1,0) 4 5 6 7
    (2,0) 8 9 10 11
    (3,0) 12 13 14 15
\end{lstlisting}

The ``central'' 2x2 subset of /var/int\_xy can be dumped with
\begin{lstlisting}[language=bash,caption={},label={}]
$ bpls g_2x2_2x2_t3.bp -d -s "1,1" -c "2,2" -n 2 var/int_xy
  integer /var/int_xy {4, 4} 
    slice (1:2, 1:2)
    (1,1) 5 6
    (2,1) 9 10
\end{lstlisting}

The last element of /var/int\_xyt for each timestep can be dumped with
\begin{lstlisting}[language=C,caption={},label={}]
$ bpls g_2x2_2x2_t3.bp -d -s "0,-1,-1" -c "-1,1,1" -n 1 var/int_xyt
  integer /var/int_xyt {3, 4, 4} 
    slice (0:2, 3:3, 3:3)
    (0,3,3) 15
    (1,3,3) 15
    (2,3,3) 15
\end{lstlisting}


\section{bpmeta}

The utility \verb+bpmeta+ is used to create the metadata file (.bp) after a large application run when the \verb+MPI_AGGREGATE+ output method  is used with \verb+have_metada_file=0+ option. 
The collection of metadata to present all data stored in many partial subfiles (under .bp.dir/) during \verb+adios_close()+ can be substantial at large scale (hundreds of thousands of cores) and may become the bottleneck in achieving the highest possible I/O performance. It can be turned off, and then only the data files under .bp.dir/ will be created. \verb+bpmeta+ can then be used on a single compute node or login node to generate the .bp metadata file. 

\begin{lstlisting}[language=bash,caption={},label={}]
usage: bpmeta [-t <T>] [-n <N>] <filename>

bpmeta processes <filename>.dir/<filename>.<nnn> subfiles and 
generates a metadata file <filename>.

  --nsubfiles | -n <N>   The number of subfiles to process in
                           <filename>.dir
  --nthreads  | -t <T>   Parallel reading with <T> threads.
                           The main thread is counted in.
\end{lstlisting}

If the number of files is not given as argument, \verb+bpmeta+ will use system calls to determine the number of files in the directory. The utility runs on a single node but it can use threads to speed up the metadata creation. 

\section{bprecover}

If you append multiple output steps into one dataset it may happen that the application aborts during writing and the whole dataset becomes corrupt. In such a case, the data up to the last output step is unharmed in the BP file but the index data is missing or corrupt along with the last output step. The \verb+bprecover+ tool is for recreating the index from parsing the file from the beginning and overwriting the corrupt last step and the index area (at the end of the file). It modifies (the end of) the file without copying the data. One must use the \verb+-f+ option to make this happen otherwise the tool just prints out what has found. 

This recovery tool only works on single BP files. If subfiles are created by an output method, one needs to recover each subfile individually, then use \verb+bpmeta+ to create the global metadata file. 

\begin{lstlisting}[language=bash,caption={},label={}] 
Usage: bprecover [-f | --force] <filename>
  -f:  do write the recovered index to the end of file
\end{lstlisting}


\section{bpdump}

The bpdump utility enables users to examine the contents of a bp file more closely 
to the actual BP format than with bpls and to display all the contents or selected 
variables in the format on the standard output. Each writing process' output is 
printed separately. 

It dumps the bp file content, including the indexes for all the process groups, 
variables, and attributes, followed by the variables and attributes list of individual 
process groups (see Listing~\ref{list-bpdump}).

\begin{lstlisting}[language=bash,caption={bpdump utility},label={list-bpdump}]
bpdump [-d var | -dump var ] <filename>
======================================================== 
Process Groups Index:
Group: temperature
    Process ID: 0 
    Time Name:
    Time: 1 
    Offset in File: 0
========================================================
Vars Index:
Var (Group) [ID]: /NX (temperature) [1]
    Datatype: integer
    Vars Characteristics: 20
    Offset(46) Value(10)
Var (Group) [ID]: /size (temperature) [2] 
    Datatype: integer
    Vars Characteristics: 20 
    Offset(77) Value(20)
...
Var (Group) [ID]: /rank (temperature) [3] 
    Datatype: integer
    Vars Characteristics: 20 
    Offset(110) Value(0)
...
Var (Group) [ID]: /temperature (temperature) [4] 
    Datatype: double
    Vars Characteristics: 20
    Offset(143) Min(1.000000e-01) Max(9.100000e+00) Dims (l:g:o): (1:20:0,10:10:0)
...
========================================================
\end{lstlisting}

