\subsection{Naming and Discovery Service}
\label{sec:naming-discovery}
\paragraph{Background:}
POSIX-compliant naming suffers from well documented scalability limitations due
to required operation serialization. Our approach would exascerbate this
problem because we decompose what would be a single file into potentially large
numbers of objects each broken into potentially multiple chunks to represent
data utility at a finer grained level to facilitate efficient storage usage.
Data migration, as a result of capacity pressures or user/system policies,
further adds to the naming management and discovery complexity.  Temporarily to
address performance requirements, we may make additional chunk replicas. In
some cases, such as with tape, having a highly compressed replica available in
a faster access medium near the tape system will offer efficient data probing
before forcing data retrieval from tape. In general, the performance replicas
will only exist as long as necessary to address performance requirements while
the tape overview version would serve as a metadata-like interface to the tape
archive. Since we must also support access through POSIX APIs for existing
applications and system integration, we must address how to bridge the drastic
differences between the objects and chunks of potentially varying fidelity with
a POSIX API-compatible interface.  Our naming service must maintain sufficient
metadata information to track or be able to find objects and chunks as they
move across storage tiers facilitating data discovery and be able to offer a
POSIX-compatible view for existing infrastructure integration.

The overriding theme for this proposal of providing a cooperative relationship
between the user and the storage system extends into this area as well.
Users should be able to understand the available data quality and approximate
retrieval times to enable making an appropriate trade-off for the current task.
Supporting this interaction requires additional metadata.  The naming service
must be able to both return a list of data chunks with their current
location and an indication of data quality. This will be combined with a
retrieval time estimate by the middleware/application to determine what version
should be retrieved.  This retrieval time estimate will be discussed elsewhere
in this proposal.

While, Hierarchical Storage Management (HSM) systems~\cite{blaze:1992:hsm}
seems similar to what we propose, that view only applies at the surface level.
HSM offers a strict caching approach for managing different storage capacities
trading off performance for capacity.  By maintaining a single namespace across
all tiers, it is possible to list a single directory view with files stored at
different tiers. HSM uses a whole file granularity. While this approach to
managing multi-tier storage works for whole files that fit on single logical
devices, it is far from ideal for scientific simulations. We will further
optimize the HSM approach to make the most efficient use of scarce storage
resources by only pulling the important/requested portions of a data set close
to the user. We propose to address this HSM limitation and support the required
complex naming and discovery metadata.

In addition to the basic naming and data tracking operations, we will also need
to incorporate authorization capabilities. Sirocco currently integrates with a
Kerberos service for authentication. Given a capability ticket, a user can
access different objects as needed. This ticket structure offers protection
services typically offered on POSIX directories and files, but can do it at the
fork level instead. This allows a reduced quality verion to be available
for the general users while the high-quality version would be limited to the
data creator. This and other considerations for security must be incorporated
into the entire naming and metadata service, particularly for
POSIX-compatibility.

While strongly encouraging data to stay in a particular storage tier,
ultimately it must be possible to flush this data to make room for other uses.
Sirocco offers a ``flush'' operation that forcibly migrates data to the
long-term resilience requirements, freeing space in the scarce resources. In
many, if not most cases, this will move data towards or actually on to a device
like tape, intended for long-term storage.

\paragraph{Approach:}
Our goal with this proposal is to offer an HSM-style capability but with a
finer granularity leveraging the inner structure and utility of scientific
data. It must handle data discovery efficiently for cases where data is spread
across several storage tiers or some of those pieces may have migrated based on
capacity pressures or user/system policies. Instead of a single file, such as
might be used to store an entire timestep output for a simulation, we will
demonstrate a metadata service with naming at the application variable (e.g.,
particle) level, but with an associated list of all of the variable pieces and
potentially different versions stored throughout the storage system (i.e., the
chunk list). By shifting to a finer-grained approach, we will enable more
effective use of fast yet small storage tiers and prioritize objects with high
importance.  With this shift to a partial variable granularity, a 1 PB output
with 500 GB of ``high interest'' data can limit this costly tier usage to just
500 GB, greatly improving storage availability in the fast tiers and focus
performance/capacity costs to get the most science from the platform.  By only
placing ``high interest'' data in the highest tier, we will effectively reduce
the pressure forcing data migration which hurts both writing performance and
the later data analytics required for generating scientific insights.  Sirocco
identifies a data segment using a container/object/fork/address tuple that can
have associated attributes. We will add a special attribute indicating that
this record is of ``high interest'' as well adapting a migration mechanisms to
take advantage of this attribute, if present, to strongly encourage keeping the
record in the highest storage tier as possible. 

We also must be able to locate or ``discover'' the data should it move.  The
challenge for discovery is that potentially, data will migrate from where it is
initially stored to a new location within the storage system.  Sirocco offers
an ability to search for data that has moved as well as forcing a particular
resilience or quality-based replica be the ``authoritative'' version migrating
the data to a particular location. We will investigate if the current Sirocco
functionality is capable of supporting the new operating modes we wish to
offer. Initial expectations suggest having bounded time guarantees for finding
data are critical for offering the QoS guarantees or even strong best effort
support.  Some potential approaches can build on CRUSH~\cite{weil:ceph} from
Ceph to offer a map of where to search for data sequentially. Given our
multiple storage tiers, we would extend this using some mechanism to shift to
the next tier to continue searching because the requested data never made it to
this location. This new work will be an expansion of Sirocco's currently
planned features.

We plan to address scalability problems by continuing with the LWFS and Sirocco
model. LWFS and Sirocco have taken an approach similar to the pure object
stores, but with a focus on the HPC setting. They have abandoned a fully POSIX
compliant metadata service as the default model in favor of a
container/object/fork/address tuple for identifying data similar to those used
for pure object stores. By having a service that addresses the object
collections that comprise a single entity, such as a variable or timestep
output, sufficient metadata is maintained to make the storage system usable
without additional heavy lifting by clients.  LWFS demonstrated a POSIX-style
namespace on the side kept in sync using a transaction process like
D2T~\cite{lofstead:2012:txn} showing that this alternative approach can support
traditional POSIX API calls even though the underlying storage system uses a
different model. Because we are not requiring the entire storage space be
addressable from a single tree root, we can offer multiple independent roots
using the Sirocco object storage. We will investigate how to make the naming
and discovery service scalable under the circumstances outlined above,
particularly to offer a coherent view given the potentially varying data
quality across different chunks.

\paragraph{Related Work:}
Structurally, the traditional POSIX naming service offering a hierarchical
space consisting of directories and files may be maintained for backwards
compatibility. However, this view will not offer the same strict semantics
POSIX defines. We must break these strict semantics to address scalability
problems forced by the serialized access to a single source for creating and
accessing files.  Several efforts~\cite{patil:2007:giga+,carns:pvfs} have
worked to reduce this contention by reducing the serialized
scope to a single directory or subtree or by allowing a single process in a
collective file operation talk with the metadata service and distribute a
handle to other participants. While these approaches help, they do not address
this key scalability limitation of serialized access - even if the metadata
service is spread across multiple nodes. The problem is the centralized listing
of where file stripes are stored or other, similar information. Instead, we
will offer a short duration to consistency POSIX view to offer the performance
and consistency requirements an exascale application demands.

Pure object stores, such as those popular in the big data
domain~\cite{Fitzpatrick:2004:memcached}, avoid this bottleneck by only
offering an object ID and offloading the ID to name mapping to the application.
Removing the metadata service completely from the system level can work well
for scale-out applications where data is created or consumed by a single
process at a time or just in a work queue rather than potentially O(1 million)
processes all actively requesting a single ``object''. To address this case,
having some system integrated metadata services to associate names with object
and chunk IDs is a preferable solution. We will offer solutions to address
legacy application and command-line style data access and migration to other
storage types using these POSIX semantics, without providing the same
synchronous consistency semantics as POSIX.

\paragraph{Challenge:}
The main challenge of incorporating the additional, rich metadata will be
joined by coordinating with the other storage and
application layer services to offer the best access times possible for data
stored in the system at available data quality levels. The developed metadata
services that drive data compression and subsetting operations must have easy,
consistent, and ideally non-blocking or locking access to this metadata
service. We must investigate how to build such a metadata and naming service
that also incorporates and maintains the additional metadata required to
support our advanced functionality.

Our research in this area will address the following specific questions:
(1) Since the metadata will be distributed and partitioned, how do we respond
to user requests within a specified time bound with accuracy? How do we
estimate the completeness of the response?, 
(2) For POSIX API requests, how do we respond if only varying data fidelity
levels are available for the requested data object?, 
(3) What level of metadata should we maintain and keep in sync to enhance
response quality within time-bound requests?, 
(4) How much time should be allowed for searching for metadata given the user
specified limits for data retrieval since the user will still have to spend the
data retrieval time from different devices? How do we integrate the read time
overhead into the metadata requests?, and
(5) How do we provide appropriate security while maintaining scalability,
particularly with the distributed metadata stores?

In this project we will first 
%\begin{tightEnumerate}
%
%\item {\bf Milestone 1}: 
demonstrate a metadata service capable of serving both POSIX
clients and our clients, but without consideration for authentication,
authorization, or data migration.
Secondly we will 
%\item {\bf Milestone 2}: D
demonstrate a time bounded search approach for finding data
within the storage system to identify data and the current location.  This will
update the cached information in the metadata service to short-circuit future
requests for the data presuming that it does not migrate again soon.
Finally, we will 
%\item {\bf Milestone 3:} D
demonstrate admission control and show scalability under
both application load as well as storage system pressures, showing that we can
tolerate both loads and maintain quality of service.

%\end{tightEnumerate}


%%% Local Variables:
%%% mode: latex
%%% TeX-master: "../proposal"
%%% End:
