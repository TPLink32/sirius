\chapter{Skel Command Reference}

Most skel commands are of the form:
skel
subcommand project
\section{Available Subcommands}

\begin{description}
  \item[skel install] \hfill \\
Usage:

{\tt skel install}
  \item[skel makefile] \hfill \\
Generates a Makefile fit for building and deploying the skeletal
application.

Requires <project>\_skel.xml and <project>\_params.xml

Usage:

{\tt skel makefile <project>}
  \item[skel params] \hfill \\
Generates a parameters file that can be customized by the user.
Note that this command creates a file called <project> params.xml.default
so as to avoid overwriting a customized parameters file. This means that
the user should copy this file to <project> params.xml and edit before
proceeding with code generation.

Requires <project>\_skel.xml

Usage:

{\tt skel params <project>}
  \item[skel replay] \hfill \\
Generates a C or Fortran code that performs the I/O operations
described by the given bp file or yaml file.

Usage:

{\tt skel replay <project> -y <yaml\_file>}

{\tt skel replay <project> -b <bp\_file>}
  \item[skel source] \hfill \\
Generates a C or Fortran code that performs the I/O operations
described by the XML descriptor and the parameters file.

Requires <project>\_skel.xml and <project>\_params.xml

Usage:

{\tt skel source <project>}
  \item[skel submit] \hfill \\
Generates a submission script for the skeletal application

Requires <project>\_skel.xml and <project>\_params.xml

Usage:

{\tt skel submit <project>}

  \item[skel xml] \hfill \\
Generates the <project>\_skel.xml file.

Requires <project>.xml

Usage:

{\tt skel xml <project>}
\end{description}

\section{Skel Utilities}
\begin{description}
  \item[skeldump (experimental)] \hfill \\
Extracts necessary metadata from a bp file to create a skeletal application.
Metadata is produced in yaml format, and is sent to standard out.

Usage:

{\tt skeldump <bpfile> > <yamlfile>}

\end{description}
