\chapter{The Parameters File}
Generation of a skeletal application requires some more information that is
found in the ADIOS configuration file. To specify this additional information,
the user must supply a Skel parameters file. The parameters file is an XML file
which contains the elements described in the following section.
Although it not overly dicult to construct your own parameters files, users
may find it more convenient to use the {\it skel params} command to automatically
generate a parameters file with default values, then simply edit the file to provide
the desired configuration.
\section{Elements}
\begin{description}
  \item[<skel-config>] \hfill \\
Each parameters file must contain exactly one
<skel-config> element as the only element at the root level of the document. The
<skel-config> element should contain one <adios-group> element and one <batch>
element, as described below.

{\it Supported Attributes:}
  \begin{description}
    \item[application]
The name of the application described by the document.
This should correspond to the project used for the skel calls, and
the name of the original XML file given to skel xml .
  \end{description}


  \item[<adios-group>] \hfill \\
The <adios-group> element is a child of the root <skel-config>
element. It corresponds to the adios-group that will be written by the I/O
skeletal application being generated. The <adios-group> element contains a collection of
<scalar> and <array> elements corresponding to the variables described for the group
in the ADIOS config file.

{\it Supported Attributes:}
  \begin{description}
    \item[name]
The name of the ADIOS group that this element describes
  \end{description}


  \item[<scalar>] \hfill \\
Represents a scalar variable that is described in the ADIOS descrip-
tor. Often integer valued scalars will be used to determine the dimensions
of arrays, thus providing appropriate values is essential for creating mean-
ingful benchmarks.

{\it Supported Attributes:}
  \begin{description}
    \item[name]
The name of the scalar variable
    \item[type]
(Optional) The type of this variable in the generated code. Supplied
for convenience by skel params .
    \item[value]
A value, of the proper type, that will be assigned to this scalar
variable in the generated code.
  \end{description}

  \item[<array>] \hfill \\
Represents an array variable that is described in the ADIOS descrip-
tor.

{\it Supported Attributes:}
  \begin{description}
    \item[name]
The name of the array variable
    \item[type]
(Optional) The type of this variable in the generated code. Supplied
for convenience by {\it skel params}.
    \item[dimensions]
(Optional) A list of the scalar values that will be used to
determine the dimensions of this array
    \item[fill-method]
Determines how the array memory will be initialized in the
generated code
  \end{description}
  \item[<batch>] \hfill \\
Describes a set of tests to be performed by the generated I/O kernel.
The <batch> element contains one or more <test> elements.

{\it Supported Attributes:}
  \begin{description}
    \item[name]
The name of the batch. Used to name the submission script and
other elements of the batch job.
    \item[cores]
The number of MPI tasks that will be used for the skeletal application
    \item[walltime]
Amount of runtime requested by the generated submission script
  \end{description}
  \item[<test>] \hfill \\
Describes a single test to be performed.

{\it Supported Attributes:}
  \begin{description}
    \item[type]
The type of test to be performed. Currently only write is supported.
    \item[group]
The ADIOS group to be written
    \item[method]
The ADIOS write method to use for writing (i.e. POSIX). A
full listing of available methods can be found in the ADIOS manual.
    \item[iterations]
The number of times to repeat this test
    \item[rm]
determines whether and when to remove the written output files. One
of \{pre, post, both, none\}
  \end{description}
\end{description}

