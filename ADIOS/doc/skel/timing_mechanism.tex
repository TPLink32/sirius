

\chapter{Low-Level Timing Mechanism}
By default, applications generated by skel will produce a summary timing re-
port, sending it to standard out. On most platforms, this will be captured in
the output file produced by the job script. The summary report contains tex-
tual information about the overall time taken by the various I/O operations. If
more detail is desired, ADIOS has a mechanism for gathering low-level timing
information for various events that occur within the ADIOS I/O calls.

\section{Using the Low-Level Timing Mechanism}
To use the low-level timing mechanism, you must use an ADIOS library that has
been built with this low-level timing mechanism enabled. Simply build ADIOS
as described in the ADIOS manual, inserting
--enable-skel-timing
in the
configure
command. We do not recommend enabling the low-level timing
mechanism while running production codes.
Once you have enabled low-level timing in ADIOS, you only need to enable
generation of low-level timing calls in your Skel settings file. This is done by
including the line:
{\tt use\_adios\_timing=yes}
in your Skel settings file. This will
cause the
skel source
command to include an additional call near the end
of the generated skeletal application to output the detailed timing information
that has been collected. The detailed timing information is written to a separate
file using XML. This will work best if used to write a single iteration of a single
group.
The low-level timing mechanism provides detailed timing information for
only some of the available write methods. As of this release, the supported
methods are
POSIX
,
MPI
LUSTRE
, and
MPI
AMR
.
\section{Extracting Timing Information}
The XML file that is produced by ADIOS contains a large amount of measure-
ment data, but it is somewhat unwieldy to work with directly. So, we have
included an additional utility,
skel
extract.py
, which allows data from the
XML file to be exported as a CSV file that is simple to load using tools such as R or Matlab.
