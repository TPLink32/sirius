\chapter{Introduction}

Skel is a tool used for the generation of I/O skeletal applications. Skel takes a
high-level description of the I/O performed by an application, and generates a full
benchmark that performs the described I/O.
Compared to a scientific application, the generated benchmark is easy to build
and run, and generally runs faster since it only performs I/O. Despite
the simplicity of the generated code, it duplicates
the I/O pattern of the target application, thus simplifying the process of
understanding the I/O performance of an application.

Skel provides a simple mechanism for testing the performance of I/O operations
that are relevant to applications of interest. This is of critical importance
when evaluating new systems or new system configurations. It is equally useful
for evaluating new I/O methods, or examining the effects of dirent parameters
to existing I/O methods. Since the benchmarks produced by skel focus
on the exact I/O patterns of applications, performance measurements obtained
from those benchmarks will correlate highly with the performance of the actual
applications.

This manual provides a detailed explanation of Skel, including the relevant
file formats, documentation of skel commands, common usage of skel, and hints
for porting skel to new platforms. Skel is young and still in development, so we
expect there will be things that do not work perfectly, as well as things that 
have changed from the previous release. Please help us improve by
letting us know when you encounter troubles. You can reach the skel developers
by sending email to lot@ornl.gov.

