\chapter{Developer Manual}

\section{Create New Transport Methods}

One of ADIOS's important features is the componentization of transport methods. 
Users can switch among the typical methods that we support or even create their 
own methods, which can be easily plugged into our library. The following sections 
provide the procedures for adding the new transport method called ``abc'' into 
the ADIOS library. In this version of ADIOS, all the source files are located in 
/trunk/src/; the core files in /trunk/src/core/, the write method in /trunk/src/write
and the read method in /trunk/src/read.

\subsection{Add the new method macros in adios\_transport\_hooks.h}

The first file users need to examine is adios\_transport\_hooks.h, which basically 
defines all the transport methods and interface functions between detailed transport 
implementation and user APIs. In the file, we first find the line that defines 
the enumeration type Adios\_IO\_methods\_datatype add the declaration of method 
ID ADIOS\_METHOD\_ABC, and, because we add a new method, update total number of 
transport methods ADIOS\_METHOD\_COUNT from 9 to 10.

1. enum Adios\_IO\_methods datatype 
\begin{lstlisting}[caption={Add a new write method, step 1}]

enum ADIOS_IO_METHOD {
    ADIOS_METHOD_UNKNOWN    = -2,
    ADIOS_METHOD_NULL       = -1,
    ADIOS_METHOD_MPI        = 0,
    ...
    ADIOS_METHOD_PROVENANCE = 8,

    `\color{javapurple}{\bf ADIOS\_METHOD\_ABC        = 9,}`

    ADIOS_METHOD_COUNT      = 10
};
\end{lstlisting}

2. Next, we need to declare the transport APIs for method ``abc,'' including init/finalize, 
open/close, should\_buffer, and read/write. Similar to the other methods, we need 
to add 

\begin{lstlisting}
    FORWARD_DECLARE (abc)
\end{lstlisting}

3. Then, we add the mapping of the string name ``abc'' of the new transport method 
to the method ID - ADIOS\_METHOD\_ABC, which has been already defined in enumeration 
type Adios\_IO\_methods\_datatype. As the last parameter, ``1'' here means the 
method requires communications, or ``0'' if not.

\begin{lstlisting}
    MATCH_STRING_TO_METHOD ("abc", ADIOS_METHOD_ABC, 1)     
\end{lstlisting}
        
4. Lastly, we add the mapping of the string name needed in the initialization functions 
to the method ID, which will be used by adios\_transport\_struct variables defined 
in adios\_internals.h.

\begin{lstlisting}
    ASSIGN_FNS (abc, ADIOS_METHOD_ABC)
\end{lstlisting}

\subsection{Create adios\_abc.c}

In this section, we demonstrate how to implement different transport APIs for method 
``abc.'' In adios\_abc.c, we need to implement at least 11 required routines: 

\begin{enumerate}
\item ``adios\_abc\_init'' allocates the method\_data field in adios\_method\_struct 
to the user-defined transport data structure, such as adios\_abc\_data\_struct, 
and initializes this data structure. Before the function returns, the initialization 
status can be set by statement ``adios\_abc\_initialized = 1.''

\item ``adios\_abc\_open'' opens a file if there is only one processor writing to 
the file. Otherwise, this function does nothing; instead, we use adios\_abc\_should\_buffer 
to coordinate the file open operations.   

\item ``adios\_abc\_should\_buffer,'' called by the ``common\_adios\_group\_size'' 
function in adios.c, needs to include coordination of open operations if multiple 
processes are writing to the same file. 

\item ``adios\_abc\_write'', in the case of no buffering or overflow, writes data 
directly to disk. Otherwise, it verifies whether the internally recorded memory 
pointer is consistent with the vector variable's address passed in the function 
parameter and frees that block of memory if it is not needed any more.  

\item ``adios\_abc\_read'' associates the internal data structure's address to the 
variable specified in the function parameter.

\item ``adios\_abc\_close'' simply closes the file if no buffering scheme is used. 
However, in general, this function performs most of the actual disk writing/reading 
the buffers to/from the file by one or more processors in the same communicator 
domain and then close the file. 

\item ``adios\_abc\_finalize'' resets the initialization status back to 0 if it has 
been set to 1 by adios\_abc\_init. 

If you are developing asynchronous methods, the following functions need to be 
implemented as well; otherwise you can leave them as empty implementation.

\item adios\_abc\_get\_write\_buffer,

\item ``adios\_abc\_end\_iteration`` is a tick counter for the I/O 
routines to time how fast they are emptying the buffers. 

\item ``adios\_abc\_start\_calculation'' indicates that it is now 
an ideal time to do bulk data transfers because the code will not be performing 
I/O for a while.

\item ``adios\_abc\_stop\_calculation`` indicates that bulk data 
transfers should cease because the code is about to start communicating with other 
nodes.
\end{enumerate}

The following is One of the most important things that needs to be noted: 

fd-\texttt{>}shared\_buffer = adios\_flag\_no,

which means that the methods do not need a buffering scheme, such as PHDF5, and 
that data write out occurs immediately once adios\_write returns.

If fd-\texttt{>}shared\_buffer = adios\_flag\_yes, the users can employ the self-defined 
buffering scheme to improve I/O performance.

\subsection{A walk-through example}

Now let's look at an example of adding an unbuffered POSIX method to ADIOS.  According 
to the steps described above, we first open the header file --``adios\_transport\_hooks.h,'' 
and add the following statements:

\begin{lstlisting}[emph={ADIOS_METHOD_POSIX_ASCII_NB}, emphstyle={\color{red}\large\bf},
                   caption={Example: add unbuffered POSIX method, step 1}]
enum ADIOS_IO_METHOD {

    ADIOS_METHOD_UNKNOWN        = -2,
    ADIOS_METHOD_NULL           = -1,
    ADIOS_METHOD_MPI            = 0,
    ...
    ADIOS_METHOD_PROVENANCE     = 8,
    // method ID for binary transport method
    ADIOS_METHOD_POSIX_ASCII_NB = 9, 
    // total method number
    ADIOS_METHOD_COUNT          = 10 
};

FORWARD_DECLARE (posix_ascii_nb);

MATCH_STRING_TO_METHOD ("posix_ascii_nb", ADIOS_METHOD_POSIX_ASCII_NB, 0)

ASSIGN_FNS (binary, ADIOS_METHOD_POSIX_ASCII_NB)
\end{lstlisting}

Next, we must create adios\_posix\_ascii\_nb,c, which defines all the required 
routines listed in Sect. 12.1.2 The blue highlights below mark out the data structures 
and required functions that developers need to implement in the source code. 

\begin{lstlisting}[emph={ADIOS_METHOD_POSIX_ASCII_NB}, emphstyle={\color{red}\large\bf},
                   caption={Example: add unbuffered POSIX method, C source of write method}]
static int adios_posix_ascii_nb_initialized = 0;

struct adios_POSIX_ASCII_UNBUFFERED_data_struct
{
    FILE *f;
    uint64_t file_size;
};

void adios_posix_ascii_nb_init (const char *parameters, 
                                struct adios_method_struct * method)
{
    struct adios_POSIX_ASCII_UNBUFFERED_data_struct * md;

    if (!adios_posix_ascii_nb_initialized)
    {
        adios_posix_ascii_nb_initialized = 1;
    }

    method->method_data = malloc (
            sizeof(struct adios_POSIX_ASCII_UNBUFFERED_data_struct));
    md = (struct adios_POSIX_ASCII_UNBUFFERED_data_struct *) 
            method->method_data;
    md->f = 0;
    md->file_size = 0;
}

int adios_posix_ascii_nb _open (struct adios_file_struct * fd, 
                                struct adios_method_struct * method)
{
    char * name;
    struct adios_POSIX_ASCII_UNBUFFERED_data_struct * p;
    struct stat s;

    p = (struct adios_POSIX_ASCII_UNBUFFERED_data_struct *) 
            method->method_data;
    name = malloc (strlen (method->base_path) + strlen (fd->name) + 1);
    sprintf (name, "\%s\%s", method->base_path, fd->name);
    if (stat (name, \&s) == 0)
        p->file_size = s.st_size;

    switch (fd->mode)
    {
        case adios_mode_read:
        {
            p->f = fopen (name, "r");
            if (p->f <= 0)
            {
                fprintf (stderr, "ADIOS POSIX ASCII UNBUFFERED: "
                        "file not found: \%s\n", fd->name);
                free (name);
                return 0;
            }
            break;
        }

        case adios_mode_write:
        {
            p->f = fopen (name, "w");
            if (p->f <= 0)
            {
                fprintf (stderr, "adios_posix_ascii_nb_open " 
                        "failed for base_path %s, name %s\n", 
                        method->base_path, fd->name 
                        );
                free (name);
                return 0;
            }
            break;
        } 

        case adios_mode_append:
        {
            int old_file = 1;
            p->f = fopen (name, "a");
            if (p->f <= 0)
            {
                fprintf (stderr, "adios_posix_ascii_nb_open"
                        " failed for base_path \%s, name \%s\n"
                        ,method->base_path, fd->name
                        );
                free (name);
                return 0;
            }
            break;
        }

        default:
        {
            fprintf (stderr, "Unknown file mode: \%d\n", fd->mode);
            free (name);
            return 0;
        }
    }
    free (name);
    return 0;
}

enum ADIOS_FLAG adios_posix_ascii_nb_should_buffer(
                            struct adios_file_struct * fd,
                            struct adios_method_struct * method,
                            void * comm) 
{
    //in this case, we don't use shared_buffer
    return adios_flag_no;
}

void adios_posix_ascii_nb_write (struct adios_file_struct * fd,
                                 struct adios_var_struct * v,
                                 void * data,
                                 struct adios_method_struct * method) 
{
    struct adios_POSIX_ASCII_UNBUFFERED_data_struct * p;
    p = (struct adios_POSIX_ASCII_UNBUFFERED_data_struct *)
            method->method_data;
    if (!v->dimensions) {
        switch (v->type)
        {
            case adios_byte:
            case adios_unsigned_byte:
                fprintf (p->f,"\%c\n", *((char *)data)); 
                break;
            case adios_short:
            case adios_integer:
            case adios_unsigned_short:
            case adios_unsigned_integer:
                fprintf (p->f,"\%d\n", *((int *)data)); 
                break;
            case adios_real:
            case adios_double:
            case adios_long_double:
                fprintf (p->f,"\%f\n", *((double *)data)); 
                break;
            case adios_string:
                fprintf (p->f,"\%s\n", (char *)data); 
                break;
            case adios_complex:
                fprintf (p->f,"\%f+\%fi\n", 
                         *((float *)data),*((float *)(data+4))); 
                break;
            case adios_double_complex:
                fprintf (p->f,"\%f+\%fi\n", 
                         *((double *)data),*((double *)(data+8))); 
                break;
            default:
                break;
        }
    } 
    else
    {
        uint64_t j;
        int element_size = adios_get_type_size (v->type, v->data);
        printf("element_size: \%d\n",element_size);
        uint64_t var_size = adios_get_var_size (v, fd->group, v->data) /
                                element_size;
        switch (v->type)
        {
            case adios_byte:
            case adios_unsigned_byte:
                for (j = 0;j < var_size; j++)
                    fprintf (p->f,"\%c ", *((char *)(data+j)));
                printf("\n");
                break;
            case adios_short:
            case adios_integer:
            case adios_unsigned_short:
            case adios_unsigned_integer:
                for (j = 0;j < var_size; j++)
                    fprintf (p->f,"\%d ", *((int *)(data+element_size*j)));
                printf("\n");
                break;
            case adios_real:
            case adios_double:
            case adios_long_double:
                for (j = 0;j < var_size; j++)
                    fprintf (p->f,"\%f ", * ( (double *)(data+element_size*j)));
                printf("\n");
                break;
            case adios_string:
                for (j = 0;j < var_size; j++)
                    fprintf (p->f,"\%s ", (char *)data);
                printf("\n");
                break;
            case adios_complex:
                for (j = 0;j < var_size; j++)
                    fprintf (p->f, "\%f+\%fi ", *((float *)(data+element_size*j)),
                            *((float *)(data+4+element_size*j)));
                printf("\n");
                break;
            case adios_double_complex:
                for (j = 0;j < var_size; j++)
                    fprintf (p->f,"\%f+\%fi ", *((double *)(data+element_size*j)),
                            *((double *)(data+element_size*j+8)));
                printf("\n");
                break;
            default:
                break;
        } 
    }
}

void adios_posix_ascii_nb_get_write_buffer (struct adios_file_struct * fd,
                                            struct adios_var_struct * v,
                                            uint64_t * size,
                                            void ** buffer,
                                            struct adios_method_struct * method)
{
*buffer = 0;
}

void adios_posix_ascii_nb_read (struct adios_file_struct * fd,
                                struct adios_var_struct * v, 
                                void * buffer,
                                uint64_t buffer_size,
                                struct adios_method_struct * method)
{
    v->data = buffer;
    v->data_size = buffer_size; 
}

int adios_posix_ascii_nb_close (struct adios_file_struct * fd,
                                struct adios_method_struct * method)
{
    struct adios_POSIX_ASCII_UNBUFFERED_data_struct * p;
    p = (struct adios_POSIX_ASCII_UNBUFFERED_data_struct *)
        method->method_data;
    if (p->f <= 0)
    {
        fclose (p->f);
    }
    p->f = 0;
    p->file_size = 0; 
}

void adios_posix_ascii_nb_finalize (int mype, 
                                    struct adios_method_struct * method)} 
{
    if (adios_posix_ascii_nb_initialized)
        adios_posix_ascii_nb_initialized = 0; 
}

\end{lstlisting}

The binary transport method blocks methods for simplicity. Therefore,  no special 
implementation for the three functions below is necessary and their function bodies 
can be left empty:

\begin{lstlisting}
adios_posix_ascii_nb_end_iteration (struct adios_method_struct * method) {}
adios_posix_ascii_nb_start_calculation (struct adios_method_struct * method) {}
adios posix_ascii_nb stop_calculation (struct adios_method_struct * method) {}
\end{lstlisting}

Above, we have implemented the POSIX\_ASCII\_NB transport method. When users specify 
POSIX\_ASCII\_NB method in xml file, the users' applications will generate ASCII 
files by using common ADIOS APIs. However, in order to achieve better I/O performance, 
a buffering scheme needs to be included into this example.

\section{Profiling the Application and ADIOS}

There are two ways to get profiling information of ADIOS I/O operations. One way 
is for the user to explicitly insert a set of profiling API calls around ADIOS 
API calls in the source code. The other way is to link the user code with a renamed 
ADIOS library and an ADIOS API wrapper library. 

\subsection{Use profiling API in source code}

The profiling library called libadios\_timing.a implements a set of profiling API 
calls. The user can use these API calls to wrap the ADIOS API calls in the source 
code to get profiling information. 

The adios-timing.h header file contains the declarations of those profiling functions. 

\begin{lstlisting}[frame=single, backgroundcolor=\color{gray85}]
/*
 * initialize profiling 
 *
 * Fortran interface
 */
int init_prof_all_(char *prof_file_name, int prof_file_name_size);

/*
 * record open start time for specified group
 *
 * Fortran interface
 */
void open_start_for_group_ (int64_t *gp_prof_handle, char *group_name, 
                            int *cycle, int *gp_name_size);

/*
 * record open end time for specified group
 *
 * Fortran interface
 */
void open_end_for_group_(int64_t *gp_prof_handle, int *cycle);

/*
 * record write start time for specified group
 *
 * Fortran interface
 */
void write_start_for_group_(int64_t *gp_prof_handle, int *cycle);

/*
 * record write end time for specified group
 *
 * Fortran interface
 */
void write_end_for_group_(int64_t *gp_prof_handle, int *cycle);

/*
 * record close start time for specified group
*
 * Fortran interface
 */
void close_start_for_group_(int64_t *gp_prof_handle, int *cycle);

/*
 * record close end time for specified group
 *
 * Fortran interface
 */
void close_end_for_group_(int64_t *gp_prof_handle, int *cycle);

/*
 * Report timing info for all groups
 *
 * Fortran interface  
 */
int finalize_prof_all_();

/*
 * record start time of a simulation cycle
 *
 * Fortran interface 
 */
void cycle_start_(int *cycle);

/*
 * record end time of a simulation cycle
 *
 * Fortran interface 
 */
void cycle_end_(int *cycle);

\end{lstlisting}

An example of using these functions is given below....

\begin{lstlisting}[language=Fortran, frame=single, backgroundcolor=\color{gray85}]
...
! initialization ADIOS
CALL adios_init ("config.xml"//char(0))
! initialize profiling library; the parameter specifies the file where 
! profiling information is written
CALL init_prof_all("log"//char(0))
...
CALL MPI_Barrier(toroidal_comm, error )
! record start time of open
! group_prof_handle is an OUT parameter holding the handle for the 
! group 'output3d.0'
! istep is iteration no.
CALL open_start_for_group(group_prof_handle, "output3d.0"//char(0),istep)

CALL adios_open(adios_handle, "output3d.0"//char(0), "w"//char(0))

! record end time of open
CALL open_end_for_group(group_prof_handle,istep)

! record start time of write
CALL write_start_for_group(group_prof_handle,istep)

#include "gwrite_output3d.0.fh"

! record end time of write
CALL write_end_for_group(group_prof_handle,istep)

! record start time of close
CALL cose_start_for_group(group_prof_handle,istep)

CALL adios_close(adios_handle,adios_err)

! record end time of close
CALL close_end_for_group(group_prof_handle,istep)

...
CALL adios_finalize (myid)

! finalize; profiling information are gathered and 
! min/max/mean/var are calculated for each IO dump
CALL finalize_prof()

CALL MPI_FINALIZE(error)


\end{lstlisting}

When the code is run, profiling information will be saved to the file ''./log'' 
(specified in init\_prof\_all ()). Below is an example.

{\small 
\begin{lstlisting}[language=Fortran, frame=single, backgroundcolor=\color{gray85}]
Fri Aug 22 15:42:04 EDT 2008
I/O Timing results
Operations   : min                 max                 mean                var
cycle no       3
io count       0
# Open       : 0.107671            0.108245            0.108032            0.000124
# Open start : 1219434228.866144   1219434230.775268   1219434229.748614   0.588501
# Open end   : 1219434228.974225   1219434230.883335   1219434229.856646   0.588486
# Write      : 0.000170            0.000190            0.000179            0.000005
# Write start: 1219434228.974226   1219434230.883336   1219434229.856647   0.588486
# Write end  : 1219434228.974405   1219434230.883514   1219434229.856826   0.588484
# Close      : 0.001608            0.001743            0.001656            0.000036
# Close start: 1219434228.974405   1219434230.883514   1219434229.856826   0.588484
# Close end  : 1219434228.976040   1219434230.885211   1219434229.858482   0.588489
# Total      : 0.109484            0.110049            0.109868            0.000137
cycle no       6
io count       1
# Open       : 0.000007            0.000011            0.000009            0.000001
# Open start : 1219434240.098444   1219434242.007951   1219434240.981075   0.588556
# Open end   : 1219434240.098452   1219434242.007962   1219434240.981083   0.588556
# Write      : 0.000175            0.000196            0.000180            0.000004
# Write start: 1219434240.098452   1219434242.007962   1219434240.981083   0.588557
# Write end  : 1219434240.098631   1219434242.008158   1219434240.981264   0.588558
# Close      : 0.000947            0.003603            0.001234            0.000466
# Close start: 1219434240.098631   1219434242.008158   1219434240.981264   0.588558
# Close end  : 1219434240.099665   1219434242.009620   1219434240.982498   0.588447
# Total      : 0.001132            0.003789            0.001423            0.000466

\end{lstlisting}
}

The script ``post\_script.sh'' extracts ``open time'', ``write time'', ``close 
time'', and ``total time'' from the raw profiling results and saves them in separate 
files: open, write, close, and total, respectively.

To compile the code, one should link the code with the -\textit{ladios\_timing 
-ladios} option. 

\subsection{Use wrapper library}

Another way to do profiling is to link the source code with a renamed ADIOS library 
and a wrapper library. 

The renamed ADIOS library implements ``real'' ADIOS routines, but all ADIOS public 
functions are renamed with a prefix ``P''. For example, adios\_open() is renamed 
as Padios\_open(). The routine for parsing config.xml file is also changed to parse 
extra flags in config.xml file to turn profiling on or off.

The wrapper library implements all adios pubic functions (e.g., adios\_open, adios\_write, 
adios\_close) within each function. It calls the ``real'' function (Padios\_xxx()) 
and measure the start and end time of the function call. 

There is an example wrapper library called libadios\_profiling.a. Developers can 
implement their own wrapper library to customize the profiling.

To use the wrapper library, the user code should be linked with -\textit{ladios\_profiling 
-ladios}. the wrapper library should precede the ``real'' ADIOS library. There 
is no need to put additional profiling API calls in the source code. The user can 
turn profiling on or off for each ADIOS group by setting a flag in the config.xml 
file.

\begin{lstlisting}
<adios-group name="restart.model" profiling="yes|no">
    ...
</adios-group\>
\end{lstlisting}
