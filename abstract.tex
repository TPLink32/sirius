% Special formatting for the abstract
\titleformat{\chapter}[block]
%{\filcenter\normalfont\huge\bfseries}{}{0pt}{\huge}
{\filcenter\normalfont\LARGE\bfseries}{}{0pt}{\LARGE}
%{\filcenter\normalfont\Large\bfseries}{}{0pt}{\Large}
%{\filcenter\normalfont\Large\bfseries}{}{0pt}{\large}
\titlespacing*{\chapter}{0pt}{-0.35in}{0pt} %%% * kill first para indentation

\chapter*{\MakeUppercase{\proposaltitle}}
\phantomsection
\addstarredchapter{Abstract}
%%\adjustmtc
\markboth{Abstract}{}
\label{sec:abstract}

% Restore standard formatting for chapter titles
\titleformat{\chapter}[block]
{\normalfont\Huge\bfseries}{\thechapter.}{10pt}{\Huge}
\titlespacing*{\chapter}{0pt}{-0.25in}{0pt} %%% * kill first para indentation

%% \ifthenelse{\equal{\proposalcontent}{abstract}}{ % Standalone abstract-only
%%    \typeout{output-controls: empty pagestyle for standalone abstract}
%%    \thispagestyle{empty}
%% }{}

% Vertical spacing is careful controlled in order to maximize space
% available while maintaining a reasonable experience.

\vspace{-\abovedisplayskip}
\medskip

% Lead PI slug is centered internally
\textbf{% Generated by bin/config-leadpi-slug on Fri Jan 17 21:40:42 EST 2014
\begin{center}
Lead PI: Scott Klasky, \ORNL\\
Email: \emph{klasky@ornl.gov}, Telephone: 865-241-9980 , Fax: 865-241-4811
\end{center}
}

\vspace{-\belowdisplayskip}
\vspace{-\abovedisplayskip}
\smallskip

\begin{center}
% Generated by bin/config-participant-table on Thu, Jul 09, 2015  4:57:56 PM
\newlength{\longestpientry}
\setlength{\longestpientry}{\maxof{\longestpientry}{\widthof{\textbf{Scott Klasky} (\emph{klasky@ornl.gov})}}}
\setlength{\longestpientry}{\maxof{\longestpientry}{\widthof{\textbf{Gerald Lofstead} (\emph{gflofst@sandia.gov})}}}
\setlength{\longestpientry}{\maxof{\longestpientry}{\widthof{\textbf{Carlos Maltzahn} (\emph{carlosm@soe.ucsc.edu})}}}
\begin{tabularx}{\textwidth}{l>{\raggedright\arraybackslash}X}
\emph{Institution} & \makebox[\longestpientry][l]{\emph{Principal Investigator (Email)}} \quad \emph{Additional Senior Personnel}
\\
\textbf{\ornl} & \makebox[\longestpientry][l]{\textbf{Scott Klasky} (\emph{klasky@ornl.gov})}\hspace*{1em}Hasan Abbasi, Gary Liu, Kimmy Mu, Feiyi Wang\\
\textbf{\snl} & \makebox[\longestpientry][l]{\textbf{Gerald Lofstead} (\emph{gflofst@sandia.gov})}\hspace*{1em}Matthew Curry, Lee Ward\\
\textbf{\ucsc} & \makebox[\longestpientry][l]{\textbf{Carlos Maltzahn}
 (\emph{carlosm@soe.ucsc.edu})}\hspace*{1em}\\
 \textbf{Rutgers} & \makebox[\longestpientry][l]{\textbf{Manish Parashar}
 (\emph{parashar.rutgers.edu})}\hspace*{1em}\\
\end{tabularx}

\end{center}
 
\vspace{-\belowdisplayskip}
\vspace{-\abovedisplayskip}
\medskip

\begin{center}
\textbf{ABSTRACT}
\end{center}

\vspace{-\belowdisplayskip}

% Even with the space manipulation above, we need to cheat a little
\enlargethispage{2\baselineskip}
Exascale scientific discovery will be
severely bottlenecked without sufficient new research into managing and
storing the large amounts of data that will be produced during the
simulation, and analyzed for months afterwards. In this project we will
demonstrate novel techniques to  facilitate efficient mapping of data
objects, even partitioning individual variables, from
the user space onto multiple storage tiers, and enable application-guided
data reductions/transformations to address capacity and bandwidth
bottlenecks. 
Our goal in this project is
to address the associated I/O and storage challenges in the context of
current and emerging storage landscapes, and expedite insights into mission
critical scientific processes; which is associated with {\bf theme two} of the FOA. To that end, we will build on the
capabilities offered by ADIOS and DataSpaces at ORNL that provide I/O abstractions
and services, the Sirocco peer-to-peer file system at
Sandia and the object storage and annotation expertise of UC Santa Cruz, to
explore application and multi-tier storage aware data management
solutions.    This project
brings together a  team with strong expertise in I/O middleware (ORNL, Rutgers), file system (SNL, UCSC) and storage (UCSC), and connects and
coordinates these key storage components in a seamless fashion.

Our objective here is to reduce the time to knowledge, not just for a single
application, but for the entire workload in a multi-user environment, where
the storage is shared among users.
We achieve this goal by allowing selectable data quality, by trading its accuracy and error
in order to meet the time or resource constraint. 
We will explore beyond 
checkpoint/restart I/O, and will address the challenges posed by 
key
data access patterns in the knowledge gathering process.
Ultimately, we will take the knowledge from the storage system to provide vital feedback to the middleware 
so that the best possible decisions can be autonomically made between the user intentions and
the available system resources.  
We will test our prototypes  on current and future DOE system with many of today's applications, including the
s XGC1, GTC, QMCPack, and SpecFM3D simulations. 
Our solutions will provide  new functionalities and APIs for
1) specifying, at the application level, data annotations that enable the
quantification of the relative importance and utility of data objects and
enable partitioning  across the storage hierarchy;
2) specifying selectable performance/quality/cost tradeoffs from both the
application and system perspectives;
3) evaluating these tradeoffs at runtime during data placement and movement,
and executing the resultant policies in an autonomic system using models,
heuristics and continuous learning; and
4) leveraging techniques such as application-aware data compression, 
and I/O prioritization to enforce these
policies.
\noindent

\vfill
