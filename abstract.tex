% Special formatting for the abstract
\titleformat{\chapter}[block]
%{\filcenter\normalfont\huge\bfseries}{}{0pt}{\huge}
{\filcenter\normalfont\LARGE\bfseries}{}{0pt}{\LARGE}
%{\filcenter\normalfont\Large\bfseries}{}{0pt}{\Large}
%{\filcenter\normalfont\Large\bfseries}{}{0pt}{\large}
\titlespacing*{\chapter}{0pt}{-0.35in}{0pt} %%% * kill first para indentation

\chapter*{\MakeUppercase{\proposaltitle}}
\phantomsection
%\addstarredchapter{Abstract}
%%\adjustmtc
\markboth{Abstract}{}
\label{sec:abstract}

% Restore standard formatting for chapter titles
\titleformat{\chapter}[block]
{\normalfont\Huge\bfseries}{\thechapter.}{10pt}{\Huge}
\titlespacing*{\chapter}{0pt}{-0.25in}{0pt} %%% * kill first para indentation

%% \ifthenelse{\equal{\proposalcontent}{abstract}}{ % Standalone abstract-only
%%    \typeout{output-controls: empty pagestyle for standalone abstract}
%%    \thispagestyle{empty}
%% }{}

% Vertical spacing is careful controlled in order to maximize space
% available while maintaining a reasonable experience.

\vspace{-\abovedisplayskip}
\medskip

% Lead PI slug is centered internally
\textbf{% Generated by bin/config-leadpi-slug on Fri Jan 17 21:40:42 EST 2014
\begin{center}
Lead PI: Scott Klasky, \ORNL\\
Email: \emph{klasky@ornl.gov}, Telephone: 865-241-9980 , Fax: 865-241-4811
\end{center}
}

%\vspace{-\belowdisplayskip}
\vspace{-\abovedisplayskip}
\smallskip

\begin{center}
% Generated by bin/config-participant-table on Thu, Jul 09, 2015  4:57:56 PM
\newlength{\longestpientry}
\setlength{\longestpientry}{\maxof{\longestpientry}{\widthof{\textbf{Scott Klasky} (\emph{klasky@ornl.gov})}}}
\setlength{\longestpientry}{\maxof{\longestpientry}{\widthof{\textbf{Gerald Lofstead} (\emph{gflofst@sandia.gov})}}}
\setlength{\longestpientry}{\maxof{\longestpientry}{\widthof{\textbf{Carlos Maltzahn} (\emph{carlosm@soe.ucsc.edu})}}}
\begin{tabularx}{\textwidth}{l>{\raggedright\arraybackslash}X}
\emph{Institution} & \makebox[\longestpientry][l]{\emph{Principal Investigator (Email)}} \quad \emph{Additional Senior Personnel}
\\
\textbf{\ornl} & \makebox[\longestpientry][l]{\textbf{Scott Klasky} (\emph{klasky@ornl.gov})}\hspace*{1em}Hasan Abbasi, Gary Liu, Kimmy Mu, Feiyi Wang\\
\textbf{\snl} & \makebox[\longestpientry][l]{\textbf{Gerald Lofstead} (\emph{gflofst@sandia.gov})}\hspace*{1em}Matthew Curry, Lee Ward\\
\textbf{\ucsc} & \makebox[\longestpientry][l]{\textbf{Carlos Maltzahn}
 (\emph{carlosm@soe.ucsc.edu})}\hspace*{1em}\\
 \textbf{Rutgers} & \makebox[\longestpientry][l]{\textbf{Manish Parashar}
 (\emph{parashar.rutgers.edu})}\hspace*{1em}\\
\end{tabularx}

\end{center}
 
\vspace{-\belowdisplayskip}
\vspace{-\abovedisplayskip}
\medskip

\begin{center}
\textbf{ABSTRACT}
\end{center}

\vspace{-\belowdisplayskip}

% Even with the space manipulation above, we need to cheat a little
\enlargethispage{2\baselineskip}

% Scientific discovery at the exascale will not be possible without significant
% new research into the management, storage and retrieval over the full lifespan
% of the extreme amounts of data that will be produced. Our thesis is that by
% adding application level knowledge about data to guide the actions of the
% storage system we obtain substantial benefits in the organization, storage, and
% access to extreme scale data, resulting in improved productivity for
% computational science. We will demonstrate novel techniques to facilitate
% efficient mapping of data objects from the user space onto multiple storage
% tiers, and enable application-guided data reductions and transformations to
% address capacity and bandwidth bottlenecks.  Our goal is to address the
% associated Input/Output (I/O) and storage challenges in the context of current
% and emerging storage landscapes, and expedite insights into mission critical
% scientific processes.  We aim to reduce the time to insight, not just for a
% single application, but for the entire workload in a multi-user environment,
% where storage is shared across users. We shall achieve this by allowing
% selectable data quality, through trading in accuracy and error in order to meet
% the time or resource constraint. We will explore beyond checkpoint/restart I/O,
% and will address the challenges posed by key data access patterns in the
% knowledge gathering process. Ultimately, we will take the knowledge from the
% storage system to provide vital feedback to the middleware so that the best
% possible decisions can be autonomically made between the user intentions and
% the available system resources.  We will test our prototypes on current and
% future Department of Energy (DOE) system with many of today's cutting edge
% applications to ensure our techniques can be used within our framework on
% current and next generation systems.


This project explores the use of application level knowledge to optimize the
times to insight across a workload of multiple applications in a multi-user
environment with shared storage and network resources. The basis for this
work is the notion of selectable data quality to explore the tradeoffs
between accuracy of results, resource requirements, and time to insight on
systems with shared, oversubscribed computational and storage resources. 
In the last several years extreme scale computational science was conducted
using dedicated computational resources, with shared resources, such as
network and storage, being overprovisioned. This resulted in little to no
impact from resource contention. The increased scale of recent, and incoming
future systems, has placed a greater emphasis on mitigating contention. 

Our thesis is that by adding application level knowledge
about data to guide storage system behaviors, we will obtain substantial benefits
in the organization, storage, and access to extreme scale data resulting in
improved productivity for computational science. We will demonstrate novel
techniques to facilitate efficient and effective data placement onto
multiple storage tiers and enable application-guided data reductions and
transformations to address capacity and bandwidth bottlenecks. Our goal is
to address the associated data management challenges in the context of
current and emerging storage landscapes and expedite insights into mission
critical scientific processes.

\noindent

\vfill
