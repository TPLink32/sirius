\subsection{Migration}

One important research question in this proposal is whether and how a liberated
Hierarchical Storage Manager designed for online use, such as Sirocco, might be
employed in supporting the higher level concepts discussed previously.
Conceptually, in this effort, data is accessed by the middleware IO libraries
at various tiers in the lower-level storage hierarchy. While Sirocco does
autonomously group the distributed set of media into like groups and manage
movement of data between these groups the various motivations to do so include
both the traditional ones found in classical HSM (Hierarchical Storage
Management) as well as the novel.

Sirocco groups like media into aggregated volumes on a service node, and like
service nodes with like media into pseudo-tiers. It accomplishes both using
attributes such as latency to first byte, bandwidth to and from the media, and
some function representing the resilience capabilities of the attached media.
In addition to these static attributes, when grouping servers into
pseudo-tiers, Sirocco also may employ more dynamic attributes such as CPU load
and the rate-of-change of the media use.

Grouping into pseudo-tiers is not a formal, nor rigid, concept within Sirocco,
at present. In classic HSM the tiering is often a formally expressed concept,
embodied both in the architecture and the implementation. In Sirocco, though,
it is not directly expressed. Instead, a client or server that attempts to make
a decision relative to migration and staging uses the attributes discussed
above to constrain the choice of candidates. For instance, a server acting as
an off-node writeback cache for a compute node might only have access to
volatile dynamic RAM for use as storage media and a client might deposit data
that is marked for persistent media as a resilience constraint. At some point,
then, such a server must copy or move the data to another Sirocco server with
media matching, or exceeding, the resilience constraint. At that point in time,
all servers managing appropriate persistent media are candidates but
second-order attributes, such as latency and bandwidth of a candidate's managed
media would tend to dominate the ranking. Thus, the concept of a ``tier''
becomes an artifact of decision rather than a formal concept.

Further complication arises when a Sirocco server is forced to make choices for
capacity reasons. The Sirocco architecture dictates that a server function as a
victim cache. As media managed by a server fills the server will begin to
manage for capacity. In order to do so, currently, it need only verify or make
a copy of the data on another server that meets the resiliency constraint of
the held data prior to removing the local copy and in the current
implementation the server makes no attempt either to remember it ever had the
data or to notify others about what it is doing.  others that it has removed
the local copy or remember that it has 

Taken together then, the above at least complicates the higher level research
questions discussed in this proposal, potentially even renders some
unachievable since a few involve determinism. We are forced, then, to consider
either or both mitigating management policies within and between Sirocco
servers as well as explicitly allowing clients to participate in capacity
management decisions, for a time at least, on the relevant Sirocco servers.

Our research intent involves short term and long term activities. In the short
term we will modify the relevant clients to provide hit and miss related data
and associated activities and explore an enhancement to our storage servers
that would provide some, hopefully tightly, bounded estimate about time to
first byte for a potential read operation. In the long term we hope that an
augmented client-server API providing usage hints will allow us to modify local
server policy algorithms to more effectively manage it's media and communicate
to other servers holding copies of the relevant data how they might best manage
theirs. In the event that this approach proves ineffective at meeting the
overall approach described in this proposal we will explore modifying Sirocco
servers to provide notifications about locations of the relevant data as they
migrate and stage as well as supporting client-directed pinning of location,
for a period at least. Client-directed pinning could introduce potentially
serious negative impacts to Sirocco's overarching goals though, since the
fundamental design of Sirocco assumes much about opaque and unrestricted data
movement, so this is contemplated only as a last resort.

