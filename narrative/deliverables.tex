\section{Deliverables}
The following are the per year project deliverables:

\textbf{Year 1}
\begin{itemize}
\item Develop new techniques for data description that allow users to 
describe the data utility based on their expectations. (ORNL)

\item Assemble and evaluate general techniques for refactoring of data
that will be embedded in the system and can be selected by users or applications. (Brown/ORNL)
\item Explore the use of application hints and utility of data to guide the initial placement of data. 
(Rutgers/ORNL)
\item Investigate the trade-off between ``filing'' and ``piling'' data. 
Demonstrate a time bounded search approach for finding data within the storage system
to identify data and the current location. (UCSC/Sandia)
\item Demonstrate a metadata service capable of serving both POSIX clients and our clients. (Sandia/UCSC)
\end{itemize}

\textbf{Year 2}
\begin{itemize}
\item Develop a new type of querying system that allows an application to ask the storage system
about completion timing information. (ORNL)
\item Evaluate the effectiveness and overhead to re-organize data and use the multi-layer approach 
to staging. (Brown/ORNL)
\item Research autonomic data management strategies that
can evaluate utility/cost tradeoff, and appropriately place/move data objects at runtime. 
Research runtime tracking and estimation. (Rutgers/ORNL)
\item Extend Sirocco to enforce data-centric metrics that help guide the QoS decisions. (Sandia)
(Sandia/UCSC)
\end{itemize}

\textbf{Year 3}
\begin{itemize}
\item Work with DOE applications, particularly XGC1, GTC, and SPECFM3D, to assess 
the new I/O interface including both APIs and hints. (ORNL)
\item Research techniques that can enhance migration without making the eventual 
lookup of data unbounded, and manage each tier in a scalable fashion. (ORNL)
\item Demonstrate incorporating A\&A and show scalability under both application load as
well as storage system pressures showing that we can tolerate both loads and maintain quality of
service. (Sandia)
%\item storage APIs to manipulate storing and retrieving data to specific tiers
%\item metadata management for data stored in multiple tiers using different compression
%\item metadata management for deeper data knowledge
%\item middleware hooks to manage new storage APIs
%\item example lossy and lossless compression plug-ins to demonstrate the effectiveness of this approach
\end{itemize}
