\section{Deliverables}
%The following are the per year project deliverables:
\small

\textbf{Year 1}
\begin{tightItemize}
\item Develop  techniques for data description  to 
describe the data utility based on user expectations. (ORNL)

\item Evaluate general  techniques for refactoring of data
that will be embedded in the system . (ORNL)
\item Explore the use of application hints and utility of data to guide the initial placement of data. 
(Rutgers)
\item Investigate the trade-off between ``filing'' and ``piling'' data. 
Demonstrate a time bounded search approach for finding data within the storage system
to identify data and the current location. (UCSC)
\item Demonstrate a metadata service capable of serving both POSIX clients and our clients. (Sandia)
\end{tightItemize}
\textbf{Year 2}
\begin{tightItemize}
\item Develop a querying system to allow applications to inquire
about completion timing information. (ORNL)
\item Evaluate the effectiveness and overhead to re-organize data and use the multi-layer approach 
to staging. (ORNL)
\item Research autonomic data management strategies that
can evaluate utility/cost trade-off, and appropriately place/move data objects at runtime. 
Research runtime tracking and estimation. (Rutgers)
\item Extend Sirocco to enforce data-centric metrics that help guide the QoS decisions. (Sandia)
\item Prototype time bounded search approach for finding data within the storage system (UCSC)
\end{tightItemize}
\textbf{Year 3}
\begin{tightItemize}
\item Work with DOE applications to assess 
the new I/O interface including both APIs and hints. (ORNL, Rutgers)
\item Prototype new QoS methods for storing \& retrieving data in given time bounds. (ORNL, UCSC)
\item Demonstrate admission control and show scalability for  application loads and 
 storage system pressures to maintain quality of
service. (Sandia, UCSC)
%\item storage APIs to manipulate storing and retrieving data to specific tiers
%\item metadata management for data stored in multiple tiers using different compression
%\item metadata management for deeper data knowledge
%\item middleware hooks to manage new storage APIs
%\item example lossy and lossless compression plug-ins to demonstrate the effectiveness of this approach
\end{tightItemize}
\normalsize
