\subsubsection{Naming Service}

A traditional POSIX naming service offers a hierarchical space consisting of
directories and files. LWFS and Sirocco have abandonded this as the default
model in favor of a container/object/fork/addrress tuple for identifying data.
LWFS demonstrated a POSIX-style namespace on the side kept in sync using a
transaction process like D2T~\cite{d2t} showing that this alternative approach
can support traditional POSIX API calls even though the underlying storage
system uses a different model.

For this project, we seek to offer a similar POSIX-style namespace for naming
objects, but store a rich metadata collection with each entry to handle data
migration, discovery, and writing and retrevial time estimations. Further, we
will also have to track data stored in different tiers, even at the variable
subset level. This additional metadata load will require a more complex naming
service that reacts to automatic data movement while caching current data
locations accelerating data access operations avoiding any data discovery
operations.

In addition to the basic naming operations, we will also need to incorporate
authorization capabilities. Sirocco currently integrates with a Kerberos
service for authentication and authorization. Given a capability ticket, a
user can access different objects as needed.

The main challenge of incorporating the additional, rich metadata will be
joined by the additional challenge of coordinating with the other storage
and application-layer services to offer the best access times possible for data
stored in the system.
