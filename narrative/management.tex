\section*{Management}
\label{sec:management}

Careful management will be required to ensure that the problems and delays that are inevitable in any research project
do not unduly disrupt overall progress. To this end, we define a detailed workplan,
establish a  management structure with defined oversight responsibilities, and establish a schedule of regular meetings and
reviews to ensure progress.  We will use google docs to share documents, emails lists for informal communications and a 
weekly phone call in the main research areas to discuss project tasks and milestones. 

Dr. Klasky will oversee the outcome of this project,  Dr. Abbasi (ORNL) will lead the Application Interface area, and Dr. Lofstead
will lead the Storage System. This project is focused on the integration of application aware-middleware which communications to the storage system,
and each of these leaders have a strong track record of tight-collaboration, both working with Dr. Klasky for their Ph.D. at the Georgia
Institute of Technology.  Our structure includes an area lead for each major research area plus
an integration lead responsible for coordinating periodic integration tests. The area leads plus the project PIs
form the management team, which will meet weekly to review progress and revise plans as needed.
We will also have a {\bf Science Advisory Board} consisting of one application Scientist, Dr. C. S. Chang (PPPL), one storage system expert from the 
OLCF, Dr. Vazhkudai (ORNL) who we will advice us yearly on our project goals and outcomes and make adjustments accordingly. We will also have our
lead Math Lead, Dr. Ainsworth, to work with the applied mathematics community to ensure that the hierarchical decomposition of data will work in theory
for current and next generation exascale simulations. 

We will conduct six-monthly {\bf integration fests}  to validate our ability to combine models, apply
combined models to real problems, and cope with the challenges of real data. We apply this approach routinely
in other projects and find that it helps to focus effort and identify problems early. The tasks of defining,
scheduling, and running critical post hoc reviews of integration tests will be coordinated by the Integration Lead, Dr. Parashar. 
These meetings will bring all hands together at least once a year.

We will monitor and report {\bf success metrics} including: 
research publications;
our ability to predict and explain performance of I/O workloads on the current machines., in different environments;
our ability to reduce data and charcterize data from an object into multiple bins, 
our ability to integrate our routines into the Sirocco file system and test the feasibily on current and next generation LCF systems, and
how tools lead to workflow performance improvements, for both reading and writing data from our exemplar applications which we will test.

\begin{table}[t]
\caption{S2E2 Integration Tests. We list for each the technology area involved and new technologies expected.}
\centering \footnotesize
\vspace{2ex}
\label{tab:ifests}
%\begin{tabular}{| p{2cm} | p{2.5cm} | p{5cm} | p{5cm} | }
\begin{tabular}{| l | l |l |l |} 
\hline
{\bf \#} &  {\bf Focus} & {\bf New technologies to be introduced} \\\hline
1 & TBD			 & TBD\ \ \\
2 &TBD & TBD\\
3 & TBD         & TBD \\
4 & TBD                                    & TBD  \\
5 & TBD                                    & TBD \\
6 & TBD                                    & TBD  \\\hline
\end{tabular}
\vspace{-8ex}
\end{table}