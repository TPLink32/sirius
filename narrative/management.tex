\section{Management}
\label{sec:management}

Careful management will be required to ensure that the problems and delays that are inevitable in any research project
do not  disrupt overall progress.Here we define a detailed work plan,
establish a  management structure with defined oversight responsibilities, and establish a schedule of regular meetings and
reviews to ensure progress.  We will use Google Docs (http://docs.google.com) to share documents, emails lists for informal communications and a 
weekly phone call in the main research areas to discuss project tasks and milestones. 

Dr. Klasky will oversee the outcome of this project,  Dr. Abbasi (ORNL) will lead the Application Interface area, and Dr. Lofstead
will lead the Storage System. This project is focused on the tight integration of application aware-middleware and a storage system backend.
%Our team has a strong track record of tight-collaboration, with both Drs. Abbasi and Lofstead working with Dr. Klasky for their Ph.D. at the Georgia
%Institute of Technology.  
Our structure includes an area lead for each major research area plus
an integration lead responsible for coordinating periodic integration tests. The area leads plus the project PIs
form the management team, which will meet weekly to review progress and revise plans as needed.
We will also have a  yearly meeting with  one application scientist we will partner with,  and one storage system expert from the 
OLCF, who will advise us on our project goals and outcomes and we will make adjustments accordingly. We will also have our
Math Lead, Dr. Ainsworth, to work with the applied mathematics community to ensure that our data-refactoring will work 
for current and next generation exascale math algorithms. 

We will conduct six-monthly {\bf integration tests}  to validate our ability to combine models, apply
combined models to real problems, and cope with the challenges of real data. We apply this approach routinely
in other projects and find that it helps to focus effort and identify problems early. The tasks of defining,
scheduling, and running critical post hoc reviews of integration tests will be coordinated by the Integration Lead, Dr. Parashar. 
These meetings will bring all hands together at least once per year.

We will monitor and report {\bf success metrics} including: 
research publications, 
our ability to predict and explain performance of I/O workloads on the current machines in different environments, 
our ability to reduce data and characterize data from an object into multiple bins, 
our ability to integrate our routines into the Sirocco file system and test the feasibility on current and next generation LCF systems, and
how tools lead to workflow performance improvements, for both reading and writing data from our exemplar applications which we will test.
