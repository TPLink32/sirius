\subsection{Managing Data in the Life Cycle}
\label{sec:managing-data-life}

\newcommand{\Sir}{Sirocco} %So I don't get the spelling wrong!

%key points based on discussion with Carlos
%1. Allocation tables and simple search won't be able to scale
%2. Carlos suggests looking at a determinstic function to decide on
%placement of data into the storage hierarchy. Instead of storing the
%metadata that describes where data is stored (which server/tier) we can
%simple associate a function for each application. 
%3. Research into what would be an approrpriate function that allows even
%splitting of data across many servers/tiers
%4. Similar to our approach for defining utility over time
%5. Migration can also use a function. We shouldn't distinguish migration
%from data movement and placement too much
%6. We need metrics that can decide how the data moves up and down in the
%hierarchy (so we can enable pre-fetching)
%7. Carlos suggested that instead of hints we call them service level
%objectives (and that leads into quality of service)

The application workflows targeted in this proposal generate very large
amounts of data, which needs to be processed and analyzed before
potential insights from the simulations can be realized. 
The overall data life cycle can be viewed as consisting of four key phases. In the first phase data 
is generated by a simulation or through a data acquisition system,
in the case of experimental and observational data (EOD). This data has 
to be appropriately initially placed (and possibly partitioned) within the heterogeneous 
multi-tier storage to optimize application I/O overheads.
%In our proposed heterogenous multi-tier storage platform, the system must
%decide where to place the output data at this time, and how to partition and
%reorganize the data to optimize this initial placement. The key optimization
%metrics during this phase is the perceived I/O overhead for the application.

In the second phase of the data life cycle, the data is managed by the system to 
meet application requirements and constraints. This includes migration, 
reorganization and reduction of the data, etc., to optimize resource
utilization, interference with running applications, and availability and
resiliency objectives. 
%Data can be moved into archival storage (tape or cold storage) during this phase. 
%
%In the second phase of the data life cycle the system takes over the
%management of the data and performs the necessary steps to meet service
%level objectives (SLOs). These steps include migration, reorganization, and
%reductions for the data. During this phase the system optimizes for resource
%utilization, interference with running applications, and availability and
%resiliency objectives. Data can be moved into archival storage (tape or cold
%storage) during this phase. 

In the third phase the data is consumed for knowledge discovery (for example
read into analytics to produce new scientific insights). It must be noted
that the knowledge discovery process does not remove the data from the
system - the data remains available until it is purged. During this phase
the storage system optimizes for read performance and data availability. 

In the final phase the data is purged from the system. The
decision to purge may be made by the system based on operational metrics from
system administrators or by the user. This phase optimizes the space available in the
system. % to prevent out of space concerns for applications. 

Our research will address data management challenges across the entire 
data life cycle in an application driven and system aware manner. 
%given our thesis that more knowledge, and more specific knowledge about the
%data, can allow the management of data to be optimized with reduced cost.
Some of the high level research questions that we will explore are as follows: 
(1) How can we initially place data so that it can be discovered and consumed efficiently?,
%What methods can be utilized in the initial placement of data that do not impose expensive
%costs on discovery, consumption and migration? 
(2) How can the placement and migration of data across a multi-tiered storage hierarchy be 
optimized at runtime, both from the application and system perspective?,
%(3) How can we migrate data throughout the layers of storage
%without making the discovery process unbounded? 
(3) How can knowledge about the application used to better prepare the data for consumption?, and 
(4) When and how do we make the decision to purge data?  %using this additional knowledge? 

%Next we provide details on the challenges and proposed research through each
% phase of the data life cycle.

% We address three main concerns for the data life-cycle in this work. First,
% we will {\it explore the initial placement} of data during the output
% process. Second, we will study techniques that can {\it enhance migration}
% without making the eventual lookup of data unbounded. And finally, we will
% study who to {\it manage each tier} in a scalable fashion. %FIXME 


% \subsubsection{Initial Placement of Data}
% \label{sec:init-plac-data}

% The generation of data from simulations or EOD sources will require the
% system to make initial placement decisions. These decisions will leverage
% the description and refactoring of data mentioned earlier, and also the
% tiered design of the storage system. The volume and velocity of data at
% exascale can be alleviated partially by utilizing in-situ and in-transit
% data staging to process the data before the storage layer. Eventually,
% however, the data, processed and transformed, has to be sent to storage. 

% % In-situ and in-transit data staging approaches can alleviate some of the
% % burden on storage, leveraging available compute and memory resources on high
% % end systems to process the data close to where it is generated. 
% % However, given current architectural trends, emerging systems will have a
% % vastly great gulf between the computing capability (the capability to
% % generate data) and the storage capability (the capability to store the
% % generated data). Moreover, the complexity and heterogeneity of the
% % architecture will introduce new tiers in the storage system. Instead of
% % being a flat, roughly homogeneous space in which data can be stored, the new
% % storage hierarchy will be multi-tiered and greatly heterogeneous. 
% %

% For the initial placement of data, we will utilize the faster layers of the storage
% hierarchy, such as DRAM and NVRAM as the target for the initial write from
% an application as much as possible. These targets can be on-node (such as in
% the design of the Summit system at ORNL) or on separate nodes (as in the
% Trinity system at LANL). Regardless, the space limitations might require
% some of the data to be directly transferred to the parallel file system.
% Partitioning of data to accomplish this has been described in
% ~\S~\ref{sec:data-refactor}. 


% To mitigate the space concerns, we will use the techniques
% described in \S~\ref{sec:data-refactor} to identify and partition the
% segments of data with the highest utility for the application and use case.
% The rest of the data will be stored on the lower layers of the file system,
% such as the parallel file system, as is done today. We will leverage the
% supported data migration techniques in \Sir~to determine the most
% appropriate location to move the data once its in stable storage. This
% migration will be guided by a combination of the utility function defined by
% the application and the resources available to the storage system. Knowing
% the specific structural and semantic information (including relationships
% between data segments) about the data will enable migration to partition the
% data further to optimize access to data for reader. For example, in
% Fig.~\ref{fig:ssio-bucket}, P3 is initially placed in Campaign storage,
% since it's size and utility allow our system to place data there, but as the
% utility of data allows this chunk to be migrated to the Long Term, and
% slower, storage layer to allow other data chunks from other users to take
% advantage of this space. Similar, P2 migrates from the Parallel FS to
% Campaign storage to clear room for other users. We will utilize this
% migration and eviction strategy for Checkpoint/Restart (C/R) data as well,
% where data can be initially placed on NVRAM and then purged after the next
% (or next two) C/R files are written to NVRAM.

% We will also explore policies that combine the knowledge of data, its
% utility to users, the cost of maintaining data in long term storage and
% other metrics to decide how to purge irrelevant data from the storage
% system. 

% Our research will ultimately address the following questions:\\
% 1) Can we do scalable migration without making the discovery process
%     unbounded;\\
% 2) What are the parameters and input for migration;
% 3)  Can we use the additional application level knowledge of data to purge
%     portions of the data from storage without making the data too much less
%     valuable?;\\
% 3)  With so many objects and no centralized directory how do we know what
%     is actually in the storage to be purged?;\\
% 4)  How do purge in a way that doesn't require a central authority,
%     doesn't interfere with other I/O and isn't bottlenecked by some sort of
%     global consistency?

\subsubsection{Data Placement and Movement}
\label{sec:init-plac-data}
When an application outputs or accesses data the storage and middleware layers needs to
decide \textit{what} data is placed \textit{where} in the multi-level storage system. 
This placement decision can have a significant impact on data management 
throughout the life time of the data. For example, our past work on data 
staging~\cite{tongipdps15,qiansc15,docan2012dataspaces,abbasi2010datastager} 
on HPC systems with multi-level memory structures has shown that different 
output techniques targeting different layers of the storage hierarchy can have a 
significant impact on the overhead observed by the application for I/O operations. 

%The key requirements here are application-driven runtime mechanisms for dynamically, 
%and deterministically, managing data placement across the layers of the distributed storage 
%hierarchy, coordinating data movement and data sharing between the components of the 
%application workflow so as to maximize its utility to the application and reduce access costs. 

A key requirement is application-driven runtime mechanisms for dynamically managing 
data placement across the layers of the distributed storage hierarchy throughout the 
data lifecycle, coordinating data movement and data sharing between the components 
of the application workflow, with the overarching goal of maximizing the relative utility 
to the application as well as the system while reducing access costs. 
%
As noted before, the complexities of heterogeneous multi-level storage structures requires 
adaptive placement policies are required to be implemented to optimally utilize storage 
resources vertically (across deep memory hierarchies) and horizontally (across nodes 
within a memory level) while accommodating dynamic application requirements and transient 
system states. 
%
Our approach is to increase knowledge about the data and its use within the 
application, and leverage this knowledge to drive data placement and overall management. 

%Our approach to data placement will leverage the increased set of knowledge about the data 
%available to the middleware and storage system, potentially allowing for greater optimizations 
%as compared to other storage solutions.

\paragraph{Soliciting application {\em hints}:}
As described in \S\ref{sec:data-refactor} one of the main components of our
proposed storage system is the ability to reorganize, refactor and reduce data 
as it is generated, and to reorganize and possibly regenerate the data as it is accessed. 
We carry this principle into the placement and movement of data by allowing
applications to define hints and policies that guide \textbf{what} data is
placed \textbf{where}. We will explore the use of application hints in two
distinct areas. First, we will study the challenges and trade-offs of either
augmenting the I/O interface with hints, or allowing the addition of an
external specification that defines the use case. Our experience with
developing modern I/O interfaces has shown that both techniques 
have value~\cite{tongipdps15,qiansc15}, and we will investigate the set of hints that are embedded in the
application code vs. those that are described within an non-compiled
specification. Second, we will study how hints can guide data
placement as data is handed off from application to storage (during a
write) and from storage to application (during the read). In both cases we
will study what minimal set of annotations and hints can allow the storage
system to minimize data movement and optimize the resources consumed by I/O. 

%Application hints can provide additional information to the runtime based on the users' knowledge of the application workflow as well as on past experiences, for example, possible data access patterns, nature of regions of interest, and domain specific information. 
%This application level information will be added to the specification of the workflow and can provide advisory inputs to efficiently help the runtime make smarter decisions about the placement and/or movement of data objects and the scheduling and execution of tasks, as well as to effectively manage trade-offs between for example, power consumption and time-to-solution. 


\paragraph{Defining data utility:} We will formulate the concept of utility (from an
application perspective) that is assigned data objects to quantify the
benefits of placement and/or movement actions to application. These
utilities may be based on the importance of the data objects, the frequency
with which they are accessed, etc., and can be used during runtime data
management to evaluate and compare different placement/movement options. 
For example, data objects with higher utility value may be
placed closer (e.g., in faster memory) to the accessing application.
Similarly, when it is time to evict data objects from a higher level storage
layer, the decision about which object to evict may take the utility into
account. Our initial work in defining such a utility is presented in~\cite{tongipdps15}.
 We will also explore autonomic approaches for balancing of the placement and 
 movement of data across the memory hierarchy autonomically, which will be driven 
 by the data utility and will leverage both user hints and information gathered at runtime (e.g., runtime data access
history, network topology, etc.).
Finally, we will explore a complementary notion of utility from a system-wide perspective 
that will capture and quantify, for example, the impact of placement decisions on 
the overall utilization of storage across multiple applications, and will enable system level runtime optimization. 


\paragraph{ Runtime tracking and estimation:}
Autonomic data placement and/or movement will leverage the information gathered 
at runtime together with predictive analysis requirements. This information may include 
runtime spatial-temporal data access patterns, and/or physical network topology
information, which will be tracked at runtime, and then be used to estimate and predict 
the data access behaviors of application workflow at runtime. 
A key research aspect here will be to anticipate accesses, for example using learning techniques or using
application knowledge, and use this information to prefetch data objects.
For example, we will identify and catalog key phases and access patterns
in the target workflows and then use it to characterize patterns at runtime
and trigger appropriate placement and/or movement actions. Our initial 
work explore these ideas is presented in~\cite{qiansc15,choi2012mining}.

% \paragraph{State of the art:} NAND flash-based devices are being
% increasingly used in HPC storage systems at different levels and for a
% variety of purposes. Several studies such as \cite{multitier}, \cite{sc10li}
% have investigated using compute node local SSDs as storage buffers, for
% example, to temporarily cache checkpoint data to support recovery from
% failures. Research efforts have also explored using deep memory devices for
% data analysis. Active Flash \cite{activeflash} proposes in-situ scientific
% data analysis by directly executing data analysis tasks on emerging storage
% devices. Minerva \cite{minerva} extends the conventional SSD architecture
% using a FPGA-based storage controller to offload data or I/O intensive
% application code to the SSD to accelerate data analysis. However, these
% solutions cannot deal with coupled application workflows, where the data
% exchange and access patterns can be both complex and dynamic. Furthermore,
% some of these solutions require hardware modifications or special access
% priorities to the HPC systems, which can limit their use to specific
% resources.

\subsubsection{Migration}
\label{sec:migration}
A complementary aspects of the data lifecycle management is the management and 
%Once the system has completed the initial data placement with minimal
%overhead to the application, it is up to the storage system to manage the
migration of data across storage servers and across the hierarchy. 
An important research aspect here is how a liberated Hierarchical Storage
Manager (HSM) designed for on-line use, such as Sirocco, might be employed in
providing support for the migration of data. Sirocco autonomously groups the
distributed set of media and service nodes into like groups and manages the
movement of data between these groups. The grouping is managed using
attributes such as latency to first byte, bandwidth to and from the media, and
some function representing the resilience capabilities of the attached media.
In addition to these static attributes, when grouping servers into
pseudo-tiers, Sirocco also may employ more dynamic attributes such as CPU load
and the rate-of-change of the media use. 

Grouping into pseudo-tiers is not a formal, nor rigid, concept within
Sirocco, at present. In classic HSM the tiering is often a formally
expressed concept, embodied both in the architecture and the implementation.
In Sirocco, though, it is not directly expressed; instead, a client or
server that attempts to make a decision relative to migration and staging
uses these attributes to constrain the choice of candidates. For instance, a
server acting as an off-node writeback cache for a compute node might only
have access to volatile dynamic RAM for use as storage media and a client
might deposit data that is marked for persistent media as a resilience
constraint. At some point, then, such a server must copy or move the data to
another Sirocco server with media matching, or exceeding, the resilience
constraint. At that point in time, all servers managing appropriate
persistent media are candidates but second-order attributes, such as latency
and bandwidth of a candidate's managed media would tend to dominate the
ranking. Thus, the concept of a ``tier'' becomes an artifact of decision
rather than a formal concept.

Give these design decisions, our research in this project will study how
migration can serve to manage storage resident data within the constraints
defined by the application and the user. 

\paragraph{Migration Policies:}
The policies that govern the migration of data across groups and across
tiers have similar challenges as data placement. The migration
policy must encapsulate, at a high level, the constraints defined by the
application, but must also manage resources on each server for capacity and
load reasons. We will explore policy formulations that allow \Sir~to move
data within a tier and continue to meet service objectives. Another component
of migration is the temporal function that determines the utility of data
through its life time on storage. If the utility decays over time, for
example, the initial constraints of data (such as resilience requirements)
will be modified and allow more leeway in how the system will handle migration.
However, as data is widely distributed {\it the scalability of this approach must
be investigated}. 
We will explore how the utility function can be used to modify migration
parameters in a scalable manner, without requiring a centralized authority
to make decisions. 

\paragraph{Bounded estimates for lookup:}
One challenge for storage systems with continuous decentralized migration is
the timeliness of lookup when applications issue read calls. This is due to
how no central directory exists for where data is at anytime. There are
solutions to this problem that utilize lookup tables which are updated
whenever a server initiates a migration process. However, these solutions
are not scalable. In order to have predictable performance the lookup
process must complete in a bounded time. Our potential solution here is to use a
distributed hashing function to determine exactly where data is placed,
reducing lookup to an $O(1)$ operation. We explore the discovery of data in
\S~\ref{sec:naming-discovery}. To support this discovery process, we will
investigate how the migration of data can be performed in a way that enables
bounded estimates  for a potential read operation.

% We will explore the use of a  hashing function
% that will leverage determinism to initiate the lookup process. If at most
% $k$ migration steps can happen 
% Our research intent involves short term and long term activities. In the short
% term we will modify the relevant clients to provide hit and miss related data
% and associated activities and explore an enhancement to our storage servers
% that would provide some, hopefully tightly, bounded estimate about time to
% first byte for a potential read operation. 


% Further complication arises when a Sirocco server is forced to make choices for
% capacity reasons. The Sirocco architecture dictates that a server function as a
% victim cache. As media managed by a server fills the server will begin to
% manage for capacity. In order to do so, currently, it need only verify or make
% a copy of the data on another server that meets the resiliency constraint of
% the held data prior to removing the local copy and in the current
% implementation the server makes no attempt either to remember it ever had the
% data or to notify others about what it is doing.  others that it has removed
% the local copy or remember that it has 

% Taken together then, the above at least complicates the higher level research
% questions discussed in this proposal, potentially even renders some
% unachievable since a few involve determinism. We are forced, then, to consider
% either or both mitigating management policies within and between Sirocco
% servers as well as explicitly allowing clients to participate in capacity
% management decisions, for a time at least, on the relevant Sirocco servers.


% \paragraph{Migration Policies:}
% In the long term we hope that an
% augmented client-server API providing usage hints will allow us to modify local
% server policy algorithms to more effectively manage it's media and communicate
% to other servers holding copies of the relevant data how they might best manage
% theirs. In the event that this approach proves ineffective at meeting the
% overall approach described in this proposal we will explore modifying Sirocco
% servers to provide notifications about locations of the relevant data as they
% migrate and stage as well as supporting client-directed pinning of location,
% for a period at least. Client-directed pinning could introduce potentially
% serious negative impacts to Sirocco's overarching goals though, since the
% fundamental design of Sirocco assumes much about opaque and unrestricted data
% movement, so this is contemplated only as a last resort.

\subsubsection*{Making data available on multiple tiers}
\label{sec:manag-data-mult}
 Providing data to users for knowledge discovery is
one of the primary purposes of our storage system design. There is a
spectrum of approaches on how
this can be accomplished: On one end the storage system can aim to maintain
fully accurate data in the highest tiers of the storage hierarchy. This
approach is, obviously, limited due to the disparity between size of data
and available space in the higher tiers. Another approach is to utilize the
techniques described in \S\ref{sec:data-refactor} to create auxiliary views of
the data that, with user defined error, represent the data, and place these
high priority  data sets on the higher tiers for faster access. 

There are a
number of drivers that push approaches to one or the other extreme:
(1) if a high tier has sufficient room for an application's working
set, approaches that move the working set into that tier will perform
well; Conversely, if a high tier very rarely captures working sets,
it is better to fill it with auxiliary data that speeds up access
to the working set captured by a lower tier; working sets only
reduce capacity misses but the concept can be generalized to the
likelihood of having the right data in place, i.e. pro-actively
moving data to high tiers accessed in predictable patterns reduces
compulsory misses as well; (2) if the application generates parallel
read/write requests and strong consistency semantics are important,
approaches that use auxiliary data to speed up access to persistent
shared data might yield better performance than approaches that
require coherence overhead to keep all copies on higher tiers
consistent; (3) if the application's access patterns are mismatched
with low tier access characteristics, a combination of approaches
that use auxiliary data and move data between tiers in order to
convert access patterns into a better match will perform well.

% Any other drivers?
\paragraph{Thinning of Data:} 
This spectrum enables interesting strategies
to manage space pressure on high tiers: instead of just evicting
pieces of data (blocks, pages, objects, files), one can free space
by ``thinning'' data so that accesses within more limited views or
lower resolutions can still occur without misses while requests
outside of views or higher resolutions can leverage the information
stored in auxiliary data. Conversely, data on a high tier can be
``enriched'' (e.g. turned into a higher resolution) if there is
value in doing so. We will explore how the storage system can perform these
operations to meet the service level objectives for each use case, while
optimizing the overall utility from the system perspective. 

\paragraph{Locality Properties:} The decision on when to thin and when to
enrich data on a particular level depends on \emph{access patterns}.
Approaches that leverage access patterns in applications at leadership facilities
 have shown to greatly improve performance and reduce the amount
of overhead required for data management~\cite{he:hpdc13}. The
challenge is for applications to communicate these access patterns
to the I/O stack layers that manages tiers outside of the address
space. In the proposed work we will develop an abstraction for
\emph{locality} that allow applications or runtime profilers to
describe locality properties of access patterns, and implement
services that leverage locality properties to dynamically determine
the value of thinning or enriching data at a particular tier.
%Consider the example of a road navigation system: the locality
%properties of each navigation client are based on physical constraints
%of speed, direction, likely resolution of map, and the fact that
%clients almost never go off-road.

\subsubsection*{Purging}
\label{sec:purging-data}
An important phase of the data lifecycle is the purging of data from the system, 
which presents significant challenges, both from the systems and policy perspectives. 
Deciding what data to purge and when to purge it can be non-trivial, and it is 
essential that this is done with the application in the loop. One approach we will 
explore in this efforts will be extending the notion of system and application utility. 
The former can be used to determine when data object need to purged while the 
latter can help determine the relative importance of data objects to the 
application and identify which data objects can be purged. Developing policies 
for purging is a more challenging problem and will require engaging  the 
user community. However, a key research problem that will be addressed in
 this effort is developing the abstractions that can be used to consistently and 
 unambiguously express these policies as well as the mechanisms that can be 
 used to correctly enforce these policies. 

%{\color{red}(only outline since I have to go)}
%1. Purging is important from the perspective of preventing uncontrolled
%growth of data on a system
%2. Purging is hard because no one wants to lose data and we need to explore
%the policies that govern it
%3. If we have a good utility function and a good way to refactor the data we
%could purge the ``less important'' pieces and maintain a high level of
%knowledge without requiring very high amount of storage
%3. Purging needs to happen in a decentralized way but global utility has to
%be maximized - thats a pretty hard problem 




% \subsubsection{Preliminary work and results} 
% \label{sec:prework}

% Our recent work \cite{tongipdps15}
% explored a two-tiered staging method that spans both DRAM and solid state
% disks (SSD). It allows us to support both code coupling and data management
% for data intensive simulation workflows in a loosely coupled manner. In
% addition, our application-aware adaptive data placement has demonstrated the
% effectiveness of using user provided access information as ``hints'', and
% runtime data access pattern history to improve data access efficiency and
% overall end-to-end execution. Besides, additional data placement
% optimization \cite{qiansc15} that leverages physical network topology along
% with data access patterns has also demonstrated lower data access costs and
% good scalability. Our work have been deployed and proven with real
% applications, e.g., coupled combustion simulations (DNS-LES) that are part
% of the ExaCT Exascale Co-design Center and coupled Fusion simulations
% (XGC0-XGCa) that are part of the EPSI SciDAC on current high-end systems
% such Titan at ORNL. 


%%% Local Variables:
%%% mode: latex
%%% TeX-master: "../proposal"
%%% End:
