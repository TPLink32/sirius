\subsubsection{APIs}

The new APIs will provide new functionality of specifying selectable performance/quality/- cost tradeoffs from both the application and system perspectives based upon the user guided rules/policy and runtime system monitoring status. It allows the middleware to make best possible decisions from the feedback of storage system knowledge, such that it will embed user intentions and the available system storage. The user would like the ability to get a certain amount of Quality of Service such that they can then make decisions when the expected bandwidth, for example is less than what they desire and will then place in a certain set of rules which the system can make autonomic decisions to help decide what should be done. We want the system to give the users a certain amount of currency in terms of bandwidth, storage space on each level, and latency expectations. These notions will be fuzzy but they will allow the user to make ad-hoc decisions to figure out what needs to be saved.

When user ask the system to write their data, and they would like to get an estimated time which they can then figure out, through a series of rules, whether they want to write out all of the data or wait for a later time to write or write a reduced amount of data. The decision is based upon if the system monitoring model could give us the estimate time during the reading and writing process, so that rules can be in place from the user perspective to make adaptive decisions. 

We will extend ADIOS with new APIs to provide such functionalities. If user decide to write or read data to or from hierarchical storage systems, they will rely on the API to send their intentions for inquiry and examine the system status, including how much data they would like to write/read, desired bandwidth, data compression method, etc, or the user can express their intension of writing data right now no matter what the traffic is now. After receiving the negotiation signal, the API return value will contain the estimate time of write/read data, potential bandwidth, concurrency, etc. User will adapt their decision from the system feedback and determine how much data they will write/read if the estimated time is too long for them. 

The capability of APIs provide dynamic runtime information to user and make the system more transparent to them rather than a black box. The end user could analyze or visualize data in a more comfortable position, where they are able to envision how much longer they are going to wait for data. 
