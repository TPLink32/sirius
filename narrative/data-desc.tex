\subsubsection{Description of Data}
In order for the middleware and storage layer to understand data,
I/O intentions, and act accordingly, describing data in a way that can
be understood by the system is needed. In this project, this is achieved 
by managing data as \textit{objects}.
An object is the smallest unit of data that consists of raw data and metadata,
and a user space variable, such as \textit{temperature}, may consist of a collection
of objects possibly with different accuracy or resolution.
A key advantage of representing data as \textit{objects}, rather than 
bytes, is that it allows data semantics, user intentions, QoS requirements, and
the relationship between user data to be readily captured and embedded with the 
raw data. As an example, retrieving a 3D field data generated from a simulation and then visualizing it
requires the coordinates and connectivity variable (1) to be accessed together and (2) with low
latency in order to achieve a good user experience. The description of data, 
as enabled by managing data as objects, bridges 
the semantics gap between applictions, middleware, and storage systems,
and allows the system to understand user-level data, execute QoS requirements and policies, and
optimize application and system performance.

The description of data is essentially metadata and it may include conventional attributes, such as 
data type, size, dimensionality, relationship to other data, and etc. In particular,
relationship to other data objects can be captured by implementing \textit{soft links}
and different types of relationships can be expressed by different types of soft links. 
In this project, a key metadata attribute
that enables QoS scheduling, data placement, and ???, is \textit{data utility value}, which
captures the priority of objects - it is to our belief that
for exascale science, users must prioritize a small set of objects to be saved on 
higher-level capacity-limited storage layer
to avoid the slow access to large-capacity lower-level storage, such as tapes.
We propose a new technique, described by a utility function plugged in by a user, where objects
can be cast into multiple buckets of data, and each bucket can be prioritized differently according
to its utility value.
Once this is done, the middleware system will construct the description of data, re-organize
and place data to achieve the desired QoS goals and policies.


\paragraph{State of the art:} ADIOS implements a binary-packed data format that allows
data characteristics such as min, max and index to be wrapped around data chunks. A
direct benefit is that each data chunk can be operated independently and I/O concurrenty
can be maximized. We will build upon this capability and further augment the format
to express data utility. Recently DAMSEL \cite{damsel} provides a rich metadata representation
and management layer that captures the relationship between data blocks for scientific applications.
This allows application data to be mapped to storage system efficiently, without overburdening
users to managing complex data model, such as AMR, from the user space. However, the aspect of 
data importantance hasn't been explored, which we believe will be important for exascale storage solutions. 

\paragraph{Proposed research approach:} 
We will design and deveop new APIs through which users can describe the utility of data
based on their expectations. This can be done by plugging user level functions that will
partition user level data into objects and be executed by the middleware to calculate the 
utility value for each object. The utility value along with other metadata will be encapsulated
by the middleware and utilized by the underlying storage system to management data placement.
