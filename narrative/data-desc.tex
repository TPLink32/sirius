\subsubsection{Description of Data}

One of the key advantages of representing data as \textit{objects}, rather than 
bytes, is that it allows data semantics, user intentions, QoS requirements, and
the relationship between user data to be readily captured and embedded with the 
raw data. This description of data, as enabled by managing data as objects, bridges 
the semantics gap between applictions, middleware, and storage systems,
and allows the system to understand user-level data, execute QoS requirements, and
optimize application or system performance.

Data Models then constitute a collection
of objects which are then placed together via a relationship. In other words, we can think of a
simulation which contains a three dimensional mesh, and that the coordinates of the mesh, along
with the connectivity of the mesh will generally be accessed together. This also us to organize
the data in the storage system in such a way that will allow a user to easily to describe this 
collection of data together, and it will allow the user to describe relationships between the objects
in the data model. 

The challenge which all users face today, and will face much more on exascale systems is that 
the users must prioritize which objects to save, and how often (in a simulation which produces more
data in time)  to save the information. As this data gets large,  users realize that these objects will
eventually move from a fast storage into either a slower form of storage, such as tape, or it will be
purged from the system.  We propose a new technique, described by a utility curve, where objects
can be cast into multiple buckets of data, and these buckets can be prioritized with one-another.
Once the data is re-organized, users will be able to use an API to describe the importance of this data.
This importance will allow users to specify: 1) the relationships of different objects with one another,
2) the expected lifetime of the object.

The relationship of the objects to one another can be done by

\paragraph{State of the art:} 
\paragraph{Proposed research approach:} 
We will first create an API where users can describe the utility based on their expectations of how
long they want each piece to live in the different storage areas
\paragraph{Preliminary work and results:}
