

\begin{itemize}

\item describe data size problems for getting scientific results out
\item describe annotating and differentiating data storage according to both
use and inherent data qualities (e.g., contains desireable features)
\item describe idea of using heavy compression for less interesting areas and less or no compression for more interesting areas.
\item describe inherent errors in scientific calculations
\item describe lossy compression within error bounds
\item describe how incorporating plug-ins for application aware data compression opportunities to aid data storage
\item describe Sirocco's current state and future designs as a solid foundation on which to work
\item describe how these new features can enhance a system like Sirocco to reduce data intensity
\item describe metadata challenges that we will also have to face.

\end{itemize}

every section should include:

Problem and background, one or more solution approaches, and investigation goals/questions.

Technical Areas
 
1. Application Intentions.\\
2. Metadata Management (middleware and storage system)\\
3. Pluggable infrastructure for data transformation (storage system)\\
4. Intelligent data mapping to storage/memory hierarchies (middleware and storage system)\\
5. Predictable Performance and resolution tradeoffs\\
6. Data re-generation. (middleware \& storage)\\
7. Learning Motifs and data re-organization.\\

\section*{Introduction}

Exascale scientific discovery will be severely bottlenecked without
sufficient new research into managing and storing the large amounts of data
that will be produced during the simulation, and analyzed for months
afterwards.  
%
In this project we will demonstrate novel techniques to
facilitate efficient mapping of data objects, even partitioning individual
variables, from the user space onto multiple storage tiers, and enable
application-guided data reductions/transformations to address capacity and
bandwidth bottlenecks, while constraining the error to be within user
provided bounds.
%
We will address the associated I/O and storage challenges in the context of
current and emerging storage landscapes, and expedite insights into critical
scientific processes, demonstrating the validity of our approach in key DOE
domains. 

The metric we are most intrested in optimizing is time to knowledge. Current
approaches to addressing the I/O bottleneck fall into two broad
categories. Parallel file system approaches that optimize the throughput for
an entire system, and I/O middleware approaches that optimize the
performance of a single application. Both approaches have seen success but
are unlikely to overcome the major obstactles in reaching exascale. Instead
of trying to optimize throughput, we will seek to reduce the time to
knowledge. This is the most significant metric for scientific applications
where the desired outcome is not storing data, but rather executing a
knowledge extraction process on the data. Moreover, we aim to perform the
optimization not for a single application, but rather for workload of a
multi-user multi-application system environment.

Our approach leverages the expected characteristics of exascale storage
hardware. The storage layer will be partitioned into multiple heterogenous
tiers with vastly different performance characteristics. This difference
between layers will be further exacerbated by the constraints on capacity
and data lifetime within a layer. Tape archival storage will still maintain
data long term, but access to this data will be orders of magnitude slower
than the next layer and naively accessing data from archival storage will
greatly impact productivity. 

We will achieve reduced time to knowledge using a combination of techniques;
\begin{enumerate}
\item Data annotations specified at the application level to quantify the
  relative important, utility and lifetime of data objects;
\item Partitioning of data objects across the storage hierarchy utilizing
  the additional knowledge embedded in the annotations;
\item Evaluating the tradeoffs at runtime to guide data placement, movement,
  and migration across storage layers using models, heuristics and continuous
  learning;
\item Utilize the additional knowledge available about the data to perform
  application-aware data compression and I/O prioritization;
\end{enumerate}


%%% Local Variables:
%%% mode: latex
%%% TeX-master: "../proposal"
%%% End:
