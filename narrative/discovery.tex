\subsubsection{Discovery}

Hierarchical Storage Management (HSM) systems offer a strct caching approach
for managing different storage capacities trading off performance for capacity.
By maintaining a single namespace across all tiers, it is possible to list a
single directory view with files stored at different tiers. While this approach
to managing multi-tier storage works, it is far from ideal for scientific
simulations.

Our goal with this proposal is to offer a similar capability, but use a finer
granularity. Instead of a single file such as might be used to store an entire
timestep output for a simulation, we will demonstrate offering the same
capability at a subset of a single variable level. By shifting to a
finer-grained approach, we will enable more effective use of close/fast/small
storage tiers. Traiditional HSM stores an entire file on a tier making room if
insufficient space is available. With this shift to a partial variable
granularity, a 1 PB output with 500 GB of ``high interest'' data can limit this
costly tier ussage to just 500 GB greatly enhancing usability.

Tagging a variable subset as ``high interest'' requires intervention from the
application and/or middleware to determine what data meets this criteria. The
storage system itself simply needs to offer an ability to perform different
actions based on this information.

The challenge for discovery is that potentially, data will migrate from where
it is initially stored to a new location within the storage system. Sirocco
offers an ability to search for data that has moved as well as forcing a
particular resilience-based replica be the ``authoritative'' version. We will
investigate if the current Sirocco functionality is capable of supporting the
new operating modes we wish to offer. Initial expectations suggest having
bounded time guarantees for finding data are critical for offering the quality
of service guarantees we wish to offer. This new work will be an expansion of
Sirocco's currently planned features.

The other aspect of discovery is the negotiation between the user and the
storage system for a data quality/retrevial time tradeoff. The naming service
will work hand-in-hand with the discovery, data migration, and time estimation
services to offer the best possible options for data retrevial based on quality
of service requested.
