\subsection{Resource Management}

Sirocco currently offers automatic resource management capabilities. However,
these features are strictly driven on resource capacity and data resilience
requirements. For this project, we seek to expand the resource management
capabilities to support additional metrics to guide resource management
decisions. Our basic goal is to offer data-centric metrics that help guide the
forced resource management decisions.

As stated in the introduction, we seek to make a storage stack that works with
the user. Towards that end, we must incorporate data attributes into this
process. For example, particularly for a capability run, the storage system
should give priority to data subset tagged with high importance to stay in fast
storage tiers. Enabling this priority will be driven by other job
characteristics such as job size or a manually set priority inserted by system
administrators. It should also store the highest quality data possible given
the performance requirement. Ideally, the storage system will store the raw
data at full fidelity. When this tier approaches capacity, data should migrate
not just based on age, but also based on this priority. This will help ensure
the shortest time to insight by making sure the most important machine runs
have priority to use system resources.

Existing efforts on scheduing resources~\cite{io-cop,dorier} attempt to offer
admission control to maximize storage performance. Unfortunately, these efforts 
are limited to participating applications and only applications from a single
platform. For this approach to be effective, the storage system must offer an
approach that handles clients from all connected platforms and does not
require modification to use new storage access APIs. By using a priority
tagging system, we will offer defaults suitable for all applications that can
be informed by simple extensions to the job scheduler or more advanced
management through new APIs.

What else do we need to talk about?

