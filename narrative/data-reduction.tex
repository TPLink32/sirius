\subsection{Data Refactoring} \label{sec:data-refactor} 

The classical workflow where the entire dataset is written to storage for
later analysis will no longer be viable in exascale, simply because the amount of generated data will 
be too large due to capacity and performance reasons.
In the future, it will be vital to take advantage of \textit{a priori}
information when 1) writing and reading from the SSIO layer in order to gain
higher performance and predictability, and to prioritize the data that
is most useful for end users so that I/O can be finished in the time available, and 2)
performing {\em in situ} operations and analysis before storing the information.
In the fusion use case described earlier which shares commonalities with many other
DOE applications, \textit{a priori} information can indeed be provided by application
scientists regarding which data should be sent over to the SSIO layer, which we refer
to in our proposal as {\bf SIRUS}, so that
minimally, the most science relevant data can be available for subsequent analysis.
This allows the science to be done even when the storage is in busy state - we believe
in a shared multi-user environment, it occurs often \cite{liu_hotstorage}, and gives users
high performance variability.
In this project, this capability of prioritizing data is accomplished by data refactoring, 
which includes re-organization and reductions of the data.
There are many methodologies to refactor data and the best choice will generally
be application dependent. However, our observation is that, once the choice is settled for an application, 
it will not typically change from run to run. 

To refactor data effectively and efficiently, it is important to understand
the critical point when the time and resources required to
identify and perform the ``best methods'' outweighs the intended gains. Another
critical research question concerns the quantification and control of the loss
of information resulting from refactoring the data and using a reduced dataset. 
A basic issue allied to refactoring is to understand how much information is
actually present in a dataset and whether a refactoring based on a reduced
order representation might prove effective. It is useful to classify scientific
data into two basic categories: structured data (in the sense that it satisfies
a known or relatively simple model), or unstructured data (in which information
content follows no obvious or explicit model).  Although scientific data
generally contains random components (due to finite precision, measurement and
calibration effects etc.), useful scientific data is never purely random.
Broadly speaking, the path from data to knowledge consists of extracting
underlying models or patterns from the datasets, and interpreting the resulting
models. 

Ideally, scientists would like to perform the entire analysis stage {\em in situ}
to extract the relevant information and in so doing would effectively
circumvent the large data issue completely.  The catch, of course, is that this
is unlikely to be possible since, by their nature, large scale simulations aim
to discover new information that is often hidden in the form of higher order
effects amongst the data deluge. In particular, this means if data thinning or
truncation is applied haphazardly then one is in danger of effectively throwing
out the information one is seeking. It would seem that one cannot store the
entire data set in an easily accessible source, due to sheer size, yet one
cannot reduce it prior to archival without risking losing the information one
is trying to extract.  Viewed in this way, the problem would appear
intractable. However, the above discussion fails to recognize that much of the
data is redundant in an information theoretic sense.  That is to say, the
amount of information contained in the dataset is often significantly less than
the amount of data.  The difficulty stems from the fact that one does not know
about this redundancy in advance \emph{without the benefit of priori
knowledge}. This knowledge can often be provided by the user or can be acquired
through experience of dealing with differing runs of the same code. 

Deep application knowledge means one can sometimes achieve drastically superior
reduction in the total data compared with what one might achieve otherwise.
However, even in the absence of such high level knowledge, we must equip SIRUS with generic data reduction techniques and re-organization
techniques. For instance, certain basic information on the semantics of the
data is needed and must be supported by the overall infrastructure. Likewise,
information regarding typical access patterns for data in a given application
should be placed in SIRUS. Scientific applications often read and
analyze data in a structured fashion and failure to respect this pattern, or to
exploit it through prefetching, can significantly increase the overall
execution time of the application. Accurate prefetching can only be achieved
with access pattern analysis (\S\ref{sec:managing-data-life}). 

%% I don't understand this sententce This requires SIRUS
% to be capable of
%prefetching data in an asynchronous fashion to take the bytes on the file.

\paragraph{State of the art:} {\bf Data Reduction Techniques.} Data compression
broadly consists of three basic steps: (1) Pre-processing to extract as much
structure as feasible. This step is variously referred to as filtering,
preconditioning, and preprocessing. But in all cases, the objective is to
represent the data in a form that becomes more amenable to compression.  (2)
Removal of redundancy such as omission of duplicates, etc.  Finally, (3) the
resulting data is compressed using entropy coding. The level of compression
that can be achieved in the final step is limited by the Shannon information
content of the data \cite{Gray:book}, and many standard tools are available for
efficient and perfect entropy coding. The potential for compression through
removal of redundancy is generally minor. The main scope for achieving
significant compression (e.g. a compression rate of two) therefore
lies in the pre-processing step.

Research in the construction of pre-processing techniques is still in its
infancy, especially in context of exascale simulations. Techniques that have
been advocated range from simple difference filtering, bit-sorting, prefix
transformation through to simply reordering the data.  What these, and other
catch-all methods fail to exploit is a priori information on the nature and
provenance of the data. 

Previous
work~\cite{lakshminarasimhan2011compressing,%
gong2012multi,jenkins2012byte,gong2013parlo,boyuka2014transparent,%
tang2014improving, sato2014user}
has focused on several alternative techniques to not only reduce the data
but also re-organize the data. Previously we have examined:


\begin{itemize} 
	
\item Precision based re-organization. This is where the most significant bytes
	of the data are all grouped together from each object. This data
	generally will have a higher utility than the data with the least
	significant bytes. 
        %This process has the following three steps which
	%need to be efficiently implemented when data gets placed to or retrieved from the storage.
	The data needs to be re-arranged which involves potentially intensive 
        memory operations and needs to be done as much in situ as possible without
        involving the slow storage systems, and inter-processor communications.
	%At most this requires
	%two copies of an individual dataset in memory, and we will investigate
	%techniques to allocate and deallocate this memory if the user will
	%specify that the data will be overwritten after it is written to the
	%storage system. This is often the case for many of the  quantities
	%written from the simulation, but there are many cases where, for
	%example, we want to write all of the particles from a Particle In Cell
	%(PIC) simulation.  Since the particles will be used later in the
	%calculation, we need to duplicate the storage. Our observation with
	%working with the XGC1, GTC, and PiconGPU simulations 
	% is that we can temporary increase the
	%storage of the particles, and then release them since the temporary
	%arrays used in the calculations are often freed when a PIC iteration is
	%finished. 
%
\item Frequency-based re-organization.  Another common approach to classifying
	the importance of data is to re-organize data according to it frequency.
	This is commonly used in streaming data, and data reduction techniques
        such as those used JPEG-2000~\cite{jpeg2000}, which allow mechanisms
	to support spatial random access or region of interest
	access at varying degrees of granularity. It demonstrates the possibility of storing
	the same data using different quality. In relation to this project, this
        capability allows us to place the lowest frequency chunks in the fastest storage, and the
	highest frequency chunks on either the slowest storage tiers, or if the data
	sizes are prohibitively  costly, not even write out these pieces. 
%	This binning of data in frequency space can take as an input
%	(storage  {\color{red}ending )? CHECK-tsr} bandwidth knowledge, storage size limitations, and user
%	intentions of how long the majority of the data will be frequently
%	accessed. 
	 In order to fully take advantage of frequency based
	re-organization, the team has devised a scheme where the data pre-conditioner
	first sorts the data in bins with fixed length, and then wavelets and spline fits are used to
	re-organize the values.  This re-organization effectively produces
	a smoother data, allowing for much better reduction later.
\end{itemize}

\paragraph{Proposed research approach:} It is now widely recognized that any
opportunities for {\em in situ} analysis should be exploited as far as
possible. What is less well understood, is how one should incorporate a priori
information and how one can identify possibilities for {\em in situ} analysis.
A preprocessor would be employed in situ and used in a similar fashion to
pre-processors in standard compression approaches such as FPC
\cite{BurtscherFPC} in which an XOR operation against the previous data is used
as a pre-processor.  We shall exploit a priori information in the preprocessing
stage through the use of relatively computationally intensive preprocessors
(compared with FPC) that seek to extract structure in the data {\em in situ}.
This approach takes advantage of the increasingly wide cost gap between CPU
operations versus storage-based operations. 
We will assemble a suite of general techniques for refactoring of data that
will be embedded in the system and can be selected either by the user or by the
application itself. The techniques will range from crude catch-all methods such
as those employed by general compression routines such as FPC, through to
multi-resolution analysis using wavelets, including the methods on which we
have worked (cited above).  These will then be mapped over into the different
layers of the storage hierarchy.  We will also cater for user supplied
``plug-ins'' for detection of features and store higher resolution data around
the feature and lower resolution values which are further away from the
features. The plug-ins could be used to override the general embedded
techniques and would provide a mechanism for the transfer of a priori knowledge
from the user to the hierarchy without imposing a burden on either. 

To implement these plugin capabilities, we will build upon our existing data staging 
capability and further allow data to be
efficiently staged, re-organized and mapped to a multi-tiered storage. Various
processing scenarios such as in-situ, in-transit or a combination of the two
will be researched and developed to meet our needs.
%The open questions that will be addressed include: 
%1) do we move the data over to
%other nodes to then reduce and re-organize, 2) do we first reduce the data and
%then later re-organize this data on different nodes?, or 3) do we use different
%cores on a node to do data-refactoring.  
Similarly we will examine the cost of reading the data. In the case where only high importance being retrieved,
overall time to move the data from SIRUS to the
application is expected to fast. A key challenge is in the assessment of retrieving
both high and low important data that reside across storage tiers and possibly migrate to
new locations as a result of capacity pressure. There is a
clear penalty of having to discovery data (\S\ref{sec:naming-discovery}), take data in a hierarchical layout and then re-compose
this back to a single-layer view.  
%Additional research questions which must be
%addressed: Will some types of data (e.g. PIC data) be more
%conducive  for data-refactoring than others? How much semantic-domain knowledge
%will we need to get a dramatic improvement of the storage system over not doing
%anything special? 

We will measure these methods using the fusion XGC, SPECFM3D~\cite{SPECFEM3D}, and the
fusion GTC~\cite{klasky2003grid}, and the
corresponding analysis routines which they use for their largest simulations.
Since all three of these codes are part of the
OLCF CAAR~\cite{CAAR} project for early access, there is strong belief that these
simulations will be close to exascale simulations and we will be able to run
these application on SIRIUS to examine the full impact of data-refactoring.

\paragraph{Challenges:} Maximal data reduction in SIRUS
would result from storing only the code and the input parameters, and simply
recalculating all of the results. This extreme is not feasible in practice
owing to the size of the computation involved, but sheds light on the path to
achieving a step change in data reduction. We shall investigate the possibility
of writing a minimal set of information, and placing an ``auditor'' code in a
container-like environment so that data can be re-generated with a {\em local}
approximation of the full simulation. This is an entirely new approach to
tackle the data overload by using preprocessors in which information is not
simply stored in the form of raw data, but rather in the form of appropriate
parameterized algorithms. An ``auditor'' is a code which mimics the exascale simulation, 
but typically can run using less degrees of freedom. This means that the code is a good
local (in space and time) approximation of the exascale simulation and will be much less
expensive to execute (so that it can be run in situ), and follows the behavior of the simulation
for short periods of time. This allows us to ``audit'' the simulation and when the differences 
become large in a local region of space, we write out the results of the data in this local region.
The hope is that this will be done only locally and that with a good ``auditor''  this will not happen
too frequently. This then allows us to use the ``auditor'' to reproduce results during the analysis 
process using an inexpensive computation and put less pressure on SIRUS
%
% 
%   A priori information on the nature of the data is
%used to choose the basic algorithm while the actual values of the data are used
%to tune the parameters used in the reconstruction procedures. We shall refer to
%this paradigm as an Auditor.  We will examine the use of several auditors for
%the simulations which we will work with. One approach is to use a fluid-code to
%audit a kinetic code, which can be a good approximation over short periods of
%time. 

