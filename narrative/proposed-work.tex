\section{Proposed Research and Methods}
\label{subsec:challenges}
<<<<<<< HEAD
\label{sec:proposed}

\begin{enumerate}
\item Re-factoring of the data
  \begin{itemize}
  \item What kind of reorganization would work when?
  \item Would it take too much time and resources + memory and how can we
    manage the cost of the reorganization
  \item Will the users loose too much value from the data during refactoring
    and how can we control/bound this loss of value
  \end{itemize}
\item Quality of Service
  \begin{itemize}
  \item Can we make any gurantees in a scalable decentralized system?
  \item What quality of service metrics can be make gurantees on in a scalable
    way?
  \item What kind of error rates will we have in our estimation and what error
    rates are acceptable?
  \item How much will the qos scheduler, admission control, manager cost in
    terms of resources and time
  \item How do we make the scheduler scalable with the size of the system?
  \end{itemize}
\item Discovery and naming
  \begin{itemize}
  \item Which data block and what quality/type do we respond to a query
    with?
  \item How will we scale a fully distributed storage system's metadata
    management
  \item Can we accelerate the discovery process with the addition of an
    application/user level metadata store that tracks where data was
    originally written and how was it structured?
  \item Can we provide time estimates for read/write calls in the presence of
    an unbounded discovery system? If not how do we bound the discovery time?
  \item Time cost of discovery vs time cost of actually reading the data
    (latency vs bandwidth essentially)
  \end{itemize}
\item Migration
  \begin{itemize}
  \item Can we do scalable migration without making the discovery process
    unbounded
  \item What are the parameters and input for migration
  \item Can we use the additional application level knowledge of data to purge
    portions of the data from storage without making the data too much less
    valuable?
  \item With so many objects and no centralized directory how do we know what
    is actually in the storage to be purged?
  \item How do purge in a way that doesn't require a central authority,
    doesn't interfere with other I/O and isn't bottlenecked by some sort of
    global consistency
  \end{itemize}
\end{enumerate}


=======
{\bf {\color{red}Norbert will clean these pieces}}
>>>>>>> d16c2739bab5a833eb4d1de8805a12f25b751fc2
Our research into the storage system and middle ware layers must address
many of these challenges including
% \begin{enumerate}
% \item How do we describe the user intentions at the API layer and have this
%   communicated down to the middle-ware and storage layers?
% \item How do we allow users to define their user defined compression
%   techniques and have this data read back from the storage system layers?
%   Where do we execute the decompression techniques 
% \item How do we evaluate the tradeoffs at runtime to guide data placement,
%   including the movement of data across the network, across the different
%   storage tiers? (data movement and quality tradeoff)
% \item Can we use different forms of learning techniques as daemons on the
%   system to perform data migration from one storage tier to another? For
%   example, if we see that a user is looking at one time slice after another
%   for a certain object in their dataset which is in the slower tiers of the
%   hierarchy, do we automatically propane up the data which has not been
%   requested in the hope that this will be requested data? (prediction and
%   prefetching, user defined compression on movement)
% \item Can we understand the true need of campaign storage and understand the
%   possible impact of the campaign storage if we run a hadoop-like
%   file-system instead of a lustre/gpfs file system? (additional layers)
% \item Can we limit the amount of data duplication? Data space is going to be
%   very constrained on the exascale system so we must ensure that minimal
%   copies are made of data, and when data is duplicated, they are removed
%   through a garbage collection routine. (space management)
% \item Can we build a model to give us the time estimates during the reading
%   and writing phases, so that rules can be in place from the user
%   perspective to make adaptive decisions. For example, if the file system
%   says that reading will take 3 months, users can they place in rules to
%   then read in a subset of data, and they can understand which data can be
%   read quickly and which pieces will take more time. (estimation)
% \item Can we understand how to place code in the system which will allow
%   data-regeneration to take place. (regeneration)
% \item How does the storage system do a better job in managing request from
%   all of the users on a LCF than today? We realize that today users who have
%   a better middlware system can often lock other users from getting high
%   performance when they are running. If we have the concept of currency
%   which is eventually used in the same extent as node-hours, then users will
%   have to be able to think about how much storage and how much bandwidth
%   they can choose. One question that needs to be understood is if there are
%   times when the system sees that there are very few storage system
%   resources being used, then the lucky users can get the bandwidth cheaper
%   than at times of heavy usage. How can we enforce this? (Fairness)
% \item How do we manage all of the metdata not only from the principle
%   objects, but from the sub-objects? How well will this scale when we have
%   users who can potentially create billions of objects from their
%   simulation? (scalability and discovery)

\paragraph{Deliverables and artifacts}

\begin{enumerate}
\item Add plug-in architecture to Sirocco to support selective data compression
\item Add specific tier destinations command to Sirocco (assuming that the caching mechansism cannot achieve the desired effects).
\item Develop profiling system to determine how a data set is used during preparation runs prior to a capability run to determine how to optimize data placement for capability run analytics.

\end{enumerate}

{\bf \color{red}we need to condense these challenges into a few basic principles - 4 at most - SAK}

We propose to overcome these challenges by pioneering a new {\bf knowledge-centric approach} to a dynamic storage system
and I/O layer which takes into account user prioritization and utility to optimize the SSIO layer. The approach is based on 
four underlying principles:

\underline{Principle 1: A knowledge-centric system} , let user knowledge  to define data policies. talk about this pieces...

\underline{Principle 2: A SSIO system which integrates system knowledge together with application knowledge} to understand
how to organize data in the deep storage hierarchy

\underline{Principle 3: Predictable Performance in the SSIO layer} so that intelligent choices can be made by both the user and the system

\subsection{Data Description and Application Interface}
\label{sec:data-description}
In order to enable the middleware and storage layer to make appropriate decisions based on user's intentions, data must be
described in a way that communicates these intentions to
the system. In this project, we label the smallest unit of data
saved in the storage system as a \textit{chunk}.  A chunk not only consists of 
raw data, but also metadata that provides additional
knowledge about that data.
A user space (application) variable, such as for example \textit{particles}, may
consist of a collection of chunks, possibly with different accuracy or
resolution, and representing a different portion of the variable.
A key advantage of having attributes at this low level is that it allows data
semantics, user intentions, QoS requirements, and the relationships between user
data to be readily captured and embedded with the raw data so that each chunk
can be managed independently or collectively, depending on the need.

  We illustrate our approach through a simple example shown in Figure~\ref{fig:ssio-bucket},
  in which the user attempts to write \textit{Particles}, \textit{Field 1},
  \textit{Field 2}, and \textit{Mesh} arrays. In particular, \textit{Mesh} can be a group
  of variables in the application code, with their relationship implicitly described
  by the data model. Using the conventional POSIX-compliant block interface, file
  systems and storage are oblivious to user space variables and the relationships between
  them. Recent advances in I/O middleware systems such as ADIOS \cite{liu_helloadios}
  support the capturing and understanding of data semantics and relationships between  
  user space variables. As such, users can request any portion of a variable to be retrieved
  during post-processing, or in situ alongside simulations~\cite{docan2012dataspaces}. 
  In this project we will leverage this ability of plugging rich metadata from ADIOS (and 
  DataSpaces) and related experiences, and will further describe the utility of data for data 
  chunks (presented in Section~\ref{sec:managing-data-life}). This will enable user data to 
  be effectively managed at the storage system level, and will allow the SSIO layer to 
%  
%   
%     retrieving and visualizing 3D field data generated from a simulation
%requires the mesh (consisting of coordinates and connectivity variables) be
%accessed simultaneously and with low latency to achieve a good user experience.
%Managing data at this low level
bridge the semantic gap between applications, middleware, and storage layers, allowing the 
system to understand user-level data, execute QoS requirements and policies, and optimize 
application and system performance from the user (application) perspective.

Furthermore, we will explore new APIs that will enable users to include a {\it plugin} function when 
they write data. For example, when a user writes Field 1, they can pass in a {\it plugin} function 
to re-factor their data (\S~\ref{sec:data-refactor}), which re-organizes data and facilitates the 
placement of its different chunks, e.g., F1 and F2,  onto different storage tiers, possibly with 
a different lifetime specified for each chunk, with the overall objective of maximizing its 
utility to the user (application).

To further illustrate the potential benefits of our approach, the two chunks of Field 1 (F1 and F2) 
described above may be defined based on system {\it precision} so that F1 is of higher value 
than F2 (e.g. F1 contains the first 3 digits of each double precision value, and the exponent and 
sign, and F2 contains the rest of the data). 
%The user might also indicate through additional semantics that F1
%can use a sorting pre-conditioner which allows the data to be sorted, for better compression.
%
The utility of F1 might be defined as (1, (8 hours, 30 days, 100 days, 1000 days), 
(2,(1 hour, 4 days, 100 days, 300 days). This allows the initial data chunks to be initially 
placed one way in the storage layers and later migrated to different layers as shown in 
Figure~\ref{fig:ssio-bucket} .
 
In addition to be embedded with the data, the data description may also be stored within 
a separate metadata service to aid QoS requirements. Such a data description would contain
conventional attributes such as data type, size, dimensionality, and the relationship to other 
data. In our example, the relationship of the Mesh chunks (M) and the field chunks (F) can 
be captured in this way and would allow the system to understand that the user will only 
want to read in F with M, and want the same portion of data from each chunk.

When a user is reading this data at a later time the user can define how much time they will 
be willing to allow the read operations to take as part of the data description. For example, 
the user might want to visualize F, which means they need to read in F along with the mesh 
M. The user may also specify which time slices to read and tell the system how long they 
are willing to wait. The system can then use this information and deliver the highest value 
chunks it can while satisfying the time constraint.  Note that the user may also include 
requirements/constraints about the accuracy of the data to read allowing the system to 
appropriately place and read data chunks to meet these requirements.
%We will also allow additional information  from the user so they can express rules about the possible

%The new APIs will enable the bridge by specifying selectable
%performance/quality/- cost trade-offs from both the application and system
%perspectives based upon the user guided rules/policy and runtime system
%monitoring status. It allows the middleware to make best possible decisions
%from the feedback of storage system knowledge, such that it will embed user
%intentions and the available system storage.  We want the system to give the
%users a certain amount of currency in terms of bandwidth, storage space on
%each level, and latency expectations. These notions will be fuzzy but they
%will allow the user to make ad-hoc decisions to figure out what needs to be
%saved.

%The data description may be stored with the individual chunks or within a
%separate metadata service or both to aid QoS requirements. It would contain
%conventional attributes such as data type, size, dimensionality, and the
%relationship to other data. ]



Our research will address the following questions:
1) How can users interact with the storage system efficiently?,
2) How can users express their intentions into actionable items which can take place in the middleware and/or storage layer?, and
3) How can users annotate data with a utility metric?

\paragraph{State of the art:}
Our current data annotation techniques demonstrated in
ADIOS~\cite{lofstead:2009:adaptible} implement a binary-packed (BP) data format that
allows data characteristics such as min, max and index to be wrapped around
data chunks. A direct benefit is that each data chunk can be operated upon
independently and I/O concurrency can be maximized. We will build upon this
capability and further augment the BP format to include data utility metrics.
Recently,
DAMSEL~\cite{damsel} has provided a rich metadata representation and management
layer that captures the relationship between data blocks for scientific
applications.  This allows application data to be mapped to storage system
efficiently, without overburdening users with the management of complex data models, such
as adaptive mesh refinement (AMR), from the user space. 
FFS~\cite{ffs} implements a type system that provides highly efficient binary
data communication and XML-like data description, facilitating data sharing
between collaborators. Additionally, NetCDF~\cite{netcdf} and HDF5~\cite{hdf5}
both use a fixed data format to provide hierarchical data descriptions for scientific data.
Despite these efforts, the aspect of data importance has yet to be 
explored, and we believe it will be vital for exascale storage solutions.

\paragraph{Proposed research approach:} 
We will design and develop new techniques for data management, description,
and access that allows users to describe the data utility based on their
expectations. This can be done by allowing users to apply plugins that
provide data-specific functionality to the middleware and storage system.
These plugins may be executed by the system to
calculate the utility metrics for each chunk, handle custom compression
and decompression, and perform other data-specific services, thus allowing
the system to more effectively manage data placement and retrieval.
%
%We will design an updated I/O API that will expand on the current POSIX I/O
%semantics by adding new information to the I/O request.

\begin{wrapfigure}{R}{0.6\textwidth}
        \begin{centering}
        \vspace{-4ex}
        \includegraphics[scale=0.7]{graphics/SSIO-bucket.pdf}
        %\vspace{1ex}
        \caption{The system manages the data across the various stages of the data lifecycle}
        \label{fig:ssio-bucket}
        \end{centering}
      \vspace{-1ex}
\end{wrapfigure}
%\lipsum[1]

We will explore the space of application level hints that can be easily provided
to the storage system. These hints will enable the storage system to make the
most appropriate optimization decisions for this user in the context of the
multi-user environment rather than trying to rely on a one-size-fits-all
approach. In particular, we expect an application to provide hints on the
the time that an I/O operation will occur and
the expected data lifetime for output sets. For read calls we envision
hints that provide latency and precision requirements to the system. 

%In particular, the relationship to other chunks can
%be captured by including \textit{typed links}, each representing a different
%relationship type.  
In this project, a key set of metrics will be stored as a metadata
attribute is \textit{data utility}.  It enables QoS scheduling, data placement
decisions, data lifetime decisions, and captures the value of the chunk.
% It is
%our belief that exascale science will require users to prioritize a small set
%of chunks to be saved on higher-level capacity-limited storage to avoid the
%slow access to large-capacity lower-level storage, such as tape. 
% We propose a
%new technique, described by a \textit{utility function} provided by a user, where chunks
%can be cast into multiple data chunks and each bucket can be given a different utility value.  
Based on the utility, the middleware
system will construct the data description and re-organize and place data to
achieve the desired QoS goals and policies.
 We will formulate the concept of utility (from an
application perspective) that is assigned data chunks to quantify the
benefits of placement and/or movement actions to applications. These
utilities may be based on the importance of the data chunks, the frequency
with which they are accessed, etc., and can be used during runtime data
management to evaluate and compare different placement/movement options. 
For example, data chunks with higher utility value may be
placed closer (e.g., in faster memory) to the application.
Similarly, when it is time to evict data chunks from a higher storage
tier, the decision about which chunk to evict may take the utility into
account. Our initial work in defining such a utility is presented in~\cite{tongipdps15}.
 We will also explore autonomic approaches for balancing the placement and 
 movement of data across the memory hierarchy autonomically, which will be driven 
 by the data utility and will leverage both user hints and information gathered at runtime (e.g., runtime data access
history, network topology, etc.).

%
%A key insight in this proposed project will be the increased interaction of
%the application with the storage system.
%The users would like to have the ability to obtain information about, and even
%negotiate with the system to determine the
%level of QoS is possible given
%a prospective set of I/O operations and current state of the system, so that
%they can then make decisions about when and what data to access.

Towards this end, we propose to
address the QoS requirements by providing applications with mechanisms to
specify the quality of I/O service, interrogate the storage system, and react
to the responses. Guided by our past work we propose to explore the design of
mechanisms for this purpose.  In our example, the user will also request that
the data in (P1, F1, G1, and M1) are given a higher bandwidth requirement than
the other chunks of data. We will investigate how much additional information users
can provide at the time of writing in order for the system to place these chunks within
the QoS requirements. We will also investigate techniques which would allow the lowest
value chunks to never be written to the system if the time it takes to write exceeds the user
requirements.



We will also develop a querying mechanism that allows the SSIO layer
to get estimated timing information of an I/O operation from the storage system, which
manages all storage resources and understands data via data description, as well
as capacity availability before the I/O operation will
be issued to the storage system. 
%This new query mechanism will allow users to
%adapt their applications to system dynamics, e.g., bandwidth/space fluctuations, and
%further reduce and expand data.

%We will explore how these query functions can be integrated into common applications with
%minimal code disruptions and a set of policies that applications can use to
%respond to these new information. For example, the user might  tell the system that
%the total time to write P, F1, F2, and the Mesh can never exceed 500 seconds. When the
%system sees that this can NOT be met, the system can drop P4 (for example), such that
%the time to write is met, and only the lowest value chunk is not written.

Users may also describe the data-refactoring routine by either requesting a system-provided
routine, or their application-aware plugin which the user provides to the system. We will
create APIs for the plug-ins to use so that they will be unified in the way to create them
amongst all of the users. Plug-ins which are given back to the community may be 
eventually incorporated into a system-level plugin if the community sees the benefit for
these more general plugins.

%Users will also need to describe information about the data in order to choose
%the optimal data-refactoring routine. Currently, we understand that
%users can tell us that the data represents: 1)~phase-space (e.g., the particles
%in our example), 2)~spatial-changes (e.g., the mesh), and 3)~space-time (e.g.,
%the fields on the mesh). We need this information to select
%the best possible data
%refactoring methods. Although users can insert a user-defined data-refactoring
%method, system-level methods can also use this information to make better decisions.
%We will investigate new techniques and new semantics capable of using
%this information to best prioritize and re-organize data.


% Removed the following because there are details on it.
%Finally, we will study the use of an external data annotation system that can
%provide information to the storage system without requiring recompilation of the application.


We will explore these techniques through the ADIOS
framework using many of our existing applications, including
XGC1~\cite{chang2006integrated}, GTC~\cite{klasky2003grid}, and SPECFM3D~\cite{SPECFEM3D}.
%We will design an API
%that will be integrated with the storage system to provide I/O time estimations so that
%users can decide which data priorities will be written and read. The utility
%value for each chunk will be used to sort the chunks into prioritized bins.
%Chunks in a particular bin will be assigned an appropriate storage
%tier and lifetime.
%As an example, Figure~\ref{fig:ssio-bucket} illustrates a use case for the XGC1 code.
%The user can pick a data-refactoring scheme (see
%\S~\ref{sec:data-refactor}) which will classify the particle data into four different
%value levels. The user can then pick a data utility, such as three months
%for the highest value level and twelve months for the lowest value level.
%The user will then specify all of the chunks one wants
%to write (particles, two fields, and a mesh in this example) and the system
%will return a time estimate for these write operations -
%in the case that this time estimate is too high, the user may
%compromise by specifying the highest value element to be written, skipping the rest in order to save time.
%Then the data will be placed
%onto the storage tiers, where for example the two highest value buckets
%will be stored in the Parallel File System, and the next highest value items
%will be placed to a lower tier within the storage hierarchy. 
%Depending on the capabilities of the actual storage system, 
%our proposed system will decide if the data will go through the parallel file system 
%towards a lower tier or can be staged to that tier directly. 
% Description of the utility function research

%An important research question that we must address is the semantics of
%describing the utility function, and the possibility that the users will
%accept this, and if the system can then use this information, and make 
%decisions based on this utility function. We realize that the system will
%try to optimize across all of the different user request, and the utility
%function must have defaults that users can see and change
%when the defaults would have a possible negative impact on
%their knowledge discovery. 

%The assumptions which we will make in choosing a utility function is that:
%1) There will only be a small number of priority levels that users will use
%to re-prioritize their data; 2) The number of tiers in the storage system will
%be small (e.g. NVRAM, PFS, Campaign Storage, HPSS), 3) The default
%lifetime of objects in each tier will be specified by the resource by
%a tuple (size, lifetime).  The user can then utilize this information to
%place information to the system such as: store the highest priority data
%as high up on the hierarchy for 6 months, and then migrate this data to tape.
%The user can also define the utility for CR files as that they have high
%priority but their lifetime is for a short period of time; i.e. when the
%next CR file appears on the system or the next two CR files. Research in
%describing this utility function will be in conjunction with users
%and we will evaluate how policies can be placed in the system and how well
%this utility function can be observed by the user.
 

Finally, we will explore a complementary notion of utility from a system-wide perspective 
that will capture and quantify, for example, the impact of placement decisions on 
the overall utilization of storage across multiple applications, and will enable system level runtime optimization. 
%\begin{wrapfigure}{R}{0.6\textwidth}
%        \begin{centering} 
%        \vspace{-4ex}
%	\includegraphics[scale=0.7]{graphics/SSIO-bucket.pdf}
%        %\vspace{1ex}
%        \caption{The system manages the data across the various stages of the data lifecycle}
%        \label{fig:ssio-bucket}
%        \end{centering}
%      \vspace{-1ex}
%\end{wrapfigure}
%\lipsum[1]


%
%
%
%We will design new techniques, available to both users and the storage system,
%to raise or lower these sub-chunks up and
%down the storage stack. The SSIO layer will then manage the system metadata so
%that the user can transparently access any of these sub-chunks without
%the need to know where the data currently resides.
%At read time, the middleware will be required to 
%provide a time estimate, and then allow the users to decide if
%they will read less (e.g., only the highest value sub-chunks)
%if the time estimate
%for reading does not meet requirements. Existing Sirocco functionality may be
%sufficient for these purposes, but it must still be proven in practice.
%

%
%
%Finally we will investigate the use of additional information
%on top of the data model.
%This information will be used to indicate data associations that exceed
%traditional data models. For example, the fields and mesh in our example have
%an inherent relationship. A common data model will describe all of the
%variables in the mesh (the coordinates, the units, and the types of elements if
%this were a finite-element mesh). We will explore new mechanisms that will
%group data together making it explicit that M1, F1, and G1 need to be organized
%together.  In other words, it will not be helpful if data from M1 and M2 were
%organized together along with only F1. When the users wants to visualize data
%from the field, F1, they will need only M1, and M2 would be additional
%information. Likewise, if the user wants to see F1 and F2, then they will need
%the mesh from M1 and M2. This requires that the time to access all of this
%data is approximately the same.
%We will investigate what choices the SSIO system must make in order to
%keep these pieces "together,'' from a performance perspective. Can the storage
%system move everything to a lower layer when all of it doesn't fit, or is there
%other, less tightly related data that can be evicted to preserve the collective
%retrieval performance?  We need to investigate the potential problems when this
%occurs and how the users can specify additional information to ensure they get
%the best possible data from the SSIO layer.

%We seek to provide dynamic runtime information to the user and make the
%system more transparent. This would allow the end user to analyze or
%visualize data in a more
%predictable way since they would have more confidence of how much time
%the data retrieval process would take.

%Firstly, we will explore the augmentation of I/O application programing
%interfaces (I/O APIs) to allow applications to both specify timing and quality
%information and also query the storage system for timing estimates. If user
%decide to write or read data to or from hierarchical storage systems, they
%will rely on the API to send their intentions for inquiry and examine the
%system status, including how much data they would like to write/read, desired
%bandwidth, data compression method, etc, or the user can express their
%intension of writing data right now no matter what the traffic is now.
%Through this interface we expect the application and user to gain insights
%into how long a single output or input call will take given the required
%quality information, and then react to these estimates by adjusting the
%quality or restricting the scope of the data required. Likewise we will
%explore how an application can provide timing information to the storage
%system to allow the storage system to make optimization decisions to best meet
%the requirements from the application. 
%
%Secondly, we will also explore external data annotations, such as those
%provided by the configuration file in ADIOS. Through the use of these external
%augmentation the user can provide insights to the storage system on the
%relative value of the data, expected life time and performance
%characteristics, as well as relationships between different data sets. With
%this information the storage system can make optimizations specific to a use
%case. We expect these augmentations to be particularly important for eviction
%of data from a storage layer, and migration of data sets to a different
%storage layer. 

\paragraph{Challenges:}
These new techniques offer both technical and adoption challenges. We will only
consider the technical challenges. The proposed techniques and features aim to
expose broad system level environmental information to the application and
allow the application to adapt dynamically based on this information.  One
aspect of this challenge is the design of an interrogative API that offers a
negotiation between the application and the storage system to make adaptation
decisions.  Since we will work closely with many of the leading LCF
applications, we need to make sure that our additional APIs and semantics in
the storage layer will be accepted by these applications and eventually by the
rest of the community. This requires careful examination of the full range
of possible
design choices to ensure stable APIs and useful semantics.



%%% Local Variables:
%%% mode: latex
%%% TeX-master: "../proposal"
%%% End:
 % includes data-desc, apis, and qos-app.   

% \subsection{Application Interface}

%\subsubsection{Description of Data}
In order for the middleware and storage layer to understand data,
I/O intentions, and act accordingly, describing data in a way that can
be understood by the system is needed. In this project, this is achieved 
by managing data as \textit{objects}.
An object is the smallest unit of data that consists of raw data and metadata,
and a user space variable, such as \textit{temperature}, may consist of a collection
of objects possibly with different accuracy or resolution.
A key advantage of representing data as \textit{objects}, rather than 
bytes, is that it allows data semantics, user intentions, QoS requirements, and
the relationship between user data to be readily captured and embedded with the 
raw data. As an example, retrieving a 3D field data generated from a simulation and then visualizing it
requires the coordinates and connectivity variable 1) to be accessed together and 2) with low
latency in order to achieve a good user experience. The description of data, 
as enabled by managing data as objects, bridges 
the semantics gap between applictions, middleware, and storage systems,
and allows the system to understand user-level data, execute QoS requirements and policies, and
optimize application and system performance.

The description of data is essentially metadata and it may include conventional attributes, such as 
data type, size, dimensionality, relationship to other data, and etc. In this project, a key metadata attribute
that enables QoS scheduling, data placement, and ???, is \textit{data utility}, which
captures the priority of objects - it is to our belief that
users must prioritize a small set of objects to be saved on 
higher-level capacity-limited storage layer in exascale
to avoid the slow access to large-capacity lower-level storage, such as tapes.
We propose a new technique, described by a utility function, where objects
can be cast into multiple buckets of data, and each bucket can be prioritized differently according
to its utility value.
Once the data is re-organized, users will be able to use API (section ??)
to describe the importance of this data.

\paragraph{State of the art:} ADIOS implements a binary-packed data format that allows
data characteristics such as min, max and index to be wrapped around data chunks. A
directy benefit is that each data chunk can be operated independently and I/O concurrenty
can be maximized. We will build upon this capability and further augment the format
to incorporate data utility. 

\paragraph{Proposed research approach:} 
We will design and deveop new APIs through which users can describe the utility of data
based on their expectations.


{\bf {\color{red}Scott will integrate these pieces}}
\subsubsection{Data Reduction Techniques}

Mark

 % includes data-reorg and the api specs move to the above section -
\subsubsection{Data Reorganization Techniques}
One of the key element to get efficient data access during reading is the re-organization of the
data to ensure that the data that will be read back will involve minimal seeks on the file system 
along with aggregating similar data together to get optimal performance and placing the data
across multiple disks to get high levels of concurrency during reading. We will create APIs to
allow users to group similar data together in a data-model, and annotations (initially done in the
ADIOS layer) to allow relationships of variables to be expressed. We will then investigate both
system (general) data-re-organization techniques based on the utility of data and the relationships
of the data, and application-specific routines (for our initial set of evaluation test cases).  Once
we can generalize the application specific routines to a set of motifs, we can then allow this piece
of code to propagate to the general routines. In all cases we will investigate re-organizing the data
into an integer (most likely 8) bins, where they fall into the most likely accessed data to the least
likely accessed data. These methods will NOT be used for checkpoint restart data, since this
data needs to be accessed exactly how it is originally in memory.  We generally see that these
techniques are highly useful for analysis and  visualization data, where users can inherently place
their intentions of the data (life time of the data for eviction policies, prioritization of data 
importance). 


\paragraph{State of the art:}

\paragraph{Proposed research approach:}
General methods to re-organize the data fall into the following categories
\begin{itemize}
\item Precision based re-organization. This is where the most significant bytes of the data are
all group together from each object. This data generally will have a higher utility then the data with
the least significant bytes.  This process has the following three steps which need to be
efficiently implemented when data gets placed to the storage system and from the storage 
system. 
%
First, the data needs to be re-arranged which involves memory operations and needs
to be done in-situ. This operation involves no communication.  Data needs to be copied, so
the memory requirements are increased. At most this requires two copies an individual dataset
in memory, and we will investigate techniques to allocate and deallocate this memory if the user
will specify that the data will be overwritten after it is written to the storage system. This is often
the case for many of the data quantities written form the simulation, but there are many cases
where, for example, we want to write all of the particles from a Particle In Cell (PIC) simulation. 
Since the particles will be used later in the calculation, we need to duplicate the storage. Our
observation with working with the XGC1, GTC, Warp, and PiconGPU simulations (all leadership
class simulations) is that we can temporary increase the storage of the particles, and then 
release them since the temporary arrays used in the calculations are often freed when a PIC
iteration is finished.. 
%
The second step to give the metadata that allows the system to understand that an object is now contained in multiple bins of data. One of the challenges is the ability to bring together multiple bins
of data together (the high priority data along with the medium priority data) efficiently when the 
data is being read.  One research question is how we can manage the metadata efficiently.
We propose to place this knowledge into the middleware which will communicate this to the 
storage system.  This will give the storage system potential ways to increase the concurrency 
when accessing this information. For example, we can place the highest priority bits of an object
on the fastest storage system, and the next level of importance in the campaign storage. If a 
user request precision to a level where we need to access both, concurrently we can be
accessing this data, and moving the data to memory where it can be re-arranged. 
%
Since data will be placed in different bins along the hierarchy we will need to understand how
to initially place this in the fastest storage, and then evict this from the fastest storage without
effecting the performance of the simulation which is generating this data. Policies must be 
placed to ensure that we do not create any internal interference during these operations. We
will investigate doing this with ``meta-bots'' first introduced in the LWFS project and have 
these bots move the data.{\bf probably belongs in a different section} 

\item Frequency-based re-organization. FFT, Wavelets.
Another common technique to classify the importance of data is to first re-organize the data
in frequency space and then specify that as the frequency increases the importance of data
decreases.   Common technique which is used for streaming data, and data reduction techniques such as JPEG-2000~\cite{jpeg2000}
is to  code streams which have regions of interest that offer several mechanisms to support spatial random access or region of interest access at varying degrees of granularity. It is possible to store different parts of the same data using different quality.

This allows us to place the lowest frequency pieces in the fastest storage, and
the highest frequency either on the slowest storage tiers, or if the data sizes are prohibitively  
costly, not even write out the highest frequency. This binning of data in frequency space can
take as an input (storage bandwidth knowledge, storage size limitations, and user intentions
of how long the majority of the data will be frequently accessed. 

These operations also require data transformations (for example an FFT and an inverse FFT)
and then re-arranging the data in real-space once we re-organize this. The research questions
we must address are: 1) what is the cost of re-organzing the data, 2) how do we 

\item Histogram binning. Re-organize the data through histograms to ensure we have enough
phase-space coverage. Where we have finner coverage we can move that data to other
quantiles of the storage. 

\item Multi-resolution.  
A


\item Region of Interest. Using error techniques similar to AMR, we can understand where there
is data changing more rapidly and save this data at a higher resolution version than the smoother
regions of space.
{\bf Mark: can you fill these out much more and talk about the new techniques we will work on}.
\end{itemize}
\paragraph{Preliminary work and results:}
Text here
%%% Local Variables:
%%% mode: latex
%%% TeX-master: "../proposal"
%%% End:



%\subsubsection{QOS in the Application}
%proposed idea
A key insight in this proposed project has been the increased interaction of
the application with the storage system. Towards this end, we propose to
address the quality of service requirements by porivding the application
with mechanisms to specify the quality of I/O service, interrogate the
storage system, and react to the responses. Guided by our past work in
ADIOS\cite{lofstead2008flexible} we propose to explore the design of two
mechanisms for this purpose. Firstly, we will explore the augmentation of
I/O applictation programing interfaces (I/O APIs) to allow applications to
both specify timing and quality information and also query the storage
system for timing estimates. Through this interface we expect the
application and user to gain insights into how long a single output or input
call will take given the required quality information, and then react to
these estimates by adjusting the quality or restricting the scope of the
data required. Likewise we will explore how an application can provide
timing information to the storage system to allow the storage system to make
optimization decisions to best meet the requirements from the application. 
%
Secondly, we will also explore external data annotations, such as those
provided by the configuration file in ADIOS. Through the use of these
external augmentation the user can provide insights to the storage system on
the relative value of the data, expected life time and performance
characteristics, as well as relationships between different data sets. With
this information the storage system can make optimizations specific to a use
case. We expect these augmentations to be particularly important for
eviction of data from a storage layer, and migration of data sets to a
different storage layer. 

%execution




%%% Local Variables:
%%% mode: latex
%%% TeX-master: "../proposal"
%%% End:

{\bf {\color{red}Hasan will integrate these pieces, Manish after Hasan}}
\subsection{Data Placement and Movement}
\newcommand{\Sir}{Sirocco}

%key points based on discussion with Carlos
%1. Allocation tables and simple search won't be able to scale
%2. Carlos suggests looking at a determinstic function to decide on
%placement of data into the storage hierarchy. Instead of storing the
%metadata that describes where data is stored (which server/tier) we can
%simple associate a function for each application. 
%3. Research into what would be an approrpriate function that allows even
%splitting of data across many servers/tiers
%4. Similar to our approach for defining utility over time
%5. Migration can also use a function. We shouldn't distinguish migration
%from data movement and placement too much
%6. We need metrics that can decide how the data moves up and down in the
%hierarchy (so we can enable pre-fetching)
%7. Carlos suggested that instead of hints we call them service level
%objectives (and that leads into quality of service)
The application workflows targeted in this proposal generate very large
amounts of data, which needs to be processed and analyzed before
potential insights from the simulations can be realized. 
In-situ and in-transit data staging approaches can alleviate some of the
burden on storage, leveraging available compute and memory resources on high
end systems to process the data close to where it is generated. 
However, given current architectural trends, emerging systems will have a
vastly great gulf between the computing capability (the capability to
generate data) and the storage capability (the capability to store the
generated data). Moreoever, the complexity and heterogeniety of the
architecture will introduce new tiers in the storage system. Instead of
being a flat, roughly homogenous space in which data can be stored, the new
storage hierarchy will be multi-tiered and greatly heterogeneous. 

We proposed to advance the state of the art in the three major parts of the
data lifecycle. First, we will explore the optimizations and techniques to
scalably manage data when it is generated by the application. Second, we
will study the management of resources and data at rest. And finally, we will
investiage how data is consumed, either by being read and processed (for
example within analytics) or by being purged from the system. 

Towards this end, we will utilize the faster layer of the
storage hierarhcies, such as DRAM, NVRAM and remote NVRAM, as the target for
the initial write from an application. To mitigate the space concerns, we
will use the techniques described in Section~\ref{sec:refactoring} to
identify and partition the segments of data with the highest utility for the
application and use case. The rest of the data will be stored on the
parallel file system, as is done today. We will leverage the supported data
mirgration techniques in \Sir to determine the most appropriate location to
move the data once its in stable storage. This migration will be guided by a
combination of the utility function defined by the application and the
resources avialable to the storage system. Knowing the specific structural
and semantic information (including relationships between data segments)
about the data will enable migration to partition the data further to
optimize access to data for reader. 
We will also explore policies that combine the knowledge of data, its
utility to users, the cost of maintaining data in long term storage and
other metrics to decide how to purge irrelevant data from the storage
system. 

\subsubsection{Data Placement}
\label{sec:data-placement}

Our past work on data
staging\cite{tongipdps15,qiansc15}\cite{docan2012dataspaces}\cite{abbasi2010datastager}
in HPC systems, coupled with the experience in developing high performance
I/O transports for ADIOS \cite{liu2014hello,lofstead2008flexible} has
provided us with insights on how to manage the complex I/O requirements for
exascale applications, particularly in the face of the aforementioned
multi-tier storage hierarchy. A key requirements is application-driven
%
runtime mechanisms for dynamically managing data placement across the layers
of the distributed storage hierarchy,
%
coordinating data movement 
%
and data sharing between the components of the application workflow so as to
maximize its utility to the application and reduce access costs. Our
approach to data placement will leverage the increased set of knowledge
about the data available to the middleware and storage system, potentially
allowing for greater optimizations as compared to other storage solutions. 
The key components of our research include

\paragraph{Application Hints:}
As described in Section~\ref{sec:refactor} one of the main components of our
proposed storage system is the ability to refactor and reduce data as it is
generated, and to reorgnized and regenerate the data as it is accessed. We
carry this principle into the placement and movement of data by allowing
applications to define hints and policies that guide \textbf{what} data is
placed \textbf{where}. We will explore the use of application hints in two
distinct areas. Firstly, we will study the challenges and tradeoffs of either
augmenting the I/O interface with hints or allowing the additional of an
external specification that defines the use case. Our experience with
developing modern I/O interfaces has shown that both techniques have value,
and we will investigate the set of hints that are embedded in the
application code vs. those that are described within an non-compiled
specification. Secondly, we will study how hints can guide the initial data
placement as data is handed off from application to storage ( during a
write) and from storage to application (during the read). In both cases we
will study what minimal set of annotations and hints can allow the storage
system to minimize data movement and optimize the resources consumed by I/O. 

\paragraph{Data Utility:} We will formulate the concept of utility (from an
application perspective) that is assigned data objects to quantify the
benefits of placement and/or movement actions to application. These
utilities may be based on the importance of the data objects, the frequency
with which they are accessed, etc., and are used during runtime data
management. For example, data objects will higher utility value may be
placed closer (e.g., in faster memory) to the accessing application.
Similarly, when it is time to evict data objects from a higher level storage
layer, the decision about which object to evict may take the utility into
account.

\paragraph{Autonomic in-transit data management:} 
Efficiently placing data objects
vertically across multiple memory hierarchy layers and storage levels as
well as horizontally across different storage nodes requires balancing
competing objectives. In this project we will explore autonomic data
management strategies that can evaluate utility/cost tradeoffs and
appropriately place/move data objects at runtime. The autonomic approach
will leverage both user hints and information gathered at runtime (e.g.,
runtime data access history, network topology, etc.) to place data object.

\paragraph{ Runtime tracking and estimation:}
Autonomic data placement and/or
movement will leverage the information gathered at runtime together with
predictive analysis requirements. This information may include runtime
spatial-temporal data access patterns, physical network topology
information, and etc., which will be tracked and gathered at runtime, and
then be used to estimate and predict the behavior of coupled applications in
the rest of simulation-time. A key research aspect here will be to
anticipate accesses, for example using learning techniques or using
application knowledge, and use this information to prefetch data objects.
For example, we will identify and catalogue key phases and access patterns
in the target workflows and then use it to characterize patterns at runtime
and trigger appropriate place/movement actions.

\paragraph{State of the art:} NAND flash-based devices are being
increasingly used in HPC storage systems at different levels and for a
variety of purposes. Several studies such as \cite{multitier}, \cite{sc10li}
have investigated using compute node local SSDs as storage buffers, for
example, to temporarily cache checkpoint data to support recovery from
failures. Research efforts have also explored using deep memory devices for
data analysis. Active Flash \cite{activeflash} proposes in-situ scientific
data analysis by directly executing data analysis tasks on emerging storage
devices. Minerva \cite{minerva} extends the conventional SSD architecture
using a FPGA-based storage controller to offload data or I/O intensive
application code to the SSD to accelerate data analysis. However, these
solutions cannot deal with coupled application workflows, where the data
exchange and access patterns can be both complex and dynamic. Furthermore,
some of these solutions require hardware modifications or special access
priorities to the HPC systems, which can limit their use to specific
resources.

\subsubsection{Migration}
\label{sec:migration}

One important research question in this proposal is whether and how a liberated
Hierarchical Storage Manager designed for online use, such as Sirocco, might be
employed in supporting the higher level concepts discussed previously.
Conceptually, in this effort, data is accessed by the middleware IO libraries
at various tiers in the lower-level storage hierarchy. While Sirocco does
autonomously group the distributed set of media into like groups and manage
movement of data between these groups the various motivations to do so include
both the traditional ones found in classical HSM (Hierarchical Storage
Management) as well as the novel.

Sirocco groups like media into aggregated volumes on a service node, and like
service nodes with like media into pseudo-tiers. It accomplishes both using
attributes such as latency to first byte, bandwidth to and from the media, and
some function representing the resilience capabilities of the attached media.
In addition to these static attributes, when grouping servers into
pseudo-tiers, Sirocco also may employ more dynamic attributes such as CPU load
and the rate-of-change of the media use.

Grouping into pseudo-tiers is not a formal, nor rigid, concept within Sirocco,
at present. In classic HSM the tiering is often a formally expressed concept,
embodied both in the architecture and the implementation. In Sirocco, though,
it is not directly expressed. Instead, a client or server that attempts to make
a decision relative to migration and staging uses the attributes discussed
above to constrain the choice of candidates. For instance, a server acting as
an off-node writeback cache for a compute node might only have access to
volatile dynamic RAM for use as storage media and a client might deposit data
that is marked for persistent media as a resilience constraint. At some point,
then, such a server must copy or move the data to another Sirocco server with
media matching, or exceeding, the resilience constraint. At that point in time,
all servers managing appropriate persistent media are candidates but
second-order attributes, such as latency and bandwidth of a candidate's managed
media would tend to dominate the ranking. Thus, the concept of a ``tier''
becomes an artifact of decision rather than a formal concept.

Further complication arises when a Sirocco server is forced to make choices for
capacity reasons. The Sirocco architecture dictates that a server function as a
victim cache. As media managed by a server fills the server will begin to
manage for capacity. In order to do so, currently, it need only verify or make
a copy of the data on another server that meets the resiliency constraint of
the held data prior to removing the local copy and in the current
implementation the server makes no attempt either to remember it ever had the
data or to notify others about what it is doing.  others that it has removed
the local copy or remember that it has 

Taken together then, the above at least complicates the higher level research
questions discussed in this proposal, potentially even renders some
unachievable since a few involve determinism. We are forced, then, to consider
either or both mitigating management policies within and between Sirocco
servers as well as explicitly allowing clients to participate in capacity
management decisions, for a time at least, on the relevant Sirocco servers.

Our research intent involves short term and long term activities. In the short
term we will modify the relevant clients to provide hit and miss related data
and associated activities and explore an enhancement to our storage servers
that would provide some, hopefully tightly, bounded estimate about time to
first byte for a potential read operation. In the long term we hope that an
augmented client-server API providing usage hints will allow us to modify local
server policy algorithms to more effectively manage it's media and communicate
to other servers holding copies of the relevant data how they might best manage
theirs. In the event that this approach proves ineffective at meeting the
overall approach described in this proposal we will explore modifying Sirocco
servers to provide notifications about locations of the relevant data as they
migrate and stage as well as supporting client-directed pinning of location,
for a period at least. Client-directed pinning could introduce potentially
serious negative impacts to Sirocco's overarching goals though, since the
fundamental design of Sirocco assumes much about opaque and unrestricted data
movement, so this is contemplated only as a last resort.

\subsubsection{Purging}
\label{sec:purging}


% The title might have to change...

\paragraph{Background:} There is a spectrum of approaches on how
to leverage a memory and storage hierarchy: On one end of the
spectrum is to copy or move data from one tier to another, the
closer the tier to the hierarchy's top, the faster the access to
the data it contains. On the other end of the spectrum is to use
higher tiers exclusively for auxiliary data that to some extent
represents the actual data on the lowest tier. Examples of auxiliary
data are views, indices, lossy compressions, lower resolution data,
and summary data (sometimes also known as metadata). There are a
number of drivers that push approaches to one or the other extreme:
(1) if a high tier has sufficient room for an application's working
set, approaches that move the working set into that tier will perform
well; Conversely, if a high tier very rarely captures working sets,
it is better to fill it with auxiliary data that speeds up access
to the working set captured by a lower tier; working sets only
reduce capacity misses but the concept can be generalized to the
likelihood of having the right data in place, i.e. proactively
moving data to high tiers accessed in predictable patterns reduces
compulsory misses as well; (2) if the application generates parallel
read/write requests and strong consistency semantics are important,
approaches that use auxiliary data to speed up access to persistent
shared data might yield better performance than approaches that
require coherence overhead to keep all copies on higher tiers
consistent; (3) if the application's access patterns are mismatched
with low tier access characteristics, a combination of approaches
that use auxiliary data and move data between tiers in order to
convert access patterns into a better match will perform well.

% Any other drivers?

\paragraph{Approach:} This spectrum enables interesting strategies
to manage space pressure on high tiers: instead of just evicting
pieces of data (blocks, pages, objects, files), one can free space
by ``thinning'' data so that accesses within more limited views or
lower resolutions can still occur without misses while requests
outside of views or higher resolutions can leverage the information
stored in auxiliary data. Conversely, data on a high tier can be
``enriched'' (e.g. turned into a higher resolution) if there is
value in doing so.

\paragraph{Related Work:} Mention Stanford's Legion system.

\paragraph{Challenge:} The decision on when to thin and when to
enrich data on a particular level depends on \emph{access patterns}.
Approaches that leveragee access patterns in application at national
labs has shown to greatly improve performance and reduce the amount
of overhead required for data management~\cite{he:hpdc13}. The
challenge is for applications to communicate these access patterns
to the I/O stack layers that manages tiers outside of the address
space. In the proposed work we will develop an abstraction for
\emph{locality} that allow applications or runtime profilers to
describe locality properties of access patterns, and implement
services that leverage locality properties to dynamically determine
the value of thinning or enriching data at a particular tier.
Consider the example of a road navigation system: the locality
properties of each navigation client are based on physical constraints
of speed, direction, likely resolution of map, and the fact that
clients almost never go offroad.

\paragraph{Preliminary work and results:} Our recent work \cite{tongipdps15}
explored a two-tiered staging method that spans both DRAM and solid state
disks (SSD). It allows us to support both code coupling and data management
for data intensive simulation workflows in a loosely coupled manner. In
addition, our application-aware adaptive data placement has demonstrated the
effectiveness of using user provided access information as ``hints'', and
runtime data access pattern history to improve data access efficiency and
overall end-to-end execution. Besides, additional data placement
optimization \cite{qiansc15} that leverages physical network topology along
with data access patterns has also demonstrated lower data access costs and
good scalability. Our work have been deployed and proven with real
applications, e.g., coupled combustion simulations (DNS-LES) that are part
of the ExaCT Exascale Co-design Center and coupled Fusion simulations
(XGC0-XGCa) that are part of the EPSI SciDAC on current high-end systems
such Titan at ORNL.


%%% Local Variables:
%%% mode: latex
%%% TeX-master: "../proposal"
%%% End:

\subsubsection{Migration}

One important research question in this proposal is whether and how a liberated
Hierarchical Storage Manager designed for online use, such as Sirocco, might be
employed in supporting the higher level concepts discussed previously.
Conceptually, in this effort, data is accessed by the middleware IO libraries
at various tiers in the lower-level storage hierarchy. While Sirocco does
autonomously group the distributed set of media into like groups and manage
movement of data between these groups the various motivations to do so include
both the traditional ones found in classical HSM (Hierarchical Storage
Management) as well as the novel.

Sirocco groups like media into aggregated volumes on a service node, and like
service nodes with like media into pseudo-tiers. It accomplishes both using
attributes such as latency to first byte, bandwidth to and from the media, and
some function representing the resilience capabilities of the attached media.
In addition to these static attributes, when grouping servers into
pseudo-tiers, Sirocco also may employ more dynamic attributes such as CPU load
and the rate-of-change of the media use.

Grouping into pseudo-tiers is not a formal, nor rigid, concept within Sirocco,
at present. In classic HSM the tiering is often a formally expressed concept,
embodied both in the architecture and the implementation. In Sirocco, though,
it is not directly expressed. Instead, a client or server that attempts to make
a decision relative to migration and staging uses the attributes discussed
above to constrain the choice of candidates. For instance, a server acting as
an off-node writeback cache for a compute node might only have access to
volatile dynamic RAM for use as storage media and a client might deposit data
that is marked for persistent media as a resilience constraint. At some point,
then, such a server must copy or move the data to another Sirocco server with
media matching, or exceeding, the resilience constraint. At that point in time,
all servers managing appropriate persistent media are candidates but
second-order attributes, such as latency and bandwidth of a candidate's managed
media would tend to dominate the ranking. Thus, the concept of a ``tier''
becomes an artifact of decision rather than a formal concept.

Further complication arises when a Sirocco server is forced to make choices for
capacity reasons. The Sirocco architecture dictates that a server function as a
victim cache. As media managed by a server fills the server will begin to
manage for capacity. In order to do so, currently, it need only verify or make
a copy of the data on another server that meets the resiliency constraint of
the held data prior to removing the local copy and in the current
implementation the server makes no attempt either to remember it ever had the
data or to notify others about what it is doing.  others that it has removed
the local copy or remember that it has 

Taken together then, the above at least complicates the higher level research
questions discussed in this proposal, potentially even renders some
unachievable since a few involve determinism. We are forced, then, to consider
either or both mitigating management policies within and between Sirocco
servers as well as explicitly allowing clients to participate in capacity
management decisions, for a time at least, on the relevant Sirocco servers.

Our research intent involves short term and long term activities. In the short
term we will modify the relevant clients to provide hit and miss related data
and associated activities and explore an enhancement to our storage servers
that would provide some, hopefully tightly, bounded estimate about time to
first byte for a potential read operation. In the long term we hope that an
augmented client-server API providing usage hints will allow us to modify local
server policy algorithms to more effectively manage it's media and communicate
to other servers holding copies of the relevant data how they might best manage
theirs. In the event that this approach proves ineffective at meeting the
overall approach described in this proposal we will explore modifying Sirocco
servers to provide notifications about locations of the relevant data as they
migrate and stage as well as supporting client-directed pinning of location,
for a period at least. Client-directed pinning could introduce potentially
serious negative impacts to Sirocco's overarching goals though, since the
fundamental design of Sirocco assumes much about opaque and unrestricted data
movement, so this is contemplated only as a last resort.


\subsubsection{Eviction Metrics}

% The title might have to change...

There is a spectrum of approaches on how to leverage a memory and storage hierarchy: On one end of the spectrum is to copy or move data from one tier to another, the closer the tier to the hierarchy's top, the faster the access to the data it contains. On the other end of the spectrum is to use higher tiers exclusively for auxiliary data that to some extent represents the actual data on the lowest tier. Examples of auxiliary data are views, indices, lossy compressions, lower resolution data, and summary data (sometimes also known as metadata). There are a number of drivers that push 

%\subsubsection{Policies}

One of the key differentiating features this project offers is driving storage
decisions through policies rather than continuing the current model of allowing
all application to compete, without restriction, for part or all of the storage
resources. This current approach leads to intereference
effects~\cite{lofstead:2010:adaptive} that can greatly impact IO performance
predictability. Further, with the introduction of additional tiers in the
storage stack, the performance variability will increase with greater
competiion for limited high performance resources. Unfettered, this competition
will yield lower performnce for all users than if users either restrict IO
resource uses to a disjoint subset each or even using lower raw performance
resources.

Our policies are driven from multiple different approaches. First, supporting
capability runs is a priority for capability machines. We propose to address
capability run performance by incorporating monitoring for prepartory runs at
smaller scale to profile the application output characteristics both from a
writing velocity and volume, but also for example read patterns for the data
analytics required to generate scientific insights from the raw data. By
discovering approximate data proportions that will likely be the key subsets
targeted by the simulation run, we can generate a policy to preserve a certain
storage portion for high fidelity data storagae and use slower or perhaps lower
fidelity or compressed data storage for less interesting data portions. We will
support spreading data appropriately across the storage hierarchy. In
particular, we will investigate how to place data across different layers to
meet the performance requirements for output while maintaining system
availability for other applications and offering the best possible performance
for the data analytics that will ultimately process this data.

Second, once data has been placed for future performance requirements, policies
must be able to address maintaining data on a priority basis in that location
to deliver on the placement optimization generated by the capability run
profilier.

Third, most future NVM devices have limited write endurance prompting
management to ensure fair use by all machine users. Technologies like
NAND-flash and Phase Change Memory have limited write endurance. By
incorporating policies about the proportion of write endurance a compute
allocation is entitled to, data placement decisions can be made to ensure fair
resource usage. While the offending application will suffer worse IO
performance, other applications that more carefully address their data
intensity on these limited devices will achieve higher overall performnce. Only
by instituting such policies can we encourage application users to adopt
technology to manage the limited resources.

We propose to investigate both offering a policy mechanism and how to implement
these kinds of policies in a way that addresses offering the shortest
end-to-end time for scientific insights.

%\subsubsection{APIs}

The new APIs will provide new functionality of specifying selectable performance/quality/- cost tradeoffs from both the application and system perspectives based upon the user guided rules/policy and runtime system monitoring status. It allows the middleware to make best possible decisions from the feedback of storage system knowledge, such that it will embed user intentions and the available system storage. The user would like the ability to get a certain amount of Quality of Service such that they can then make decisions when the expected bandwidth, for example is less than what they desire and will then place in a certain set of rules which the system can make autonomic decisions to help decide what should be done. We want the system to give the users a certain amount of currency in terms of bandwidth, storage space on each level, and latency expectations. These notions will be fuzzy but they will allow the user to make ad-hoc decisions to figure out what needs to be saved.

When user ask the system to write their data, and they would like to get an estimated time which they can then figure out, through a series of rules, whether they want to write out all of the data or wait for a later time to write or write a reduced amount of data. The decision is based upon if the system monitoring model could give us the estimate time during the reading and writing process, so that rules can be in place from the user perspective to make adaptive decisions. 

If user decide to write or read data to or from hierarchical storage systems, they will rely on the API to send their intentions for inquiry and examine the system status, including how much data they would like to write/read, desired bandwidth, data compression method, etc, or the user can express their intension of writing data right now no matter what the traffic is now. After receiving the negotiation signal, the API return value will contain the estimate time of write/read data, potential bandwidth, concurrency, etc. User will adapt their decision from the system feedback and determine how much data they will write/read if the estimated time is too long for them. 

The capability of APIs provide dynamic runtime information to user and make the system more transparent to them rather than a black box. The end user could analyze or visualize data in a more comfortable position, where they are able to envision how much longer they are going to wait for data. 


% \subsection{Storage System}

{\bf {\color{red}Gary will integrate these pieces}}
\subsection{Resource Management}

Sirocco currently offers automatic resource management capabilities. However,
these features are strictly driven on resource capacity and data resilience
requirements. For this project, we seek to expand the resource management
capabilities to support additional metrics to guide resource management
decisions. Our basic goal is to offer data-centric metrics that help guide the
forced resource management decisions.

As stated in the introduction, we seek to make a storage stack that works with
the user. Towards that end, we must incorporate data attributes into this
process. For example, particularly for a capability run, the storage system
should give priority to data subset tagged with high importance to stay in fast
storage tiers. Enabling this priority will be driven by other job
characteristics such as job size or a manually set priority inserted by system
administrators. It should also store the highest quality data possible given
the performance requirement. Ideally, the storage system will store the raw
data at full fidelity. When this tier approaches capacity, data should migrate
not just based on age, but also based on this priority. This will help ensure
the shortest time to insight by making sure the most important machine runs
have priority to use system resources.

Existing efforts on scheduing resources~\cite{io-cop,dorier} attempt to offer
admission control to maximize storage performance. Unfortunately, these efforts 
are limited to participating applications and only applications from a single
platform. For this approach to be effective, the storage system must offer an
approach that handles clients from all connected platforms and does not
require modification to use new storage access APIs. By using a priority
tagging system, we will offer defaults suitable for all applications that can
be informed by simple extensions to the job scheduler or more advanced
management through new APIs.

What else do we need to talk about?


\subsubsection{QoS}

\paragraph{Background:} ``Quality of Service'' (QoS) refers to the
properties of the performance qualtiy of a particular service.
Performance quality can be expressed in either relative or absolute
terms. Examples of relative performance qualities are ``fair'',
``proportional'', or ``priority/class-based'' (``the higher the
priority the better the service'' or ``1st class is better than
2nd''). The key advantage of relative performance qualities is that
they are easy to implement. The key disadvantage is the difficulty
of making any strong guarantees based on those implementations --
even relative guarantees are mired by the well-known effect of
priority inversion~\cite{lampson:cacm80}. Examples of absolute
performance qualities are ``rate-based'', ``soft real-time'', and
``hard real-time''. The key advantage of absolute performance
qualities is that it is principally easier to implement strong
guarantees: they enable an agreement between a task and a service,
also known as a \emph{reservation}, does not change meaning depending
on the service's state, as long as the service is able to isolate
performance between different tasks. The key disadvantage of absolute
performance qualities is that their implementation is non-trivial,
especially in large-scale systems.

\paragraph{Approach:} The goals of this proposal require applications
to demand guarantees in terms of absolute performance qualities.
The first key enabler for any service to make absolute guarantees
is \emph{performance isolation}, i.e. the ability to shield the
performance impact of one task from another. The most straight-forward
but least flexible strategy is to simply cap the latency of each
task, irregardless of its completeness (see for
example~\cite{decandia:sosp07}). This strategy is applicable when
incomplete tasks still have value and missing results are either
not important or can be easily retrieved in subsequent tasks. A
more generally applicable strategy is to control the admission of
tasks to the system: \emph{admission control} maintains a
\emph{utilization model} that can predict the utilization of system
resources given a new task, estimates the current utilization of
the system, and decides whether the task can be admitted without
overloading the system. Thus, admission control avoids system
overload conditions that lead to hard-to-model chaotic behaviors.
Consequently, utilization models can frequently be approximated by
automatically calibrated linear models (see for
example~\cite{skourtis:hpdc12}).

Another essential component of performance isolation is the
\emph{charging model} which determines which task(s) should be
charged how much for a request. Performance isolation is implemented
by the charging model by ensuring that performance of a task only
depends on its reservation and its workload behavior (but not on
other workloads in the system). Each task has a budget based on its
reservation and spends it based on the cost of each request determined
by the charging model. A charging model determines the cost based
on the utilization model and the interactions a request can have
with other requests (e.g. a random request to spinning media
interrupting a stream of sequential requests).

Performance isolation (including its utilization and charging models)
critically depends on \emph{workload-independent metrics}. Even
though throughput and latency are frequently the most meaningful
metrics for an application, they are not workload-independent and
therefore are inadequate for measuring the utilization of a device
across all workloads, unless one accepts worst-case estimates that
can lead to gross underutilization of resources (e.g. the throughput
of random and sequential workloads in spinning media differ by
orders of magnitudes). A common workload-independent metric is
\emph{time utilization}, i.e. the amount of time a resource is
utilized in a given time interval. Time utilization has the advantage
of having an always defined maximum and therefore makes resources
fully reservable. Even though a workload-independent metric might
at first not appear meaningful for an application, given a particular
workload an application can discover the relationship between its
workload-dependent metric and the workload-independent metric.
Because of performance isolation, the application has to discover
this relationship only once.

\paragraph{Related Work:}

\paragraph{Challenges:} The deep and heterogenous memory and storage
hierarchy we are assuming for this proposal complicates the
relationship between workload-dependent and workload-independent
metrics: the performance of a task can significantly differ depending
on what levels of the hierarchy are involved. Furthermore, an
important goal of the proposed project is to enable applications
to reason about the trade-off between resolution and latency, adding
yet another dimension to how tasks are mapped to resources.

\begin{itemized}

\item A key challenge of admission control is scalability: the
decision of whether to admit a task could potentially depend on
global knowledge of the current utilization of every single resource
in the system. Scalability therefore depends on whether admission
control is able to accurately map a new task to a relatively small
set of resources which can quickly provide up-to-date utilization.
One approach might involve pseudo-random mapping that also load-balances
as a side-effect.

\item To offer latency/latency trade-offs, the system must be able
to quickly generate a number of data production plans, involving
different parts of the hierarchy and different data resolutions.
It will then use the utilization model to estimate the latency for
each production plan and the resolutions they can provide. Here
again, pseudo-random selection could reduce the number of resources
that would be involved, thereby increasing scalability.

\item The complexity of a utilization model involving the entire
storage hieararchy with 100,000s of devices is potentially daunting.
However, the utilization model can be simplified by modeling classes
of devices as well as classes of requests. In particular, each
device could restrict access to its content via a set of well-defined
methods with known, absolute performance properties.

\item Flash devices become inherently unpredictable when reads and
writes are mixed on the same device because of garbage collection.
While write latency can be easily hidden using asynchronous I/O,
hiding read latency is more difficult. By separating reads and
writes for each device, read latency becomes predictable. The
challenge is to minimize duplication of data, at least on fast
layers and leverage redundancy across layers.

\end{itemized}

%\subsubsection{Discovery}

Hierarchical Storage Management (HSM) systems offer a strct caching approach
for managing different storage capacities trading off performance for capacity.
By maintaining a single namespace across all tiers, it is possible to list a
single directory view with files stored at different tiers. While this approach
to managing multi-tier storage works, it is far from ideal for scientific
simulations.

Our goal with this proposal is to offer a similar capability, but use a finer
granularity. Instead of a single file such as might be used to store an entire
timestep output for a simulation, we will demonstrate offering the same
capability at a subset of a single variable level. By shifting to a
finer-grained approach, we will enable more effective use of close/fast/small
storage tiers. Traiditional HSM stores an entire file on a tier making room if
insufficient space is available. With this shift to a partial variable
granularity, a 1 PB output with 500 GB of ``high interest'' data can limit this
costly tier ussage to just 500 GB greatly enhancing usability.

Tagging a variable subset as ``high interest'' requires intervention from the
application and/or middleware to determine what data meets this criteria. The
storage system itself simply needs to offer an ability to perform different
actions based on this information.

The challenge for discovery is that potentially, data will migrate from where
it is initially stored to a new location within the storage system. Sirocco
offers an ability to search for data that has moved as well as forcing a
particular resilience-based replica be the ``authoritative'' version. We will
investigate if the current Sirocco functionality is capable of supporting the
new operating modes we wish to offer. Initial expectations suggest having
bounded time guarantees for finding data are critical for offering the quality
of service guarantees we wish to offer. This new work will be an expansion of
Sirocco's currently planned features.

The other aspect of discovery is the negotiation between the user and the
storage system for a data quality/retrevial time tradeoff. The naming service
will work hand-in-hand with the discovery, data migration, and time estimation
services to offer the best possible options for data retrevial based on quality
of service requested.

{\bf {\color{red}Jay  will integrate these pieces}}
\subsubsection{Naming Service}

A traditional POSIX naming service offers a hierarchical space consisting of
directories and files. This structure has led to scalability problems because
of the serialized access to a single source for creating and accessing files.
Several efforts~\cite{giga+,pvfs,others} have worked to reduce this contention
by doing things like reducing the serialized scope to a single directory or
subtree. While these approaches help, they do not address this key scalability
limitation.

Pure object stores, such as those popular in the big data
domain~\cite{memcached,others} avoid this bottleneck by strictly offering an
object ID with the application required to manage how this ID maps to something
of interest. This approach of removing the metadata service from the system
level completely can work well for scale out applications where data is created
or consumed by a single process at a time rather than potentially O(1 million)
processes all actively collectively for a single ``object''. To address this
case, having some system integrated metadata services to associate names with
these object is a preferable solution.

LWFS and Sirocco have take an approach similar to the pure object stores, but
with a focus on the HPC setting. They have abandonded a fully POSIX compliant
metadata service as the default model in favor of a
container/object/fork/addrress tuple for identifying data similar to those
used for pure object stores. By having a service that addresses the object
collections that comprise a single thing, such as a variable or timestep
output, just enough metadata is maintained to make the storage system usable
without additional heavy lifting by clients.  LWFS demonstrated a POSIX-style
namespace on the side kept in sync using a transaction process like
D2T~\cite{d2t} showing that this alternative approach can support traditional
POSIX API calls even though the underlying storage system uses a different
model.

The challenge this proposal brings is in the additional metadata required to
track all of the different pieces of a single variable across the storage
hierarchy both horizontally and vertically and keeping this metadata in sync
as the storage system (Sirocco) manages data according to user requests and
system load. Further, the profiled performance for annotating data according to
future read needs also must be tracked. Overall, since the end-to-end quality
of service requirements must drive all data placement and movement decisions,
the metadata and naming service must be rich enough, and low overhead enough,
to support these operations as well as maintaining data accessibility.

In addition to the basic naming and data tracking operations, we will also need
to incorporate authorization capabilities. Sirocco currently integrates with a
Kerberos service for authentication and authorization. Given a capability
ticket, a user can access different objects as needed. This ticket structure
offers protection services typically offered on POSIX directories and files,
but can do it at the fork level instead. This allows a reduced quality data
version to be available for the general users while the high-quality version
would be limited to the data creator. This and other considerations for
security must be incorporated into the entire naming and metadat service.

The main challenge of incorporating the additional, rich metadata will be
joined by the additional challenge of coordinating with the other storage and
application-layer services to offer the best access times possible for data
stored in the system. The developed metadata services that drive data
compression and subsetting operations must have easy, consistent, and ideally
non-blocking or locking access to this metadata service. We must investigate
how to build such a metadata and naming service that also incorporates and
maintains the additional metadata required to support our advanced
functionality.

% \subsubsection{Estimation of Times}

Feiyi

There are two fundamental causes that can impede the I/O performance in an extreme-scale SSIO system. One is indirection, where layers of storage tiers are employed either for performance and/or for scalability reasons.  The direct consequence is that there could be multiple traversing paths from application end to the rest place of data, and more often than not, the I/O paths are not under control under any single authority. The other cause is the shared use of resources. The best effort I/O request/response nature and lack of QoS mechanisms imply that there is little guarantee in terms of expected performance.  Both indirection and shared use of resource contribute to a high probability of imbalanced use of resources thus the occurrence of congestion and degraded performance. Traditionally, there is a disconnection between storage infrastructure and rest of system software and applications: the infrastructure details are not exposed anywhere at all. One argument favoring this is to ensure a platform agnostic design. 

As an example to demonstrate that this disconnection will hurt system performance: We launch 4096 processes with each process doing a single file I/O operation against half of the Spider II file system. The traces of those files are analyzed to examine the utilization distribution of different components. Figure (a), (b) and (c) shows the resource usage distribution for OSTs, OSSes, and LNETs, respectively. We observe that there exists a significant variation in usage across components of any given type (e.g., OST, OSS or LNET). For example, some OSTs are used more than 10 times while some others are never used (corresponding to zero frequency count). Similarly, OSSes and LNETs show significant imbalance in usage under the default placement strategy. Consequently, imbalanced resource utilization increases the contention at certain components more than others. 

Given these insights, we advocate the idea that the infrastructure knowledge should find a way to relay to the upper layer for better and more effective use of resources.   We think this is more pertinent and critical given the recent development of multi-tier storage and storage component heterogeneity.  There needs to be a way for application/middleware layer to gain more exposure of storage system for more intelligent processing logic.  One prime example of such exposed knowledge can be request/response time. To most applications, this is a black box. Profiling it at the upper layer is neither efficient nor effective, as it doesn't reflect cross-layer characteristics. However, Most of storage layer does keep a detailed profiling of such information. We therefore envision and propose a histogram-based request/response profiling API that application and middleware layer can leverage and make more informative decisions.

\begin{figure}[tbh]
  \centering
  \includegraphics[width=\columnwidth]{graphics/infrastructure.pdf}\vspace{-1.2in}
  \caption{Resource usage distribution for OST(a), OSS(b) and LNETs(c). }
\end{figure}

%%% Local Variables:
%%% mode: latex
%%% TeX-master: "../proposal"
%%% End:



%%% Local Variables:
%%% mode: latex
%%% TeX-master: "../proposal"
%%% End:
