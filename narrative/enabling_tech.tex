\section*{Enabling Technologies}

Significant existing work will serve as the foundation upon which this project
will be performed.

At the lowest level is the Lightweight File Systems (LWFS) project~\cite{lwfs}.
As mentioned above, LWFS's goal is to provide the simplest core services for
a storage system and investigate adding functionality by using auxiliary
services or layering an API on top that offers more functionality. For example,
by eliminating consistency checks from the core system, applications already
managing consistency can eliminate the redundant check enhancing performance.

The current LWFS project phase is investigating new ways for managing storage
devices. This project, Sirocco, offers a peer-to-peer style data storage device
mesh wtih data migrating based on use, required resilience, and available
resources. For example, when writing data initially into Sirocco, required
resilience characteristics are provided prompting Sirocco to ensure the data is
stored on appropriate storage device(s) that meet the resilience requirement.
When a particular device is full, Sirocco determines if it should migrate the
data to a new location or if it is a copy beyond the resilience requirement. If
this copy must still be protected, Sirocco will find a place of sufficient 
resilience to move the data to before deleting the local copy. Further, when
data is requested, a client can ask any Sirocco server and Sirocco will find
the data returning it to the client. If desired, the canonical data location
can be moved on client request to offer better performance for the client. In
additional features, a stored object can have multiple data forks that can
store different versions or aspects of the same data.  For this project, we
will build on this Sirocco foundation for data placement, migration, and the
version storage (via the fork mechanism). 

The Sirocco API is much more complex than a standard POSIX call prompting
layering a more user friendly interface on top. The ADIOS~\cite{adios} API
offers an API nearly as simple as POSIX, but with the ability to change the
actual data transport mechanism without changing the source code. This affords
writing to a single output API while shifting from writing to a POSIX file
system, some middleware like DataSpaces~\cite{dataspaces}, or some other system
like Sirocco all without changing the source code. ADIOS has proven effective
for writing to POSIX file systems, data staging
systems~\cite{predata,dataspaces,nessie}, workflow systems~\cite{flexpath}, and
even nothing through the use of a NULL transport.

At a similar level as ADIOS, but can be used in conjunction, are tools like
DataSpaces~\cite{dataspaces}. They offer a way to request data set portions
from a one component by the next componet in a data flow. In DataSpaces case,
the data transfer does not have to correspond to the data from a single
process.  Instead, DataSpaces will determine where the data is and issue
requests to pull the data from the source to the sink in an on demand process.

For this project, we plan to use the Sirocco data storage mechanism to
offer managing the actual data storage on all of the different tiers. We'll add
a data management layer as part of the file system to handle data discovery and
metadata tasks. As a middleware layer, we will investigate offering differing
data compression, fidelity, and storage placement options. Through this 
combination, we will be able to deliver a complete end-to-end system.

