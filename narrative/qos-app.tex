\subsubsection{QOS in the Application}
%proposed idea
A key insight in this proposed project has been the increased interaction of
the application with the storage system. Towards this end, we propose to
address the quality of service requirements by porivding the application
with mechanisms to specify the quality of I/O service, interrogate the
storage system, and react to the responses. Guided by our past work in
ADIOS\cite{lofstead2008flexible} we propose to explore the design of two
mechanisms for this purpose. Firstly, we will explore the augmentation of
I/O applictation programing interfaces (I/O APIs) to allow applications to
both specify timing and quality information and also query the storage
system for timing estimates. Through this interface we expect the
application and user to gain insights into how long a single output or input
call will take given the required quality information, and then react to
these estimates by adjusting the quality or restricting the scope of the
data required. Likewise we will explore how an application can provide
timing information to the storage system to allow the storage system to make
optimization decisions to best meet the requirements from the application. 
%
Secondly, we will also explore external data annotations, such as those
provided by the configuration file in ADIOS. Through the use of these
external augmentation the user can provide insights to the storage system on
the relative value of the data, expected life time and performance
characteristics, as well as relationships between different data sets. With
this information the storage system can make optimizations specific to a use
case. We expect these augmentations to be particularly important for
eviction of data from a storage layer, and migration of data sets to a
different storage layer. 

%execution
We will design an updated I/O API that will expand on the current POSIX I/O
semantics by adding new information to the I/O request. We will explore the
set of application level hints that can be easily provided to the storage
system in order for the storage system to make the most appropriate
optimization decisions. In particular, we expect the application to provide
hints on the length of time before the output data is persistent and
visible to users, and the expected lifetime of the data for data output
calls. For read calls we expect the set of hints to combine latency and
precision requirements. Additional hints will also be investigated. 

We will also develop a new type of querying system that allows application
to ask the storage system about completion timing information, the response
being an estimate provided by the storage system. We will explore how these
query functions can be integrated into common applications with minimal code
disruptions, and the set of policies that applications can use to respond to
this new information. We will also explore how applications can query the
storage system to gain insights on available space in different tiers and
how applications can tailor their output data to match the available space. 
Finally we will study the use of an external data annotation system that can
provide information to the storage system without requiring the application
to be recompiled. 

\paragraph{Challenges}
The design of new APIs posses both technical and adoption challenges. Here
we will only consider the technical challenges. The proposed set of APIs aim
to expose broad system level environmental information to the application,
and allow the application to adapt dynamically based on this information.
One aspect of this challenge is the design of an interrogative API that
enables a back-and-forth between the application and the storage system to
enable the application to recieve enough information to make adaptation
decisions. 

%%% Local Variables:
%%% mode: latex
%%% TeX-master: "../proposal"
%%% End:
