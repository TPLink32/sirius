\subsubsection{QOS in the Application}
%proposed idea
A key insight in this proposed project has been the increased interaction of
the application with the storage system. Towards this end, we propose to
address the quality of service requirements by porivding the application
with mechanisms to specify the quality of I/O service, interrogate the
storage system, and react to the responses. Guided by our past work in
ADIOS\cite{lofstead2008flexible} we propose to explore the design of two
mechanisms for this purpose. Firstly, we will explore the augmentation of
I/O applictation programing interfaces (I/O APIs) to allow applications to
both specify timing and quality information and also query the storage
system for timing estimates. Through this interface we expect the
application and user to gain insights into how long a single output or input
call will take given the required quality information, and then react to
these estimates by adjusting the quality or restricting the scope of the
data required. Likewise we will explore how an application can provide
timing information to the storage system to allow the storage system to make
optimization decisions to best meet the requirements from the application. 
%
Secondly, we will also explore external data annotations, such as those
provided by the configuration file in ADIOS. Through the use of these
external augmentation the user can provide insights to the storage system on
the relative value of the data, expected life time and performance
characteristics, as well as relationships between different data sets. With
this information the storage system can make optimizations specific to a use
case. We expect these augmentations to be particularly important for
eviction of data from a storage layer, and migration of data sets to a
different storage layer. 

%execution




%%% Local Variables:
%%% mode: latex
%%% TeX-master: "../proposal"
%%% End:
