\subsubsection{Data Reorganization Techniques}
One of the key element to get efficient data access during reading is the re-organization of the
data to ensure that the data that will be read back will involve minimal seeks on the file system 
along with aggregating similar data together to get optimal performance and placing the data
across multiple disks to get high levels of concurrency during reading. We will create APIs to
allow users to group similar data together in a data-model, and annotations (initially done in the
ADIOS layer) to allow relationships of variables to be expressed. We will then investigate both
system (general) data-re-organization techniques based on the utility of data and the relationships
of the data, and application-specific routines (for our initial set of evaluation test cases).  Once
we can generalize the application specific routines to a set of motifs, we can then allow this piece
of code to propagate to the general routines. In all cases we will investigate re-organzing the data
into an integer (most likely 8) bins, where they fall into the most likely accessed data to the least
likely accessed data.




\paragraph{State of the art:}
\paragraph{Proposed research approach:}
General methods to re-organize the data fall into the following categories
\begin{itemize}
\item Precision based re-organization. This is where the most significant bytes of the data are
all group together from each object. This data generally will have a higher utility then the data with
the least significant bytes.

\item Frequency-based re-organization. FFT, Wavelets.

\item Histogram binning. Re-organize the data through histograms to ensure we have enough
phase-space coverage. Where we have finner coverage we can move that data to other
quantiles of the storage. 

\item Multi-resolution. Traditionally we can look at every other point, and have a good feeling that 
this data will give a good estimate of the data.

\item Region of Interest. Using error techniques similar to AMR, we can understand where there
is data changing more rapidly and save this data at a higher resolution version than the smoother
regions of space.
{\bf Mark: can you fill these out much more and talk about the new techniques we will work on}.
\end{itemize}
\paragraph{Preliminary work and results:}
Text here
%%% Local Variables:
%%% mode: latex
%%% TeX-master: "../proposal"
%%% End:
