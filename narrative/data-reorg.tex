%\subsubsection{Data Reorganization Techniques}
One of the key element to get efficient data access during reading is the re-organization of the
data to ensure that the data that will be read back will involve minimal seeks on the file system 
along with aggregating similar data together to get optimal performance and placing the data
across multiple disks to get high levels of concurrency during reading. We will create APIs to
allow users to group similar data together in a data-model, and annotations (initially done in the
ADIOS layer) to allow relationships of variables to be expressed. We will then investigate both
system (general) data-re-organization techniques based on the utility of data and the relationships
of the data, and application-specific routines (for our initial set of evaluation test cases).  Once
we can generalize the application specific routines to a set of motifs, we can then allow this piece
of code to propagate to the general routines. In all cases we will investigate re-organizing the data
into an integer (most likely 8) bins, where they fall into the most likely accessed data to the least
likely accessed data. These methods will NOT be used for checkpoint restart data, since this
data needs to be accessed exactly how it is originally in memory.  We generally see that these
techniques are highly useful for analysis and  visualization data, where users can inherently place
their intentions of the data (life time of the data for eviction policies, prioritization of data 
importance). 


\paragraph{State of the art:}

\paragraph{Proposed research approach:}
General methods to re-organize the data fall into the following categories
\begin{itemize}
\item Precision based re-organization. This is where the most significant bytes of the data are
all group together from each object. This data generally will have a higher utility then the data with
the least significant bytes.  This process has the following three steps which need to be
efficiently implemented when data gets placed to the storage system and from the storage 
system. 
%
First, the data needs to be re-arranged which involves memory operations and needs
to be done in-situ. This operation involves no communication.  Data needs to be copied, so
the memory requirements are increased. At most this requires two copies an individual dataset
in memory, and we will investigate techniques to allocate and deallocate this memory if the user
will specify that the data will be overwritten after it is written to the storage system. This is often
the case for many of the data quantities written form the simulation, but there are many cases
where, for example, we want to write all of the particles from a Particle In Cell (PIC) simulation. 
Since the particles will be used later in the calculation, we need to duplicate the storage. Our
observation with working with the XGC1, GTC, Warp, and PiconGPU simulations (all leadership
class simulations) is that we can temporary increase the storage of the particles, and then 
release them since the temporary arrays used in the calculations are often freed when a PIC
iteration is finished.. 
%
The second step to give the metadata that allows the system to understand that an object is now contained in multiple bins of data. One of the challenges is the ability to bring together multiple bins
of data together (the high priority data along with the medium priority data) efficiently when the 
data is being read.  One research question is how we can manage the metadata efficiently.
We propose to place this knowledge into the middleware which will communicate this to the 
storage system.  This will give the storage system potential ways to increase the concurrency 
when accessing this information. For example, we can place the highest priority bits of an object
on the fastest storage system, and the next level of importance in the campaign storage. If a 
user request precision to a level where we need to access both, concurrently we can be
accessing this data, and moving the data to memory where it can be re-arranged. 
%
Since data will be placed in different bins along the hierarchy we will need to understand how
to initially place this in the fastest storage, and then evict this from the fastest storage without
effecting the performance of the simulation which is generating this data. Policies must be 
placed to ensure that we do not create any internal interference during these operations. We
will investigate doing this with ``meta-bots'' first introduced in the LWFS project and have 
these bots move the data.{\bf probably belongs in a different section} 

\item Frequency-based re-organization. FFT, Wavelets.
Another common technique to classify the importance of data is to first re-organize the data
in frequency space and then specify that as the frequency increases the importance of data
decreases.   Common technique which is used for streaming data, and data reduction techniques such as JPEG-2000~\cite{jpeg2000}
is to  code streams which have regions of interest that offer several mechanisms to support spatial random access or region of interest access at varying degrees of granularity. It is possible to store different parts of the same data using different quality.

This allows us to place the lowest frequency pieces in the fastest storage, and
the highest frequency either on the slowest storage tiers, or if the data sizes are prohibitively  
costly, not even write out the highest frequency. This binning of data in frequency space can
take as an input (storage bandwidth knowledge, storage size limitations, and user intentions
of how long the majority of the data will be frequently accessed. 

These operations also require data transformations (for example an FFT and an inverse FFT)
and then re-arranging the data in real-space once we re-organize this. The research questions
we must address are: 1) what is the cost of re-organzing the data, 2) how do we 

\item Histogram binning. Re-organize the data through histograms to ensure we have enough
phase-space coverage. Where we have finner coverage we can move that data to other
quantiles of the storage. 

\item Multi-resolution.  
A


\item Region of Interest. Using error techniques similar to AMR, we can understand where there
is data changing more rapidly and save this data at a higher resolution version than the smoother
regions of space.
{\bf Mark: can you fill these out much more and talk about the new techniques we will work on}.
\end{itemize}
\paragraph{Preliminary work and results:}
Text here
%%% Local Variables:
%%% mode: latex
%%% TeX-master: "../proposal"
%%% End:
