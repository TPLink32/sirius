\subsubsection{QoS}

\paragraph{Background:} ``Quality of Service'' (QoS) refers to the
properties of the performance qualtiy of a particular service.
Performance quality can be expressed in either relative or absolute
terms. Examples of relative performance qualities are ``fair'',
``proportional'', or ``priority/class-based'' (``the higher the
priority the better the service'' or ``1st class is better than
2nd''). The key advantage of relative performance qualities is that
they are easy to implement. The key disadvantage is the difficulty
of making any strong guarantees based on those implementations --
even relative guarantees are mired by the well-known effect of
priority inversion~\cite{lampson:cacm80}. Examples of absolute
performance qualities are ``rate-based'', ``soft real-time'', and
``hard real-time''. The key advantage of absolute performance
qualities is that it is principally easier to implement strong
guarantees: they enable an agreement between a task and a service,
also known as a \emph{reservation}, does not change meaning depending
on the service's state, as long as the service is able to isolate
performance between different tasks. The key disadvantage of absolute
performance qualities is that their implementation is non-trivial,
especially in large-scale systems.

\paragraph{Approach:} The goals of this proposal require applications
to demand guarantees in terms of absolute performance qualities.
The first key enabler for any service to make absolute guarantees
is \emph{performance isolation}, i.e. the ability to shield the
performance impact of one task from another. The most straight-forward
but least flexible strategy is to simply cap the latency of each
task, irregardless of its completeness (see for
example~\cite{decandia:sosp07}). This strategy is applicable when
incomplete tasks still have value and missing results are either
not important or can be easily retrieved in subsequent tasks. A
more generally applicable strategy is to control the admission of
tasks to the system: \emph{admission control} maintains a
\emph{utilization model} that can predict the utilization of system
resources given a new task, estimates the current utilization of
the system, and decides whether the task can be admitted without
overloading the system. Thus, admission control avoids system
overload conditions that lead to hard-to-model chaotic behaviors.
Consequently, utilization models can frequently be approximated by
automatically calibrated linear models (see for
example~\cite{skourtis:hpdc12}).

Another essential component of performance isolation is the
\emph{charging model} which determines which task(s) should be
charged how much for a request. Performance isolation is implemented
by the charging model by ensuring that performance of a task only
depends on its reservation and its workload behavior (but not on
other workloads in the system). Each task has a budget based on its
reservation and spends it based on the cost of each request determined
by the charging model. A charging model determines the cost based
on the utilization model and the interactions a request can have
with other requests (e.g. a random request to spinning media
interrupting a stream of sequential requests).

Performance isolation (including its utilization and charging models)
critically depends on \emph{workload-independent metrics}. Even
though throughput and latency are frequently the most meaningful
metrics for an application, they are not workload-independent and
therefore are inadequate for measuring the utilization of a device
across all workloads, unless one accepts worst-case estimates that
can lead to gross underutilization of resources (e.g. the throughput
of random and sequential workloads in spinning media differ by
orders of magnitudes). A common workload-independent metric is
\emph{time utilization}, i.e. the amount of time a resource is
utilized in a given time interval. Time utilization has the advantage
of having an always defined maximum and therefore makes resources
fully reservable. Even though a workload-independent metric might
at first not appear meaningful for an application, given a particular
workload an application can discover the relationship between its
workload-dependent metric and the workload-independent metric.
Because of performance isolation, the application has to discover
this relationship only once.

\paragraph{Related Work:}

\paragraph{Challenges:} The deep and heterogenous memory and storage
hierarchy we are assuming for this proposal complicates the
relationship between workload-dependent and workload-independent
metrics: the performance of a task can significantly differ depending
on what levels of the hierarchy are involved. Furthermore, an
important goal of the proposed project is to enable applications
to reason about the trade-off between resolution and latency, adding
yet another dimension to how tasks are mapped to resources.

\begin{itemize}

\item A key challenge of admission control is scalability: the
decision of whether to admit a task could potentially depend on
global knowledge of the current utilization of every single resource
in the system. Scalability therefore depends on whether admission
control is able to accurately map a new task to a relatively small
set of resources which can quickly provide up-to-date utilization.
One approach might involve pseudo-random mapping that also load-balances
as a side-effect.

\item To offer latency/latency trade-offs, the system must be able
to quickly generate a number of data production plans, involving
different parts of the hierarchy and different data resolutions.
It will then use the utilization model to estimate the latency for
each production plan and the resolutions they can provide. Here
again, pseudo-random selection could reduce the number of resources
that would be involved, thereby increasing scalability.

\item The complexity of a utilization model involving the entire
storage hieararchy with 100,000s of devices is potentially daunting.
However, the utilization model can be simplified by modeling classes
of devices as well as classes of requests. In particular, each
device could restrict access to its content via a set of well-defined
methods with known, absolute performance properties.

\item Flash devices become inherently unpredictable when reads and
writes are mixed on the same device because of garbage collection.
While write latency can be easily hidden using asynchronous I/O,
hiding read latency is more difficult. By separating reads and
writes for each device, read latency becomes predictable. The
challenge is to minimize duplication of data, at least on fast
layers and leverage redundancy across layers.

\end{itemize}
