
%%% Local Variables:
%%% mode: latex
%%% TeX-master: "paper"
%%% End:

The storage environment targeted by SIRIUS is complex. Based on industry trends
and our own experiences with hardware projections for future HPC platforms, the
current model of adding only a burst buffer tier to the storage hierarchy will
be supplanted with a much more complex setup.

In addition to the new hardware affecting data storage opportunities, the
storage hierarchy will affect how data is later identified, located, and
retrieved. Important considerations are what metadata must be collected and
how the quality of service estimates are provided.

\subsection{Non-Volatile Memory}

The data refactoring tools SIRIUS incorporates assumes the existence of
multiple different storage technologies, each with different performance and
capacity limitations. The most interesting of these is Non-Volatile Memory
(NVM). We see the potential for NVM to be deployed in at least four distinct
machine locations, each with a motivating use case:

\begin{enumerate}

\item {\em Memory Bus} - By placing NVM on the memory bus, fast, local access
is assured. This is ideal for out of core processing workloads. While this is
not a primary use case targeted by SIRIUS, we would take advantage of this
resource, should it be available. The disadvantage of this location is the
difficulty with remote access. Using RDMA-based techniques can simplify
access, even if it still requires a block interface. High bandwidth, on package
RAM can eliminate much of the contention that would otherwise make RDMA use of
the memory-bus based NVRAM impractical.

\item {\em Node-Local SSD} - This data analytics as well as IBM Burst Buffer
model where each node has a local SSD to eliminate network and centralized
storage access overheads for read intensive workloads has other uses for the
HPC environment. As with the Memory Bus-based NVM, Scalable Checkpoint/Restart
(SCM)~\cite{moody:2010:scr} can make effective use of node-local storage for
both local and remote processes. However, SIRIUS aims for a more interesting
use. By acknowledging the future workflow components connected with the
simulation and where the system either has or will deploy those components,
pre-staging data directly where the analytics needs it can be accomplished.
SIRIUS will aid this scenario by reducing data sizes through data refactoring
or other techniques.

\item {\em In Compute Area} - The Cray Burst Buffer model where, for example,
a single node per rack is deployed as a NVM node with significant non-volatile
storage has dual use. The intent for current generation new machines like
Trinity at Sandia and Los Alamos, is to primarily serve as a fast cache for the
parallel file system. The goal is to accelerate checkpoint/restart operations
by making the writing operations as fast as possible while allowing spooling to
the parallel file system as necessary. The secondary use case, and the one
SIRIUS is most interested in, is using these nodes as data staging for workflow
usage. By moving from the node-local SSD model, potential performance impacts
from remote data access can be eliminated. The intereference effects on
inter-node, intra-application communication can also be avoided. Much like the
Node-local SSD case, SIRIUS will store only the important parts of refactored
data in these resources while asynchronously spooling the less important data
to the parallel file system directly. By ensuring direct access to the parallel
file system without going through these Burst Buffer nodes, two different,
simultaneous paths to storage exists. Noticing this extra bandwidth and seeking
an advantageous way to leverage this to help applications was a major impetus
for SIRIUS.

\item {\em In Parallel File System} - By placing NVM in the parallel file
system, the caching nature desired for the Trinity and similar machines can be
achieved, but at a data center level instead. In addition, the metadata
services can use this faster storage to accelerate metadata operations.

\end{enumerate}

With a system that offers NVM in all of these locations, potential bandwidth to
storage will be maximized only when all of these as well as existing
components, such as the parallel file system, can be used simultaneously. By
focusing on storing only the most important parts of a data set on the
expensive, close NVM resources and asynchrnously spooling the rest to slower,
cheaper, higher capacity storage, SIRIUS achieves the best possible storage
bandwidth while also accelerating time to scientific insight.

\subsection{Metadata}

The metadata required falls into three categories: data characteristics, data
location, and infrastructure characteristics.

\subsubsection{Data Characteristics}

Unlike a traditional POSIX-style file system, SIRIUS is using a richer
approach.  In addition to something related to a file name, information about
individual variables including array dimensions, the data sizes, dimensions of
each piece written, and data features are stored. The full range of this data
must be user defined and will differ from application to application. The
initial model for these characteristics are the data characteristics that
currently exist in ADIOS~\cite{lofstead:2009:adaptable}. The extensible nature
of the characteristics in ADIOS will directly relate to those in SIRIUS. The
indexing structure used in ADIOS will also translate directly. Each chunk
stored will have information about what it belongs to and what portion of that
whole it represents.

\subsubsection{Data Location}

By spreading a single data set across the entire storage hierarchy, where each
piece of data resides must either be tracked directly at the cost of
maintaining full consistency, or be found later when requested, at the cost of
the search. Which of these approaches or what hybrid using a partiular balance
of parameters will be explored as part of this project.

\subsubsection{Infrastructure Characteristics}

Almost most importnatly for the metadata are the infrastructure
characteristics. As part of providing quality of service, how the system
performs is critical.  As the system is used, each read or write to different
storage resources must be measured to generate a reasonable estimate for future
accesses under similar conditions. This is further explored in the quality of
service subsection.

\subsubsection{Quality of Service}

Quality of service deadlines are important for driving how much time is spent
performing IO. At the same time, data quality must be maintained at a
sufficient level for the resulting science to be possible.  The system must
offer a way for an end-user to negotiate with the system for trading off
between time spent in IO and the amount of data written or read.  Part of that
tradeoff will be increased or reduced data quality, depending on the system.
For example, when data is written, a few different forms may be output given
sufficient space and time. One may be every fourth element across a regular
mesh reduces the data size to $1/64^{th}$ the full data set.  Further dividing
that using byte-splitting can reduce it by another 75\%. For quick overviews to
search data sets, this reduced form can be sufficient driving which data sets
should be investigated in more detail. The key driving factor is time. The
applicaiton scientist can read 0.4\% of the full data volume with a similar
time reduction. SIRIUS will offer a negotiation process to read any data.

\begin{table}[tbp]
\centering
\caption{Reading Options for Variable ``A''}
\label{tab:results}
\begin{tabular}{|l|l|l|l|l|l|}
\hline
\# & Size & Time & Time Err & Refactoring & Data Err\\
\hline
1 & $(\frac{1}{4})^3(\frac{3}{8})$A & 10s & $\pm$ 3s & stride, byte-split & 98\% \\
\hline
2 & $(\frac{3}{4})^3(\frac{3}{8})$A & 90s & $\pm$ 30s & stride, byte-split & 58\% \\
\hline
3 & $(\frac{1}{4})^3(\frac{5}{8})$A & 16s & $\pm$ 5s & stride, byte-split & n/a \\
\hline
4 & $(\frac{3}{4})^3(\frac{5}{8})$A & 120s & $\pm$ 50s & stride, byte-split & n/a \\
\hline
\hline
5 & $(\frac{1}{4})^3$A & 1200s & $\pm$ 30s & stride & 98\% \\
\hline
6 & $(\frac{3}{4})^3$A & 2400s & $\pm$ 90s & stride & 58\% \\
\hline
\hline
7 & $(\frac{3}{8})$A & 1350s & $\pm$ 120s & byte-split & 0.02\% \\
\hline
8 & $(\frac{5}{8})$A & 2250s & $\pm$ 120s & byte-split & n/a \\
\hline
\hline
9 & A & 36s & $\pm$ 6s & wavelet & 1\% \\
\hline
\hline
10 & A & 3600s & $\pm$ 600s & none & 0\% \\
\hline
\end{tabular}
\end{table}

The byte splitting approach error rate is quite small. By splitting at byte
boundaries rather than converting to a lower precision 4-byte float, we
preserve the full exponent range and keep the first 11 bits of mantissa
precision. This leaves a $2^{-12}$ maximum error magnitude for each value--the
last 40 bits represent less precision together than the immediately preceeding
bit all by itself.

To read these various data organizations, read operations will shift to a
two-step process. First, an list of options will be requested that includes
both and error estiates. A second request will retrieve a particular data
portion from the provided list. Consider Table~\ref{tab:results}.  First, a
user will request a particular variable and timesetp. SIRIUS will return a
table similar to Table~\ref{tab:results}. For a given variable ``A'',
the end-user can then select the data quality and approximate reading time for
the operation. In this case, there are 5 options for obtaining the whole data
set (1/2/3/4, 5/6, 7/8, 9, 10). The first four entries represent a spatial and
byte split, 5/6 represent just strided writing, 7/8 represent just a
byte-split, 9 represents a wavelet compressed version, and 10 is the full
precision data.  The user would specify to read a particular data instance that
would proceed normally.  Quality of Service soft guarantees specified as part
of the request will result in SIRIUS allocating soft bandwidth reservations to
meet the request. While some of the error rates seem to render the data
unusable, consider a 3000x3000x3000 simulation domain. That would yield a 27
gigapixel image should all of the data be used. Visualization will already
radically reduce the data sizes to make usable visualizations that work on a
standard display. A 99\% reduction would yield a 270 megapixel image--still
quite large.
