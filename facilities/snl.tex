\facilitiessection{Sandia National Laboratories}{snl}

\section*{Sandia National Laboratories Computer Science Research Institute}

The mission of
the Computer Science Research Institute (CSRI) is to conduct collaborative
research and development with external researchers (particularly university
faculty) in the areas of computer science and mathematics to build Sandia’s and
DOE’s modeling and simulation capabilities. The CSRI is funded by the
University Programs element within the Advanced Simulation and Computing (ASC)
program is managed on behalf of Sandia by the Computers, Computation,
Mathematics and Information Center. 

The CSRI has established research in the areas of computer and computational
science critical to the laboratories’ and DOE’s mission. They work to identify
external research collaboration that will impact DOE objectives and provide
funds for work through university researchers (faculty and students) working at
Sandia.  The CSRI projects respond to the need for computer science and applied
mathematics research, with a portfolio of exploratory R\&D activities, which
will provide computational-enabling technologies for Sandia code development
and applications. 

The CSRI provides a mechanism by which university researchers
learn about problems in computer and computational science at DOE laboratories,
conduct leading-edge research, interact with scientists and engineers at the
laboratories, and help transfer the results of their research to programs at
the laboratories. The CSRI also enables the laboratories to maintain and expand
the computer science and mathematics expertise required for DOE projects and
initiatives to be successful. The CSRI is a focal point, both physically and in
terms of research collaborations, for university researchers, students, and
laboratory staff engaged in computer and computational science, modeling, and
simulation. This project will leverage ongoing CSRI activities to foster
additional collaborations and interactions with academic, industry, and lab
researchers as appropriate.  The CSRI is housed in a 34,500 ft2 facility that
opened in 2006.  It houses approximately 190 researchers comprising 140
full-time staff and 50 visitors and includes conference rooms and collaborative
work areas. The CSRI facility has significant local computing power with high-
bandwidth network connection to large-scale Sandia computing platforms.

The general Sandia computing environment contains a large number of diverse systems from PC’s to workstations, several different Linux cluster systems and supercomputers. This includes the 264 TF Red Sky system (22,528 2.93 GHz Intel Nehalem X5570 cores with InfiniBand) and the 38 TF Glory system (4,352 2.2 GHz AMD Opteron cores with InfiniBand). Sandia CSRI researchers on this project also have access to the 1.374 PF ACES Cielo platform at Los Alamos (Cray XE6 with 143,104 2.4 GHz AMD Opteron cores).
In addition to these production machines, Sandia CSRI system software researchers have access to several experimental architecture technology development testbed platforms, including an Intel MIC architecture testbed (42 nodes connected with QDR InfiniBand, each node contains two six-core 3.46 GHz Intel Westmere processors and two 30-core 1.05 GHz Knights Ferry co-processors), an AMD Fusion testbed (104 nodes connected with QDR InfiniBand, each node contains one Llano Fusion APU consisting of a four-core 2.9 GHz AMD K10 and a 600 MHz 400-core Radeon HD 6550D and a 256-GB Micron MLC flash drive), a Convey testbed (one Convey GV HC-1EX board containing a 2.13 GHz quad-core Intel Nehalem, 4 Xilinx Vertex6 LX760 FPGA co-processors, and 8 Xilinx FGPA programmable memory controllers), and a Tilera streaming mulit-core testbed (four Tile-Gx8036 tiles, each containing 36 1.2 GHz cores).

